\documentclass[12pt, answers]{exam}
\usepackage{amsfonts}
\usepackage{amsmath}
\usepackage[italian]{babel}
\usepackage[utf8]{inputenc}
\usepackage{embedall}
\usepackage{etoolbox}
\AtBeginEnvironment{align}{\setcounter{equation}{0}}
\embedfile{\jobname.tex}
\usepackage{graphicx}
\usepackage{listings}
\usepackage{tabularray}
\usepackage{tikz}
\usetikzlibrary{positioning, automata, decorations.pathreplacing, calligraphy}
\usepackage{xcolor}
\usepackage[bookmarks]{hyperref}

\definecolor{codegray}{gray}{0.95}
\lstdefinestyle{mystyle}{
	numberstyle=\scriptsize\color{black},
	basicstyle=\footnotesize,
	numbers=left,
	numbersep=5pt,
	}
	\lstset{style=mystyle}
	\overfullrule=10pt

	\newcommand{\grammar}[4]{\langle #1, #2, #3, #4 \rangle}
	\newcommand{\grammarP}[4]{\langle \{ #1 \}, \{ #2 \}, \{ #3 \}, #4 \rangle}
	\newcommand{\der}{\Rightarrow}
	\newcommand{\prd}{\rightarrow}
	\begin{document}
	\begin{questions}
		\question Costruite un automa a stati finiti che riconosca il linguaggio formato da tutte le stringhe sull'alfabeto $\{a, b\}$ nelle quali ogni $a$ è seguita immediatamente da una $b$.
		\begin{solution}
			\begin{center}
				\begin{tikzpicture}
					\node[state, initial, accepting] 	(q0) {$q_0$};
					\node[state] 				(q1) [right=of q0] {$q_1$};

					\path[->] (q0)	edge[loop above]	node[above]	{b}	()
					edge[bend left]		node[above]	{a}	(q1)
					(q1)	edge[bend left]		node[below]	{b}	(q0);
				\end{tikzpicture}
			\end{center}
		\end{solution}
		\question Costruite un automata a stati finini che riconosca il linguaggio formato da tutte le stringhe sull'alfabeto $\{4, 5\}$ che, interpretate come numeri in base 10, rappresentano numeri interi che non sono divisibili per 3.
		\begin{solution}
			\begin{center}
				\begin{tikzpicture}[state/.style=state with output]
					\node[state, initial]	(q0) {$q_0$ \nodepart{lower} $00$};
					\node[state] 		(q1) [right=of q0] {$q_1$ \nodepart{lower} $10$};
					\node[state] 		(q2) [below=of q0] {$q_2$ \nodepart{lower} $01$};
					\node[state, accepting]	(q3) [right=of q2] {$q_3$ \nodepart{lower} $11$}; 

					\path[->] (q0)	edge[bend left]	node[above] {\tiny 4} (q1)
					edge[bend left] node[right] {\tiny 5} (q2)
					(q1)	edge[bend left] node[below] {\tiny 4} (q0)
					edge[bend left] node[right] {\tiny 5} (q3)
					(q2) 	edge[bend left] node[left]  {\tiny 5} (q0)
					edge[bend left] node[above] {\tiny 4} (q3)
					(q3) 	edge[bend left] node[left]  {\tiny 5} (q1)
					edge[bend left] node[below] {\tiny 4} (q2);
				\end{tikzpicture}
			\end{center}
		\end{solution}
		\question Costruite un automa a stati finiti deterministico che riconosca il linguaggio formato da tutte le stringhe dell'alfabeto $\{0, 1\}$, che, interpreate come numeri in notazione binaria, denotano multipli di 4. Utilizzando il nondeterminismo si riesce a costruire una automa con meno stati? Generalizzate l'esercizio a multipli di $2^k$, dove $k > 0$ è un intero fissato
		\begin{solution}
			\begin{center}
				\begin{tikzpicture}
					\node[state, initial]	(q0) {$q_0$};
					\node[state] 		(q1) [right=of q0] {$q_1$};
					\node[state, accepting]	(q2) [right=of q1] {$q_2$};

					\path[->] (q0)	edge[loop above]	node[above] {\tiny 1} ()
					edge[bend left]		node[above] {\tiny 0} (q1)
					(q1)	edge[bend left]		node[above] {\tiny 0} (q2)
					edge[bend left]		node[below] {\tiny 1} (q0)
					(q2)	edge[loop above]	node[above] {\tiny 0} ()
					edge[bend left]		node[below] {\tiny 1} (q0);
				\end{tikzpicture}
			\end{center}
			Questo, non deterministico, diventa
			\begin{center}
				\begin{tikzpicture}
					\node[state, initial]	(q0) {$q_0$};
					\node[state] 		(q1) [right=of q0] {$q_1$};
					\node[state, accepting]	(q2) [right=of q1] {$q_2$};

					\path[->] (q0)	edge[loop above]	node[above] {\tiny 0, 1} ()
					edge			node[above] {\tiny 0} (q1)
					(q1)	edge			node[above] {\tiny 0} (q2);
				\end{tikzpicture}
			\end{center}
			Questo può essere generalizzato ai multipli di $2^k$ come
			\begin{center}
				\begin{tikzpicture}
					\node[state, initial]	(q0) 			{$q_0$};
					\node[state] 		(q1) 	[right=of q0] 	{$q_1$};
					\node	 		(dot) 	[right=of q1] 	{$\dots$};
					\node[state, accepting]	(qk) 	[right=of dot]	{$q_k$};

					\draw[thick, decorate, decoration = {brace, mirror}] ([yshift=-0.65cm] q1.south) -- ([yshift=-0.65cm] qk.south) node[pos=0.5, below] {\tiny $k$ volte};
					\path[->] (q0)	edge[loop above]	node[above] {\tiny 0, 1} ()
					edge			node[above] {\tiny 0} (q1)
					(q1)	edge			node[above] {\tiny 0} (dot)
					edge[bend left]		node[above] {\tiny 1} (q0)
					(dot) edge			node[above] {\tiny 0} (qk)
					edge[bend left]		node[above] {\tiny 1} (q0)
					(qk)	edge[loop above]	node[above] {\tiny 0} ()
					edge[bend left]		node[above] {\tiny 1} (q0);

				\end{tikzpicture}
			\end{center}
		\end{solution}
		\question Costruite un automa a stati finiti che riconosca il linguaggio formato da tutte le stringhe sull'alfabeto $\{0, 1\}$ che, interpretate come numeri in notazione binaria, rappresentano multipli di 5.
		\begin{solution}
			% divisibility by five automaton (smiley face)
		\end{solution}

		\question Considerate il seguente linguaggio:
		$$ L = \{ w \in \{a, b\}^* \mid \text{il penultimo e il terzultimo simbolo di $w$ sono uguali} \} $$
		\begin{parts}
			\part Costruite un automa a stati finiti deterministico che accetta $L$
			\begin{solution}
				\begin{center}
					\begin{tikzpicture}
						\node[state, initial]   (eps)   		{$\epsilon$};
						\node[state] 		(aba) 	[below left=of eps] 	{$aba$};
						\node[state, accepting]	(aab) 	[below=of aba] 	{$aab$};
						\node[state] 		(bab) 	[below right=of eps] 	{$bab$};
						\node[state] 		(abb) 	[below=of bab] 	{$abb$};
						\node[state] 		(baa) 	[below left=of aba] 	{$baa$};
						\node[state, accepting]	(aaa) 	[below=of aab]		{$aaa$};
						\node[state, accepting]	(bba) 	[below right=of bab] 	{$bba$};
						\node[state, accepting]	(bbb) 	[below=of abb] 	{$bbb$};

						\path[->] (eps)	edge[bend right] node[left]  {\tiny a} (aba)
						edge[bend left]  node[right] {\tiny b} (bab)
						(aba) edge[bend left]  node[above] {\tiny b} (bab)
						edge[bend right] node[right] {\tiny a} (baa)
						(bab)	edge[bend left]  node[below] {\tiny a} (aba)
						edge		 node[left]  {\tiny b} (abb)
						(baa)	edge		 node[below] {\tiny b} (aab)
						edge[bend right] node[left]  {\tiny a} (aaa)
						(abb)	edge		 node[below] {\tiny a} (bba)
						edge		 node[right] {\tiny b} (bbb)
						(aaa) edge[loop below] node[below] {\tiny a} ()
						edge             node[below] {\tiny b} (aab)
						(bbb) edge[loop below] node[below] {\tiny b} ()
						edge[bend right] node[below] {\tiny a} (bba)
						(aab) edge             node[left]  {\tiny a} (aba)
						edge		 node[above] {\tiny b} (abb)
						(bba) edge[bend right] node[right] {\tiny b} (bab);

						\draw[->] (bba) .. controls ([yshift=-3cm, xshift=4cm]bbb.south) and ([yshift=-3cm, xshift=-4cm]aaa.south) .. node[below] {\tiny a} (baa);
					\end{tikzpicture}
				\end{center}
			\end{solution}
			\part Costruite un automa a stati finiti nondeterministico che accetta $L$
			\begin{solution}
				\begin{center}
					\begin{tikzpicture}
						\node[state, initial] 	(q0) {$q_0$};
						\node[state] 		(q1) [above right=of q0] {$q_1$};
						\node[state] 		(q2) [below right=of q0] {$q_2$};
						\node[state] 		(q3) [right=of q1] {$q_3$};
						\node[state] 		(q4) [right=of q2] {$q_4$};
						\node[state, accepting] (q5) [right=of q3] {$q_5$};
						\node[state, accepting] (q6) [right=of q4] {$q_6$};

						\path[->]
						(q0)	edge[loop above]	node[above]	{a, b} 	()
						edge			node[above]	{a}	(q1)
						edge			node[below]	{b}	(q2)
						(q1)	edge			node[above]	{a}	(q3)
						(q2)	edge			node[below]	{a}	(q4)
						(q3)	edge			node[above]	{a,b}	(q5)
						(q4)	edge			node[below]	{a,b}	(q6);
					\end{tikzpicture}
				\end{center}
			\end{solution}
			\part Dimostrare che per il linguaggio $L$:
			\begin{itemize}
				\item tutte le stringhe di lunghezza 3 sono distinguibili tra loro
				\item la parola vuota è distinguibile da tutte le stringhe di lunghezza 3
			\end{itemize}
			\begin{solution}
				Per tutte le stringhe di lunghezza 3 costruiamo la seguente tabella.
				\begin{center}
					\begin{tblr}{ c | c c c c c c c c }
						z          & aaa & aab & aba        & abb        & baa        & bab        & bba & bbb \\
						\hline
						aaa        &     & a   & $\epsilon$ & $\epsilon$ & $\epsilon$ & $\epsilon$ & a          & aa         \\
						aab        &     &     & $\epsilon$ & $\epsilon$ & $\epsilon$ & $\epsilon$ & ba         & ba         \\
						aba        &     &     &            & a          & a          & bb         & $\epsilon$ & $\epsilon$ \\
						abb        &     &     &            &            & bb         & b          & $\epsilon$ & $\epsilon$ \\
						baa        &     &     &            &            &            & a          & $\epsilon$ & $\epsilon$ \\
						bab        &     &     &            &            &            &            & $\epsilon$ & $\epsilon$ \\
						bba        &     &     &            &            &            &            &            & aa         \\
						bbb        &     &     &            &            &            &            &            &            \\
					\end{tblr}
				\end{center}

				Mentre per la parola vuota cosrtruiamo
				\begin{center}
					\begin{tblr}{ c | c c c c c c c c }
						z          & aaa        & aab        & aba & abb & baa & bab & bba        & bbb \\
						\hline
						$\epsilon$ & $\epsilon$ & $\epsilon$ & aa  & b   & a   & bb  & $\epsilon$ & $\epsilon$
					\end{tblr}
				\end{center}
				È facile invece notare che le stringhe di lunghezza uno e due non sono distinguibili 
				\begin{center}
					\begin{tblr}{ c | c c c c c c c c }
						z          & aaa            & aab        & aba            & abb            & baa              & bab            & bba        & bbb \\
						\hline
						a          & $\epsilon$     & $\epsilon$ & \color{red}{?} & b              & a                & ba             & $\epsilon$ & $\epsilon$ \\
						b          & $\epsilon$     & $\epsilon$ & ab             & b              & a                & \color{red}{?} & $\epsilon$ & $\epsilon$ \\
						aa         & $\epsilon$     & $\epsilon$ & a              & aa             & \color{red}{?}   & a              & $\epsilon$ & $\epsilon$ \\
						ab         & $\epsilon$     & $\epsilon$ & bb             & a              & a                & \color{red}{?} & $\epsilon$ & $\epsilon$ \\
						ba         & $\epsilon$     & $\epsilon$ & \color{red}{?} & a              & a                & bb             & $\epsilon$ & $\epsilon$ \\
						bb         & $\epsilon$     & $\epsilon$ & b              & \color{red}{?} & bb               & a              & $\epsilon$ & $\epsilon$ \\
					\end{tblr}
				\end{center}
			\end{solution}
			\part Utilizzando i risultati precedenti, ricavate un limite inferiore per il numero di stati di ogni automa deterministico che accetta $L$
			\begin{solution}
				Dai risultati precedenti si può capire che l'automa necessita di almeno $2^3 + 1$ stati.
			\end{solution}
		\end{parts}
		\question Costruite un insieme di stringhe distinguibili tra loro per ognuno dei seguenti linguaggi:
		\begin{parts}
			\part $ \{ w \in \{a, b\}^* \mid \#_a(w) = \#_b(w) \} $
			\begin{solution}
				È facile vedere che tutte le parole in $\{a^n \mid n \geq 0 \}$ sono tra loro distinguibili, infatti
				\begin{center}
					\begin{tblr}{ c | c c c }
						z   & a & aa & aaa & aaaa & \dots \\
						\hline
						a   &   & bb & bbb & bbbb & \dots \\
						aa  &   &    & bbb & bbbb & \dots \\
						aaa &   &    &     & bbbb & \dots 
					\end{tblr}
				\end{center}
				Quindi l'insieme delle stringhe distinguibili $X = \{ a^n \mid n \geq 1 \} \cup \{ b^n \mid n \geq 1 \} \cup \{\epsilon\}$.
				Infatti per ogni stringa $w \in L$, 
				$$
				\begin{cases} 
					w \sim a^n & \text{se } \#_a(w) > \#_b(w) \wedge \#_a(w) - \#_b(w) = n \\
					w \sim b^n & \text{se } \#_a(w) < \#_b(w) \wedge \#_b(w) - \#_a(w) = n \\
					w \sim \epsilon & \text{altrimenti}
				\end{cases}
				$$
			\end{solution}
			\part $ \{ a^n b^n \mid n \geq 0 \}$
			\begin{solution}
				Qui si può fare un ragionamento simile a prima, quindi ragionando ancora sulle parole $\{ a^n \mid n \geq 0 \}$
				\begin{center}
					\begin{tblr}{ c | c c c }
						z   & a & aa & aaa & aaaa & \dots \\
						\hline
						a   &   & bb & bbb & bbbb & \dots \\
						aa  &   &    & bbb & bbbb & \dots \\
						aaa &   &    &     & bbbb & \dots 
					\end{tblr}
				\end{center}
				Queste sono ancora tutte distinguibili, quindi $X = \{a^n \mid n \geq 0 \}$.
				Possiamo infatti vedere questo linguaggio come un caso speciale del superiore.
			\end{solution}
			\part $ \{ w w^R \mid w \in \{a, b\}^* \} $ dove, per ogni stringa $w$, $w^R$ indica la stringa $w$ scritta al contrario
			\begin{solution}
				L'insieme delle stringhe distinguibili per questo linguaggio è esattamente $X = \{a, b\}^*$.
				Infatti una stringa $w$ determina univocamente il suo inverso $w^R$.
			\end{solution}
		\end{parts}
		Per alcuni di questi linguaggi riuscite ad ottenere insiemi di stringhe distinguibili di cardinalità infinita? Cosa significa ciò?
		\begin{solution}
			Tutti gli insiemi per questi linguaggi sono infiniti, quindi non possono essere riconosciuti da automi a stati finiti.
		\end{solution}
		\question Considerate l'automa di Meyer\&Fischer $A_n$ presentato nella Lezione 4 (caso peggiore della costruzione per sottoinsiemi) e mostrato nella seguente figura:
		% manca
		Descrivete a parole la proprietà che deve soddisfare una stringa per essere accettata da $A_n$.
		Riuscite a costruire un automa non deterministico, diverso da $A_n$, per lo stesso linguaggio, basandovi su tale proprietà?
		\begin{solution}
			Qualsiasi che finisce con multipli di $n$ $a$, oppure con un numero qualsiasi di $b$.
		\end{solution}

	\end{questions}
\end{document}

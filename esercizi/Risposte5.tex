\documentclass[12pt, answers]{exam}
\usepackage{amsfonts}
\usepackage{amsmath}
\usepackage[italian]{babel}
\usepackage[utf8]{inputenc}
\usepackage{embedall}
\usepackage{etoolbox}
\AtBeginEnvironment{align}{\setcounter{equation}{0}}
\embedfile{\jobname.tex}
\usepackage{graphicx}
\usepackage{listings}
\usepackage{tabularray}
\usepackage{tikz}
\usetikzlibrary{positioning, automata, decorations.pathreplacing, calligraphy}
\usepackage{xcolor}
\usepackage[bookmarks]{hyperref}

\definecolor{codegray}{gray}{0.95}
\lstdefinestyle{mystyle}{
	numberstyle=\scriptsize\color{black},
	basicstyle=\footnotesize,
	numbers=left,
	numbersep=5pt,
}
\lstset{style=mystyle}
\overfullrule=10pt

\newcommand{\grammar}[4]{\langle #1, #2, #3, #4 \rangle}
\newcommand{\grammarP}[4]{\langle \{ #1 \}, \{ #2 \}, \{ #3 \}, #4 \rangle}
\newcommand{\der}{\Rightarrow}
\newcommand{\prd}{\rightarrow}
\begin{document}
\begin{questions}
	\question Considerate il linguaggio
	$$ \operatorname{DOUBLE}_k = \{ ww \mid w \in \{ a, b \}^k \} $$
	dove $k > 0$ è un intero fissato.
	È abbastanza facile trovare un fooling set di cardinalità $2^k$ per questo linguaggio.
	Riuscite a trovare un fooling set o un extended fooling set di cardinalità maggiore?
	\begin{solution}
		Possiamo definire l'extended fooling set costituito 
		$$ X = \bigcup_{i = 0}^k \{a, b\}^i \times \{a, b\}^{k - i} $$
		Ad esempio definiamo questo insieme per $\text{DOUBLE}_2$, come
		\begin{align*}
			(\epsilon, aaaa), \\
			(a, aaa), (b, bbb), \\
			(aa, aa), (ab, ab), (ba, ba), (bb, bb), \\
			(aaa, a), (bbb, b), \\
			(bbbb, \epsilon)
		\end{align*}
		Questo contiene il set definito di sopra, e per ogni altra coppia possono avvenire due casi:
		\begin{itemize}
			\item i due elementi $x_i$ e $x_j$ hanno lunghezza diversa, allora per forza $x_i y_j$ non sarà in $\text{DOUBLE}_k$, perché sarà o troppo breve o troppo lungo
			\item i due elementi avranno $x_i$ e $x_j$ di dimensione uguale, ma saranno per forza diversi, quindi la parola ancora non sarà in $\text{DOUBLE}_k$
		\end{itemize}
		Questo insieme ha cardinalità 
		$$ 2^k + \sum_{i = 0}^{k - 1} 2 \cdot 2^i $$
	\end{solution}
	\question Considerate il linguaggio
	$$ \text{PAL}_k = \{ w \in \{a, b\}^k \mid w = w^R \} $$
	dove $k$ è un intero fissato.
	Qual è l'extended fooling set per $\text{PAL}_k$ di cardinalità maggiore che riuscite a trovare?
	\begin{solution}
		Questo è definito similmente a quello di sopra
		???
		Ed ha lo stesso numero di membri
		$$ 2^k + \sum_{i = 0}^{k - 1} 2 \cdot 2^i $$
	\end{solution}

	\question Considerate il linguaggio
	$$ K_k = \{ w \mid w = x_1 \cdot x_2 \dots x_m \cdot x \mid m > 0, x_1, \dots, x_m, x \in \{a, b\}^k, \exists i, 1 \leq i \leq m, x_i = x\}$$
	dove $k$ è un intero fissato.
	Si può osservare che ogni stringa $w$ di questo linguaggio è la concatenazione di blocchi di lunghezza $k$, in cui l'ultimo blocco coincide con uno dei blocchi precedenti.
	\begin{parts}
		\part Riuscite a costruire un (extended) fooling set di cardinalità $2^k$ o maggiore per il linguaggio $K_k$?
		\textit{Suggerimento: ispiratevi all'esercizio 1}
		\begin{solution}
			Si può generare il fooling se
			$$ X = \{ (x, x) \mid x \in \{a, b\}^k \} $$
			Questo ha dimensione $2^k$ ed è facile notare che ogni parola $xx$ appartiene a $K_k$, mentre per due indici diversi $xy$ non appartiene.
		\end{solution}
		\part Quale è l'informazione principale che un automa nondeterministico può scegliere di ricordare nel proprio controllo a stati finiti durante la lettura di una stringa per riuscire a riconoscere $K_k$?
		\textit{Suggerimento: Che ``scommessa'' può fare l'automa mentre legge la stringa in ingresso e come può verificare tale scommessa leggendo l'ultimo blocco?}
		\begin{solution}
			L'insieme di tutte le possibili stringhe di $k$ elementi nel linguaggio è finito, quindi l'automa si deve ricordare quale sottoinsieme di queste stringhe ha visto, quindi $2^{\{a, b\}^k}$.
			L'automa poi deve scegliere in uno stato in cui ha visto un certo sottoinsieme di stringhe da $k$ elementi, se la prossima che vede è la finale o se ritornare in sé stesso.
		\end{solution}
		\part Supponenete di costruire un automa deterministico per riconoscere $K_k$.
		Cosa ha necessità di ricordare l'automa nel proprio controllo a stati finiti mentre legge la stringa in input?
		\part Utilizzando il concetto di distinguibilità, dimostrate che ogni automa deterministico che riconosce $K_k$ deve avere almeno $2^{2^k}$ stati.
	\end{parts}
	\question Considerate il linguaggio
	$$ J_k = \{ w \mid w = x \cdot x_1 \cdot x_2 \dots x_m \mid m > 0, x_1, \dots, x_m, x \in \{a, b\}^k, \exists i, 1 \leq i \leq m, x_i = x\} $$
	dove $k$ è un intero fissato.
	Si può osservare che ogni stringa $w$ di questo linguaggio è la concatenazione di blocchi di lunghezza $k$, in cui il primo blocco coincide con uno dei blocchi successivi; ogni stringa di $J_k$ si ottiene ``rovesciando'' una stringa del linguaggio $K_k$ dell'esercizio 3.
	Supponete di costruire automi a stati finiti per $J_k$.
	Valgono ancora gli stessi limiti inferiori per $K_n$ o si riescono a costruire automi più piccoli?
	Rispondete sia nel caso di automi deterministici sia in quello di automi nondeterministici.
	\begin{solution}
		Non serve più ricordare gli stati visti, ma serve solo ricordare lo stato iniziale e accertarsi che ci sia un percorso che lo ripete, quindi la dimensione dell'automa (deterministico) sarà $2^k \cdot o(k)$.
	\end{solution}
	\question Ispirandovi all'esercizio 3, fornite limiti inferiori per il numero di stati degli automi che riconoscono il seguente linguaggio:
	$$ E_k = \{ w \mid w = x_1 \cdot x_2 \dots x_m \mid m > 0, x_1, \dots, x_m \in \{a, b\}^k, \exists i, j, 0 \leq i < j \leq m, x_i = x_j \} $$
	dove $k$ è un intero fissato.
	Considerate sia il caso deterministico che quello nondeterministico.
	\begin{solution}
		Nel caso non deterministico questo automa è molto simile a quello precedente, tranne che prima si ammettono una coppia non definita di blocchi di $k$ simboli.
		La scommessa è scegliere la coppia che si ripeterà.

		% CASO DETERMINISTICO???
	\end{solution}
\end{questions}
\end{document}

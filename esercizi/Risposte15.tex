\documentclass[12pt, answers]{exam}
\usepackage{amsfonts}
\usepackage{amsmath}
\usepackage{amssymb}
\usepackage[italian]{babel}
\usepackage{caption}
\usepackage[utf8]{inputenc}
\usepackage{embedall}
\usepackage{enumitem}
\usepackage{etoolbox}
\usepackage{float}
\AtBeginEnvironment{align}{\setcounter{equation}{0}}
\embedfile{\jobname.tex}
\usepackage{graphicx}
\usepackage{listings}
\usepackage{mathtools}
\usepackage{subcaption}
\usepackage{tabularray}
\usepackage{tikz}
\usetikzlibrary{positioning, automata, decorations.pathreplacing, calligraphy}
\usepackage{xcolor}
\usepackage[bookmarks]{hyperref}

\definecolor{codegray}{gray}{0.95}
\lstdefinestyle{mystyle}{
	numberstyle=\scriptsize\color{black},
	basicstyle=\footnotesize,
	numbers=left,
	numbersep=5pt,
}
\lstset{style=mystyle}
\overfullrule=10pt

\newcommand{\grammar}[4]{\langle #1, #2, #3, #4 \rangle}
\newcommand{\grammarP}[4]{\langle \{ #1 \}, \{ #2 \}, \{ #3 \}, #4 \rangle}
\newcommand{\der}{\Rightarrow}
\newcommand{\prd}{\rightarrow}

\footer{}{\thepage}{}
\begin{document}
Negli esercizi deguenti facciamo riferimento alla forma normale per gli automai a pila presentata a lezione, in cui un automa a pila è, come sempre, una 7-tupla 
$$M = \langle Q, \Sigma, \Gamma, \delta, q_0, Z_0, F\rangle$$
dove $Q$ è l'insieme finito degli \textit{stati}, $\Sigma$ è l'\textit{alfabeto di input}, $\Gamma$ è l'\textit{alfabeto della pila}, $q_0 \in Q$ è lo stato iniziale, $Z_0 \in \Gamma$ \textit{simbolo iniziale} sulla pila, $F \subseteq Q$ è l'insieme degli \textit{stati finali}.
Nella forma normale l'automa opera come segue:
\begin{itemize}
	\item all'inizio della computazione la pila contiene solo il simbolo $Z_0$, che non può essere mai rimosso o caricato sulla pila;
	\item l'input è accettato se e solo se l'automa raggiunge una configurazione in cui tutto l'input è stato letto, lo stato appartiene all'insieme $F$, la pila contiene solo $Z_0$;
	\item in una mossa è possibile caricare un simbolo in cima alla pila, rimuovere il simbolo che si trova in cima alla pila, oppure lasciare la pila inalterata;
	\item nelle mosse in cui l'automa legge un simbolo dall'input la pila non viene modificata.
\end{itemize}
La funzione di transizione $\delta$ di un automa a pila $M$ in questa forma può essere scritta come:
$$ \delta : Q \times (\Sigma \cup \{\varepsilon\} ) \times  \Gamma \rightarrow 2^{Q \times (\{-, \text{pop}\} \cup \{\text{push}(Z) \mid Z \in \Gamma\})} $$
con il seguente significato, per $q, p \in Q$, $A, B \in \Gamma$, $a \in \Sigma$:
\begin{itemize}
	\item $(p, -) \in \delta(q, a, A)$:\\
		\noindent se il simbolo in input è $a \in \Sigma$, l'automa $M$ nello stato $q$, con $A$ in cima alla pila, può leggere $a$, muovendo dunque la testina di input a destra di una posizione, e raggiungere lo stato $p$, senza modificare la pila;
	\item $(p, -) \in \delta(q, \varepsilon, A)$:\\
		\noindent l'automa $M$ nello stato $q$, con $A$ in cima alla pila, senza leggere alcun simbolo dall'input può raggiungere lo stato $p$, senza modificare la pila;
	\item $(p, \text{push}(B)) \in \delta(q, \varepsilon, A)$:\\
		\noindent l'automa $M$ nello stato $q$, con $A$ in cima alla pila, senza leggere alcun simbolo dall'input può raggiungere lo stato $p$, aggiungendo $B$ in cima all apila (si noti che $B$ non può essere $Z_0$);
	\item $(p, \text{pop}) \in \delta(q, \varepsilon, A)$:\\
		\noindent l'automa $M$ nello stato $q$, con $A$ in cima alla pila, senza leggere alcun simbolo dall'input può raggiungere lo stato $p$, rimuovendo il simbolo $A$ dalla cima della pila (si noti che $A$ non può essere $Z_0$).
\end{itemize}
Da un automa a pila $M$ nella forma precedente, abbiamo costruito una grammatica context free
$$ G = \langle V, \Sigma, P, S \rangle $$
dove l'insieme $V$ delle variabili contiene tutte le triple della forma $[qAp]$, con $q, p \in Q$, $A \in \Gamma$, più il simbolo iniziale $S$ di $G$, dunque
$$ V = \{[qAp] \mid q, p \in Q, A \in \Gamma\} \cup \{S\} $$
le produzioni in $P$ sono:
\begin{enumerate}[label=\alph*.]
	\item $[qAp] \rightarrow [qAr][rAp]$, per ogni $q, p, r \in Q$, $A \in \Gamma$;
	\item $[qAp] \rightarrow [q'Bp']$, per ogni $q, q', p, p' \in Q$, $A, B \in \Gamma$ tali che $(q', \text{push}(B)) \in \delta(q, \varepsilon, A)$ e $(p, \text{pop}) \in \delta(p', \varepsilon, B)$;
	\item $[qAp] \rightarrow a$, per ogni $q, p \in Q$, $a \in \Sigma \cup \{\varepsilon\}$, $A \in \Gamma$ tali che $(p, -) \in \delta(q, a, A)$;
	\item $[qAq] \rightarrow \varepsilon$, per ogni $q \in Q$, $A \in \Gamma$;
	\item $S \rightarrow [q_0Z_0q]$, per ogni $q \in F$.
\end{enumerate}
\begin{questions}
	\question Studiate come trasformare un automa a pila nella forma classica, vista in lezioni precedenti, in un automa a pila equivalente nella forma normale presentata in questa lezione.
	\begin{solution}
	\end{solution}
\end{questions}
\end{document}

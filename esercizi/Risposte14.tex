\documentclass[12pt, answers]{exam}
\usepackage{amsfonts}
\usepackage{amsmath}
\usepackage{amssymb}
\usepackage[italian]{babel}
\usepackage{caption}
\usepackage[utf8]{inputenc}
\usepackage{embedall}
\usepackage{etoolbox}
\usepackage{float}
\AtBeginEnvironment{align}{\setcounter{equation}{0}}
\embedfile{\jobname.tex}
\usepackage{graphicx}
\usepackage{listings}
\usepackage{mathtools}
\usepackage{subcaption}
\usepackage{tabularray}
\usepackage{tikz}
\usetikzlibrary{positioning, automata, decorations.pathreplacing, calligraphy}
\usepackage{xcolor}
\usepackage[bookmarks]{hyperref}

\definecolor{codegray}{gray}{0.95}
\lstdefinestyle{mystyle}{
	numberstyle=\scriptsize\color{black},
	basicstyle=\footnotesize,
	numbers=left,
	numbersep=5pt,
}
\lstset{style=mystyle}
\overfullrule=10pt

\newcommand{\grammar}[4]{\langle #1, #2, #3, #4 \rangle}
\newcommand{\grammarP}[4]{\langle \{ #1 \}, \{ #2 \}, \{ #3 \}, #4 \rangle}
\newcommand{\der}{\Rightarrow}
\newcommand{\prd}{\rightarrow}

\footer{}{\thepage}{}
\begin{document}
\begin{questions}
	\question Sia $\Sigma = \{a, b\}$ e $L = \{w \in \Sigma^+ \mid \#_a(w) = \#_b(w) \}$, dove $\#_\sigma(w)$ indica il numero di occorrenze del simbolo $\sigma$ nella stringa $w$, con $\sigma \in \Sigma$.
	Siano inoltre $L_a = \{w \in \Sigma^+ \mid \#_a(w) = \#_b(w) + 1 \}$ e $L_b = \{ w \in \Sigma^+ \mid \#_b(w) = \#_a(w) + 1 \}$.
	Si fornisca una definizione induttiva per le stringhe nel linguaggio $L, L_a, L_b$. Dalla definizione ricavate una grammatica per il linguaggio $L$.
	\begin{solution}
		Definiamo per mutua induzione $L, L_a$ e $L_b$.
		Caso base: 
		\begin{align*}
			a &\in L_a \\
			b &\in L_b 
		\end{align*}
		Passo induttivo:
		\begin{itemize}
			\item se $w \in L$, allora $aw \in L_a$ e $wa \in L_a$;
			\item se $w \in L$, allora $bw \in L_b$ e $wb \in L_b$;
			\item se $w \in L_a$, allora $bw \in L$ e $wb \in L$;
			\item se $w \in L_b$, allora $aw \in L$ e $wa \in L$.
		\end{itemize}
		Da qui possiamo ricavare la grammatica
		\begin{align*}
			L   &\rightarrow a L_b \mid L_b a \mid b L_a \mid L_a b \\
			L_a &\rightarrow a L \mid L a \mid a \\
			L_b &\rightarrow b L \mid L b \mid b
		\end{align*}
	\end{solution}
	\textit{Suggerimento.} Ispiratevi all'esempio presentato a lezione per il linguaggio delle parentesi bilanciate, utilizzando affermazioni come `\textit{Una stringa di $L$ inizia con una $b$ seguita da una stringa di $L_w$, oppure inizia con ...}'. 
	Per ricavare la grammatica potete considerare una variabile per ognuno dei tre linguaggi.
	Una delle possibili soluzioni vi permetterà di ottenere una grammatica per $L$ presentata a lezione.
	\question Ripetete l'esercizio precedente, sostituendo $w \in \Sigma^*$ a $w \in \Sigma^+$ nella definizione dei linguaggio. 
	Osservate che è sufficiente un unico caso base n tutta la definizione ricorsiva.
	\begin{solution}
		Precediamo similmente a sopra per mutua induzione.
		Caso base: 
		$$ \varepsilon \in L $$
		Passo induttivo:
		\begin{itemize}
			\item se $w \in L$, allora $aw \in L_a$ e $wa \in L_a$;
			\item se $w \in L$, allora $bw \in L_b$ e $wb \in L_b$;
			\item se $w \in L_a$, allora $bw \in L$ e $wb \in L$;
			\item se $w \in L_b$, allora $aw \in L$ e $wa \in L$.
		\end{itemize}
		Da qui possiamo ricavare la grammatica
		\begin{align*}
			L   &\rightarrow a L_b \mid L_b a \mid b L_a \mid L_a b \mid \epsilon \\
			L_a &\rightarrow a L \mid L a \\
			L_b &\rightarrow b L \mid L b 
		\end{align*}
	\end{solution}
\end{questions}
\end{document}

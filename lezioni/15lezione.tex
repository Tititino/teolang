\documentclass[12pt]{article}
\usepackage{amsmath}
\usepackage{amsfonts}
\usepackage{amssymb}
\usepackage{amsthm}
\usepackage[italian]{babel}
\usepackage{bytefield}
\usepackage{cancel}
\usepackage{caption}
\usepackage{embedall}\embedfile{\jobname.tex}
\usepackage{float}
\usepackage[bookmarks]{hyperref}
\usepackage{listings}
\usepackage[scr=rsfs]{mathalpha}
\usepackage{siunitx}
\usepackage{subcaption}
\usepackage{tabularray}
\usepackage[most]{tcolorbox}
\usepackage{tikz}\usetikzlibrary{automata, chains, scopes, decorations.text, patterns, decorations.pathmorphing, positioning, decorations.pathreplacing, calligraphy}
\usepackage{todonotes}
\usepackage{xcolor}

\newtheorem{teorema}{Teorema}
\newtheorem{corollario}{Corollario}
\newtheorem{proposizione}{Proposizione}
\newtheorem{proprietà}{Proprietà}
\newtheorem{lemma}{Lemma}
\newtheorem{fatto}{Fatto}
\newtheorem{definizione}{Definizione}
\newtheorem{nota}{Nota}

\renewcommand\qedsymbol{$\blacksquare$}

\usepackage{teolang}

\definecolor{codegray}{gray}{0.95}

\lstdefinestyle{mystyle}{
  numberstyle=\tiny,
  basicstyle=\footnotesize,
  breaklines=true,
  numbers=left,
  numbersep=5pt,
}
\lstset{style=mystyle}

\overfullrule=0.2cm
\begin{document}
\tableofcontents
\newpage

Dato un linguaggio ci possiamo chiedere se questo sia CF.
Ad esempio
$$ L = \{ a^l b^k c^j \mid k = j \} \overset{?}{\in} \text{CF} $$
Un linguaggio è CF se possiamo costruire una grammatica di tipo 2 o un automa a pila.
Ad esempio per il linguaggio di sopra possiamo consumare tutte le $a$ e controllare che il numero di $b$ e di $c$ sia uguale con una pila.

Prendiamo invece
$$ L = \{ a^i b^k c^j \mid i = j = j \} \overset{?}{\in} \text{CF} $$
L'automa per il linguaggio di prima non può essere adattato a questo linguaggio.

\begin{tcolorbox}
	Prendiamo la grammatica
	\begin{align*}
		S &\rightarrow [S] \mid S S \mid T \\
		T &\rightarrow (T) \mid T T \mid \epsilon
	\end{align*}
	e una derivazione
	$$ S \overset{*}{\Rightarrow} [()[]] $$
	Ora un albero di derivazione che possiamo fare per questa stringa è
	% fig 15.1
	sugli alberi di derivazione si possono fare operazione di sostituzione di sottoalberi, ad esempio nell'albero di sopra sostituenzo il sottoalbero 2 con il sottoalbero 1 otteniamo l'albero di derivazione per $[()[()]]$.
	In generale quello di sopra è un particolare albero che è rappresentato dalla derivazione
	$$ A \overset{*}{\Rightarrow} v A x, \hspace{1cm} v, x \in \Sigma^* $$
	questi sono interessanti perché possiamo ???.
	Ad esempio nell'albero prima abbiamo
	$$ S \overset{*}{\Rightarrow} [() S ] $$
	possiamo vedere che possiamo sia accorciare la derivazione
	% fig 15.2
	In questo modo possiamo generare infinite stringhe.
\end{tcolorbox}

Dal fatto che le grammatiche possono essere convertite in FNC (a patto di sacrificare la parola vuota), lavoreremo con grammatiche in FNC per semplificare la dimostrazione del pumping lemma.

Definiamo la \textit{profondità} (o altezza) di un albero come il più lungo cammino dalla radice ad una foglia.
\begin{lemma}
	Sia
	$$ G = \langle V, \Sigma, P, S \rangle $$
	una grammatica in FNC e sia $T : A \overset{*}{\Rightarrow} w \in \Sigma^*$ un albero di derivazione di altezza $h$.
	Allora la lunghezza di $w$ è minore o uguale a $2^{h - 1}$.
	$$ |w| \leq 2^{h - 1} $$
\end{lemma}
\begin{proof}
	Procediamo per induzione su $h$:
	\begin{itemize}
		\item per $h = 1$: per forza l'albero deve rappresentare una produzione della forma $A \rightarrow a \in \Sigma$, quindi $ w = a $ e 
			$$ |w| = 1 = 2^0 = 2^{1 - 1} $$
		\item la produzione applicata alla radice deve per forza essere della forma $A \rightarrow B C$, quindi l'albero si divide in due sottoalberi, un albero $T' : B \overset{*}{\Rightarrow} w'$ e un albero $T'' : C \overset{*}{\Rightarrow} w''$.
			% fig 15.3
			Questi due hanno altezza minore o uguale ad $h - 1$.
			Ora applicando l'ipotesi induttiva 
			\begin{align*}
				|w'| &\leq 2^{h - 2} \\
				|w''| &\leq 2^{h - 2}
			\end{align*}
			e 
			$$ |w| = |w'| + |w''| \leq 2^{h - 2} + 2^{h - 2} = 2^{h - 1} $$
	\end{itemize}
\end{proof}

\begin{lemma}[Pumping lemma per CFL]
	Sia $L$ un linguagio CF allora $\exists N > 0$ tale che $\forall z \in L$ con $|z| \geq N$, questa può essere scomposta
	$$ z = uvwxy $$
	tali che
	\begin{enumerate}
		\item $vwx| \leq N $
		\item $vx \neq \epsilon $
		\item $\forall i \geq 0 \mid uv^i w x^i y \in L$
	\end{enumerate}
\end{lemma}
\begin{proof}
	Sia $G = \langle V, \Sigma, P, S \rangle$ una grammatica in FNC per $L \setminus \{ \epsilon \}$.
	Sia $k = | V | $ e definiamo $N = 2^k$.

	Sia $z \in L$ con $|z| \geq N$, allora ha un albero di derivazione $T : S \overset{*}{\Rightarrow} z$
	% fig 15.4
	visto che la lunghezza di $z$ è maggiore di $2^k$, allora dal lemma precedente abbiamo che l'altezza di $T$ è almeno $k + 1$, quindi esiste un cammino dalle foglie alle radici da $k + 1$ archi, quindi $k + 2$ nodi.
	Visto che l'ultimo nodo è un terminale, durante questo cammino incontreremo $k + 1$ non terminali, e quindi almeno un non terminale si ripeterà in questo percorso.
	Sia $A$ questo non terminale.
	% fig 15.5
	Per questo non terminale $A$ vale
	\begin{align*}
		A &\overset{*}{\Rightarrow} w \\
		A &\overset{*}{\Rightarrow} v A x \\
		S &\overset{*}{\Rightarrow} u A y \\
	\end{align*}
	è facile vedere che 
	$$ S \overset{*}{\Rightarrow} u A y \overset{*}{\Rightarrow} u v A x y \overset{*}{\Rightarrow} \dots \overset{*}{\Rightarrow} u v^i A x^i y \overset{*}{\Rightarrow} u v^i w x^i y $$
	quindi abbiamo dimostrato il punto 3.

	La produzione centrale di $A$ deve essere per forza della forma $A \rightarrow B C$, supponiamo che $C$ sia il non terminale sul percorso più lungo che genera $w x$, allora visto che siamo in FNC e non possiamo generare la parola vuota, allora per forza $B$ genera qualcosa diverso da $\epsilon$, quindi abbiamo dimostrato il punto 2.

	L'altezza della parte dell'albero che genera $vwx$ è al massimo $k + 1$, cioè il numero massimo di nodi che possiamo vedere prima di trovare una ripetizione.
	Quindi utilizzando ancora il lemma di sopra, $|vwx| \leq N$.
\end{proof}
\begin{tcolorbox}[breakable]
	Riprendiamo il linguaggio di prima
	$$ L = \{ a^n b^n c^n \mid c \geq 0 \} $$
	mostriamo che non soddifsa il pumping lemma.

	Supponiamo per assurdo che $L$ sia CF e mostriamo che non può esistere una costante $N$ per cui valga il pumping lemma.
	Sia $N$ la costante di $L$, prendiamo
	$$ z = a^N b^N c^N = u v w x y $$
	Visto che per la prima condizione $|vwx| \leq N$, $vwx$ potrà contenere solo due dei tre simboli, più precisamente $vwx \in a^* b^*$ o $vwx \in b^* c^*$.
	Supponiamo che $vwx \in a^* b^*$, prendiamo $i = 0$ e la stringa $z' = uwy$, questa per la condizione 3 dovrebbe essere in $L$.
	Calcoliamo ora le occorrenze dei simboli in $z'$:
	\begin{align*}
		\#_c(z') &= N \\
		\#_a(z') + \#_b(z') &= 2N - (\#_a(vx) + \#_b(vx)) \\
		                     &\leq 2N \tag*{\tiny Per la condizione 2 $(\#_a(vx) + \#_b(vx)) \geq 1$}
	\end{align*}
	e quindi $z' = a^k b^j c^N \not \in L$ con $k, j < N$, quindi abbiamo un assurdo.
\end{tcolorbox}

\begin{tcolorbox}[breakable]
	Prendiamo il linguaggio
	$$ L = \{ w w \mid w \in \{a, b\}^* \} $$
	questo non è CF.

	Mostriamolo ancora attraverso il pumping lemma.
	Sia $N$ la costante del pumping lemma, scegliamo la stringa
	$$ z = a^N b^N a^N b^N \in L = u v w x y $$
	Utilizznado ancora la condizione 1 abbiamo due casi
	\begin{itemize}
		\item $vwx \in a^* b^* $, prendiamo la stringa $z' = u w x$, questa dovrebbe essere in $L$ per la condizione 3.
			Questa è $z' = a^N b^{N'} a^{N''} b^N$, con $N' \leq N, N'' \leq N$, ora possono essere diminuite solo le $a$, solo le $b$ o entrambe, ma in ogni caso $z' \not \in L$.
		\item $vwx \in b^* a^*$, questo a sua volta dsi divide in due casi, in base al fatto che $vwx$ sia nella prima o nella seconda parte della stringa.
			Prendendo ancora $i = 0$, $z' = a^{N'} b^{N''} a^N b^N$, con $N' \leq N$ e $N'' \leq N$, con ancora almeno uno tra $N'$ e $N''$ minore o uguale a $N$.
	\end{itemize}
\end{tcolorbox}

\begin{tcolorbox}[breakable]
	Prendiamo il linguaggio
	$$ L = \{ a^h b^j a^k \mid j = \max(h, k) \} $$
	supponiamo sia CF e chiamiamo $N$ la costante del pumping lemma.
	Prendiamo la stringa
	$$ z = a^N b^N a^N  \in L = u v w x y $$
	Anche qui ci sono due casi
	\begin{itemize}
		\item $vwx \in a^* b^*$, sappiamo che $vx \neq \epsilon$, distinguiamo tre casi
			\begin{itemize}
				\item $vx \in a^+$, prendendo $i = 2$, otteniamo $z' = a^{N'} b^N a^N$, con $N' > N$, che non fa parte di $L$
				\item $vx \in b^+$, prendendo $i = 0$, otteniamo $z' = a^N b^{N'} a^N$, con $N' < N$, che non fa parte di $L$
				\item $vwx \in a^+b^+$, prendendo $i = 0$, otteniamo $z' = a^{N'} b^{N''} a^N$, con $N'' < N$, che non fa parte di $L$
			\end{itemize}
		\item $vwx \in b^* a^*$, questo caso è simmetrco al precedente
	\end{itemize}
\end{tcolorbox}

\begin{tcolorbox}
	Prendiamo il linguaggio
	$$ L = \{ a^n b^n c^l \mid k \neq n \} $$
	è un linguaggio che rispetta il pumping lemma, ma non è CF.
	% vedremo la prossima lezione.
\end{tcolorbox}

\end{document}

\documentclass[12pt]{article}
\usepackage{amsmath}
\usepackage{amsfonts}
\usepackage{amssymb}
\usepackage{amsthm}
\usepackage[italian]{babel}
\usepackage{bytefield}
\usepackage{cancel}
\usepackage{caption}
\usepackage{embedall}\embedfile{\jobname.tex}
\usepackage{float}
\usepackage[bookmarks]{hyperref}
\usepackage{listings}
\usepackage[scr=rsfs]{mathalpha}
\usepackage{siunitx}
\usepackage{subcaption}
\usepackage{tabularray}
\usepackage[most]{tcolorbox}
\usepackage{tikz}\usetikzlibrary{automata, chains, scopes, decorations.text, patterns, decorations.pathmorphing, positioning, decorations.pathreplacing, calligraphy}
\usepackage{todonotes}
\usepackage{xcolor}

\newtheorem{teorema}{Teorema}
\newtheorem{corollario}{Corollario}
\newtheorem{proposizione}{Proposizione}
\newtheorem{proprietà}{Proprietà}
\newtheorem{lemma}{Lemma}
\newtheorem{fatto}{Fatto}
\newtheorem{definizione}{Definizione}
\newtheorem{nota}{Nota}

\renewcommand\qedsymbol{$\blacksquare$}

\usepackage{teolang}

\definecolor{codegray}{gray}{0.95}

\lstdefinestyle{mystyle}{
  numberstyle=\tiny,
  basicstyle=\footnotesize,
  breaklines=true,
  numbers=left,
  numbersep=5pt,
}
\lstset{style=mystyle}

\overfullrule=0.2cm
\begin{document}
\tableofcontents
\newpage

% ???
Bisogna trovare un $i$ tale per cui 
$$ m + (i - 1)(l + r) $$
per cui la somma non sia un numero primo.
Scegliendo $i = m + 1$, allora 
$$ m + m(l + r) = m(l + r + 1) $$
non è primo.

\beign{nota}
Se l'alfabeto è di una lettera sola, non c'è differenza tra regolari e context free.
\end{nota}

\beign{tcolorbox}[breakable]
Sia
$$ \mathcal{L} = \{ a^n b^n c^k \mid k \neq n \} $$
Intuitivamente non è CF, infatti posso usare una pila per confrontare le $a$ e le $b$, ma una volta fatto questo ho perso l'informazione su $n$.

Mostriamolo con il pumping lemma. 
Scegliamo fissiamo la costante $N$ e una scegliamo una stringa
$$ z = a^m b^m c^j = u v w x y  \in \mathcal{L} \mid |<| \geq N $$
quindi $2m + j \geq N$.

Analizziamo la composizione di $vwx$:
\begin{itemize}
	\item $vwx \in a^+$, questo è facilmente risolvibile mostrando che se si aumentano o diminuiscono le $a$ il loro numero diventa diverso da quello delle $b$
	\item $vwx \in b^+$, è analogo al caso di sopra
	\item $vwx \in c^+$, se $j = 1$ allora facilmente possiamo fissare una $i$ tale che rende il numero delle $c$ uguale a $m$
		Alternativamente possiamo mostrare un caso non valido anche con la stringa 
		$$ z = a^{N + N!} b^{N + N!} c^N $$
		se assumiamo che $|vx| = k$, in
		$$ u v^i w x^i y = a^{N + N!} b^{N + N!} c^{N + k(i - 1)} $$
		e $0 < k \leq N$, vogliamo che $k(i - 1) = N!$, quindi scegliamo $i - 1 = \frac{N!}{k}$, cioè
		$$ i = 1 + \frac{N!}{k} $$
	\item $vwx \in a^+b^+$, questo a sua volta si divide in vari sottocasi
		\begin{itemize}
			\item $v \in a^+ b^+$, cioè il confine tra $a$ e $b$ cade in $v$, è facile mostrare che $v^i$ sarebbe una stringa composta da $a$ seguite da $b$ seguite ancora da $a$ e così via
			\item $w \in a^+ b^+$, allora $v \n a^*$ e $x \in b^*$ abbiamo ancora altri casi
				\begin{itemize}
					\item se $v$ e $x$ sono di dimensione diversa è facile
					\item se $v$ e $x$ sono di dimensione uguale, cioè
						$$ v = a^h, b = b^h $$
				\end{itemize}
\end{itemize}
In questo esempio il pumping lemma non si può applicare. % ???

\end{tcolorbox}

Sia $T$ un albero con alcune foglie marcate.
E definiamo dei nodi interni speciali detti \textit{branch point} definiti come nodi che hanno almeno due figli marcati o due figli con discendenti marcati.
Definiamo i \textit{nodi speciali} come tutti i branch point e i nodi che hanno almeno un figlio marcato.
% 16.1
supponiamo di marcare un altro nodo
% 16.2
\begin{lemma}
	Sia $G = \langle V, \Sigma, P, S\rangle$ una grammatica in FNC e prendiamo un albero di derivazione 
	$$A \overset{*}{\Rightarrow} w, \hspace{1cm} A \in V, \w \in \Sigma^* $$
	e $w$ contiene alcune posizioni marcate.

	Se il numero massimo di nodi speciali su un cammino dalla radice alle foglie è minore o uguale a $k$, allora il numero di posizioni marcate in $w$ è $\leq 2^k - 1$.
\end{lemma}
Questo è una generalizzazione del lemma della lezione precedente, nell'altro supponevamo marcata l'intera stringa.

\begin{proof}
	Procediamo per induzione su $k$
	\begin{itemize}
		\item $k = 1$, quindi in ogni cammino dalla radice ad una foglia esiste al massimo un nodo speciale.
			% 16.3
			Supponiamo che questo nodo speciale sia un branch point
			% 16.4
			ma questo non può valere visto che in FNC l'unica produzione che può generare un terminale è della forma $C \Rightarrow a$, quindi anche questi due dovrebbero essere nodi speciali.
			Quindi esiste una singola foglia marcata, infatti se ne esistesse più di una, allora il loro nodo in comune sarebbe un branch point.
		\item supponiamo che sia vero per valori $< k$, e mostriamo che è vero per $k$.
			% 16.5
			Fermiamoci al primo nodo speciale dalla radice, e che questo sia un branch point, per ipotesi induttiva nell'abero di sinistra e di destra ci saranno al massimo $2^{k - 2}$ posizioni marcate.
			Inoltre $z_0$ e $z_3$ non possono avere posizioni marcate, altrimenti il branch point sarebbe più in alto.
			Quindi $\leq 2^{k - 2} + 2^{k - 2} = 2^{k - 1}$.
	\end{itemize}
\end{proof}
\begin{lemma}[Lemma di Oyder]
	Sia $L \in \text{CF}$, allora esiste una costante $N$ tale che per ogni $z \in L$ sono marcate $N$ posizioni e possiamo scomporre $z$ tale che
	$$ z = u v w x y $$
	tale che
	\begin{itemize}
		\item $vx$ contiene almeno una posizione marcata
		\item $vwx$ contiene al più $N$ posizioni marcate
		\item per ogni $i > 0$, $u v^i w x^i y \in L$
	\end{itemize}
\end{lemma}
Quindi il pumping lemma è un caso speciale di questo in cui ogni posizione è marcata.
\begin{proof}
	Sia $G = \langle V, \Sigma, S, P \rangle$ una grammatica in FNC, definiamo $k = |V|$ e fissiamo $N = 2^k$.
	Prendiamo una stringa $z \in L$ con almeno $N$ nodi marcati.
	Prendiamo il cammino dalle foglie alla radice che contiene il maggior numero di nodi speciali.
	In base al lemma di prima, il massimo numero di nodi speciali sul cammino è $\geq k + 1$.
	Percorrendo il cammino e leggendo le variabili che compaiono sui nodi speciali, sia questa $A$, troveremo almeno una ripetizione.
	I sottoalberi di questa coppia di definisce le nostre parti $u, v, w, x$ e $y$.

	Visto che la prima occorrenza di $A$ sarà un branch point, questa sarà una produzione del tipo $A \rightarrow BC$, uno dei due rami potrerà alla seconda $A$ e sicuramente il secondo porterà ad una foglia marcata, quindi in $v$ e $x$ c'è almeno un branch point.
\end{proof}

Con questo nuovo lemma proviamo a dimostrare l'esempio di prima
\begin{tcolorbox}[breakable]
	Sia
	$$ L = \{ a^n b^n c^k \mid k \neq n \} $$
	con $N$ costnte per $L$.
	La stringa con cui si può arrivare ad un assurdo è
	$$ z = a^N b^N c^{N + N!} = uvwxy $$
	con tutte le $a$ marchiate.

	Visto che $vx$ deve contenere almeno una posizione marcata, allora questo deve avere almeno una $a$ al suo interno.
	Questo ci restringe ai tre casi
	\begin{itemize}
		\item $vwx \in a^+$, il numero di $a$ cresce, ma non il numero di $b$
		\item $vwx \in a^+ b^+$, si divide in alcuni sottocasi in base a dove il confine tra le $a$ e $b$ cade
			\begin{itemize}
				\item $v \in a^+ b^+$, ripetere $v$ fa sì che la stringa perda la struttura e porta ad avere $a$ dopo le $b$
				\item analogo a sopra se il confine si trova su $x$
				\item $v \in a^+$ e $x \in b^+$, supponiamo più precisamente che
					$$ v = a^l, x = b^r $$
					\begin{itemize}
						\item se $l \neq r$ è ovvio che la stringa non sia valida, anche solo per $i = 0$ cancellerei un numero diverso di $a$ e di $b$
						\item se $l = r$, la stringa che otterremmo ripetendo $i$ volte sarebbe
							$$ a^{N + l(i - 1) b^{N + l(r - 1)} c^{N + N!} $$
							se si scegle $i = 1 + \frac{N!}{l}$ allora otteniamo che le $a$ e le $b$ sono uguali al numero di $c$ e $N!$ è sicuramente divisibile da $l$.
					\end{itemize}
			\end{itemize}
		\item $vwx \in a^+ b^N c^+ $, 
			\begin{itemize}
				\item se $v$ e $x$ contengono tipi di lettere diverse, come prima ripetendo si perde la struttura
				\item per il resto si tratta in maniera analogo a il secondo caso
			\end{itemize}
	\end{itemize}
\end{tcolorbox}

% esercizi
\begin{tcolorbox}[breakable]
	Sia
	$$ \mathcal{L} = \{ a^p b^q c^r \mid p = q \vee q = r \} $$
	e 
	$$ \mathcal{L} = \{ a^p b^q c^r \mid p = q \otimes q = r \} $$
\end{tcolorbox}
\end{document}

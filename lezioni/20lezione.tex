\documentclass[12pt]{article}
\usepackage{amsmath}
\usepackage{amsfonts}
\usepackage{amssymb}
\usepackage{amsthm}
\usepackage[italian]{babel}
\usepackage{bytefield}
\usepackage{cancel}
\usepackage{caption}
\usepackage{embedall}\embedfile{\jobname.tex}
\usepackage{float}
\usepackage[bookmarks]{hyperref}
\usepackage{listings}
\usepackage[scr=rsfs]{mathalpha}
\usepackage{siunitx}
\usepackage{subcaption}
\usepackage{tabularray}
\usepackage[most]{tcolorbox}
\usepackage{tikz}\usetikzlibrary{automata, chains, scopes, decorations.text, patterns, decorations.pathmorphing, positioning, decorations.pathreplacing, calligraphy}
\usepackage{todonotes}
\usepackage{xcolor}

\newtheorem{teorema}{Teorema}
\newtheorem{corollario}{Corollario}
\newtheorem{proposizione}{Proposizione}
\newtheorem{proprietà}{Proprietà}
\newtheorem{lemma}{Lemma}
\newtheorem{fatto}{Fatto}
\newtheorem{definizione}{Definizione}
\newtheorem{nota}{Nota}

\renewcommand\qedsymbol{$\blacksquare$}

\usepackage{teolang}

\definecolor{codegray}{gray}{0.95}

\lstdefinestyle{mystyle}{
  numberstyle=\tiny,
  basicstyle=\footnotesize,
  breaklines=true,
  numbers=left,
  numbersep=5pt,
}
\lstset{style=mystyle}

\overfullrule=0.2cm
\begin{document}
\tableofcontents
\newpage
Data la grammatica
$$ G = \langle V = \{S, T, A, B\}, \Sigma = \{a, b, c, d\}, S, P \rangle $$
con $P$ definito come
\begin{align*}
		S &\rightarrow cAT \mid c T \\
		T &\rightarrow  d S \mid d \\
		A &\rightarrow B a A \mid B a \\
		B &\rightarrow b B \mid \varepsilon
\end{align*}
% ????
% !!! MISSING !!!
\begin{align*}
	S &\rightarrow c A T \mid c T \\
	T &\rightarrow A S \mid d 
\end{align*}

% fig 20.1

Proviamo a fare la conversione da automa a linguaggio regolare:
$$ \left { \begin{array}{ccc} S & = & c Q + c T \\ Q & = & A T \\ T & = & d S + d \end{array} $$
sostituendo la terza nella prima e la seconda abbiamo
$$ \left { \begin{array}{ccc} S & = & c Q + c d S + c d \\ Q & = & A d S + A d \\ T & = & d S + d \end{array} $$
e poi sostituiamo la seconda nella prima
$$ \left { \begin{array}{ccc} S & = & c A d S + c A d + c d S + c d \\ Q & = & A d S + A d \\ T & = & d S + d \end{array} $$
e semplificando
$$ \left { \begin{array}{ccc} S & = & (c A d + c d)S + c A d + c d \\ Q & = & A d S + A d \\ T & = & d S + d \end{array} $$
che corrisponde a $S = (cAd + cd)^* (cAd + c) = (cAd + cd)^+$.

Alla regola $A \rightarrow B a A \mid B a$ corrisponde il linguaggio regolare $A \overset{*}{\Rightarrow} (B a)^+$, e a $B \rightarrow b B \mid \varepsilon$ corrisponde ovviamente $B \overset{*}{\Rightarrow} b^*$.
Quindi sostituendo nel regolare di $A$ abbiamo $A \overset{*}{\Rightarrow} (b^* a)^+$, e sostituendo in $S \overset{*}{\Rightarrow} (c (b^* a)^+ d + cd)^+$.
Quindi $L(G) = (c (b^* a)^* d)^+$.

\begin{teorema}[Teorema di Chomsky-Schutzenberger]	% sciutzenberger, pronunciabile alla francese (bergie) o alla tedesca (bergher)
	Sia $\Omega_k$ un alfabeto di $k$ tipi di parentesi
	$$ \Omega_k = \{ (_1, (_2, \dots, (_k, )_1, )_2, \dots, )_k \} $$
	$D_k \subseteq \Omega_k^*$ è detto un linguaggio di Dyck, se è il linguaggio delle parentesi bilanciate correttamente.
	Sia $L \subseteq \Sigma^*$ un linguaggio CF, allora 
	$$ \exists k > 0 \mid h : \Omega_k \rightarrow \Sigma \; \text{morfismo} \; \text{e} R \subseteq \Omega_k \; \text{regolare} $$
	tale che $L = h(D_k \cap R)$.
\end{teorema}
Non dimostrato.

\begin{tcolorbox}[breakable]
	Prendiamo il caso banale $L = D_1$, allora $k = 1$ e $h = \text{id}$ e $R = \Omega_1^*$.
\end{tcolorbox}
\begin{tcolorbox}[breakable]
	Prendiamo il caso 
	$$L = \{ a^n b^n \mid n \geq 0 \} $$
	allora $k = 1$ e 
	$$ h = \begin{cases} ( = a \\ ) = b \end{cases} $$ 
	e infine $R = (^* )^*$.
\end{tcolorbox}

\begin{tcolorbox}[breakable]
	Prendiamo il caso 
	$$L = \{ w w^R \mid \mid w \in \{a, b\}^* \} $$
	allora prendiamo $k = 2$ e 
	$$ h = \begin{cases} ( = a & \\ a = ) \\ [ = b \\ ] = b \end{cases} $$ 
	e infine il filtro $R = (\text{\texttt{$($}} + \text{\texttt{$[$}})^* (\text{\texttt{$]$}} + \text{\texttt{$)$}})^*$.
\end{tcolorbox}

\begin{tcolorbox}[breakable]
	Prendiamo il caso 
	$$L = \{ x \mid \mid x \in \{a, b\}^* \wedge x = x^R \} $$
	allora prendiamo $k = 4$ e similmente a prima definiamo $h$
	$$ h = 
	\begin{cases} 
		\langle = a \\
		\rangle = \varepsilon \\
		\lfloor = b \\
		\rfloor = \varepsilon \\
		( = a \\ 
		a = ) \\ 
		[ = b \\ 
		] = b 
	\end{cases} $$ 
	e infine il filtro $R = (\text{\texttt{$($}} + \text{\texttt{$[$}})^* ( \varepsilon + \text{\texttt{$\langle \rangle$}} + \text{\texttt{$\lfloor \rloor$}}) (\text{\texttt{$]$}} + \text{\texttt{$)$}})^*$.
\end{tcolorbox}
Quindi il riconoscimento di un linguaggio CF si riconduce a riconoscere un linguaggio di parentesi.
Quindi si potrebbe pensare di costruire una macchina
% fig 20.2
Il componente $h^{-1}$ può essere realizzato come un componente a stati finiti.
Quindi l'unica cosa non a stati finiti è il componente $D_k$.

\section{Lingauaggi unari}
Un linguaggio unario è un linguaggio tale che $| \Sigma | = 1$.
Mostreremo che nel caso unario i linguaggi CF e i regolari sono la stessa classe.

Supponiao di avere un automa deterministico per un alfabeto unario
% fig 20.3
Visto che c'è un solo simbolo, questi sono formati da una serie di stati e opzionali loop indietro.
Il linguaggio riconosciuto da sopra è
$$ L = \varepsilon, a^6, a^11, a^16, \dots = a^0 + a^6(a^5)^* $$
Modificandolo in questo modo
% fig 20.4
$$ L = a^0 + a^5(a^6)^* + a^6(a^5)^*$$
Questi rappresentano più o meno successioni numeriche.

Prendiamo ora l'automa
% fig 20.5
questo riconosce
$$ L = a^8 + a^{11}(a^5)^* + a^{12}(a^5)^* $$
% ???? mi sono perso

\begin{lemma}
	Supponiamo di avere una serie di naturali
	$$ l_1, l_2, l_3, \dots $$
	e un $q > 0$.
	Supponiamo di avere linguaggi del tipo
	$$ K_i = a^{l_i} (a^q)^* $$
	allora l'unione
	$$ \bigcup_{i \geq 0} K_i \in \text{Reg} $$
\end{lemma}
Noi abbiamo visto che l'unione di linguaggi regolari è ancora regolare, ma qui il risultato importante è che l'unione può essere di un numero potenzialmente infinito di regolari.il risultato importante è che l'unione può essere di un numero potenzialmente infinito di regolari.
Ad esempio questo non vale per l'unione infinita di 
$$ \bigcup_{i \geq 0} a^i b^i $$

Tutti questi possono essere ricondotti all'automa di sopra purché abbiano tutti lo stesso periodo $q$.

\begin{lemma}
	Sia $N$ un intero e prendiamo dei linguaggi
	$$ K_i = a^{l_i} (a^{q_i})^* \subseteq a^* $$
	quindi senza un periodo fissato.
	Se però che $\forall i \mid 1 \leq q_i \leq N$, allora l'unione
	$$ \bigcup_i K_i \in \text{Reg} $$
\end{lemma}
Infatti possiamo generare per il lemma di prima i linguaggi regolari per ogni $q$, e visto che $N$ è finito facciamo l'unione finita di tutti i linguaggi per ogni $q$ nel range.

\begin{teorema}
	$L \subseteq a^* \in \text{CFL}$ implica $L$ regolare.
\end{toerema}
\begin{proof}
	Utilizzeremo il pumping lemma per i linguaggi CF.
	Sia $N$ la costante del pumping lemma per $L$.
	Prendiamo $a^m \in L$ con $m \geq N$, allora
	$$ a^m = u v w x y $$
	sappiamo che
	\begin{itemize}
		\item $vx \neq \varepsilon$, e chiamiamo $q_m = |vx|$
		\item $|vwx| \leq N$
		\item $\forall i \geq 0 \mid u v^i w x^i y \in L$, la lunghezza di questa stringa è (escludendo il caso $i = 0$) 
			$$ m + (i - 1)q^m $$
			quindi $a^{m + (i - 1)q_m} \in L $
			cioè $a^m (a^{q_m})^* \subseteq L$
	\end{itemize}
	quindi $q \leq q_m \leq N$.
	Prendiamo
	\begin{align*}
		L' &= \{ a^m \in L \mid m < N \} \\
		L'' &= \{ a^m \in L \mid m \leq N \} 
	\end{align*}
	e vale che
	$$ L = L' \cup L'' \subseteq L' \cup \bigucp_{m \geq N} a^m (a^{q_m})^* \subseteq L $$
	quindi
	$$ L = L' \cup \bigcup_{m \geq N} a^m(a^{q_m})^* $$
	e $L'$ è regolare perché finito e per il lemma di prima anche il secondo è ancora regolare, visto che $1 \leq q_m \leq N$ per il pumping lemma.
	E visto che l'unione di due regolari è regolare, allora $L$ è regolare.
\end{proof}
Dati 
% fig 20.7
si può vedere che generalmente il periodo è il $\text{MCD}$ tra i due periodi dei linguaggi.

Nei linguaggi unari si può mostrare lo strano risultato che per passare da un nodeterministico a un deterministico il costo è $e^{\sqrt{n \ln{n}}}$

\begin{teorema}[Teorema di Parikh]
	Dato un alfabeto $\Sigma$ di qualsiasi dimensione e una stringa $w$.
	Sia ad esempio $\Sigma = \{a, b\}$, associamo a questa stringa il vettore 
	$$ \phi(w) = ( \#_a(w), \#_b(w) ) $$
	chiamato immagine di Parikh.
	E definiamo l'immagine di Parikh del linguaggio come
	$$ \phi(L) = \{ \phi(w) \mid w \in L \} $$
	Se $L \in \text{CFL}$, allora $\exists L' \in \text{Reg}$ tale che $\phi(L) = \phi(L')$.
\end{teorema}
Quindi se non ci interessa l'ordine dei simboli, il linguaggi regolari e i linguaggi CF sono uguali.
Un corollario è che vale l'uguaglianza del CF e dei regolari nel caso di alfabeti unari.

\begin{tcolorbox}[breakable]
	Sia 
	$$ L = \{ a^n b^n \mid n \geq 0 \} $$
	e 
	$$ \phi(L) = \{ (n, n) \mid n \geq 0 \} $$
	allora possiamo prendere il linguaggio regolare $R = (ab)^*$ e vale che
	$$ \phi(L) = \phi(R) $$
\end{tcolorbox}

% prossima volta automi a pila two way
\end{document}



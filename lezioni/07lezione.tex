\documentclass[12pt]{article}
\usepackage{amsmath}
\usepackage{amsfonts}
\usepackage{amssymb}
\usepackage{amsthm}
\usepackage[italian]{babel}
\usepackage{bytefield}
\usepackage{cancel}
\usepackage{embedall}\embedfile{\jobname.tex}
\usepackage[bookmarks]{hyperref}
\usepackage{listings}
\usepackage[scr=rsfs]{mathalpha}
\usepackage{siunitx}
\usepackage{tabularray}
\usepackage{tcolorbox}
\usepackage{tikz}\usetikzlibrary{automata}
\usepackage{todonotes}
\usepackage{xcolor}

\newtheorem{teorema}{Teorema}
\newtheorem{corollario}{Corollario}
\newtheorem{proposizione}{Proposizione}
\newtheorem{proprietà}{Proprietà}
\newtheorem{fatto}{Fatto}
\newtheorem{definizione}{Definizione}
\newtheorem{nota}{Nota}

\renewcommand\qedsymbol{$\blacksquare$}

\usepackage{teolang}

\definecolor{codegray}{gray}{0.95}

\lstdefinestyle{mystyle}{
  numberstyle=\tiny,
  basicstyle=\footnotesize,
  breaklines=true,
  numbers=left,
  numbersep=5pt,
}
\lstset{style=mystyle}

\overfullrule=0.2cm
\begin{document}
\tableofcontents
\newpage
\begin{center}
	\begin{tblr}{c c c}
		op & DFA & NFA \\
		\hline
		unione & $n^\prime \cdot n^{\prime\prime}$ & $1 + n^\prime + n^{\prime\prime} $ \\
		concatenazione & $n^\prime \cdot 2^{n^{\prime\prime}} $ & $n^\prime + n^{\prime\prime}$ \\
		star           & $1 + 2^{n^\prime} $ & $n^\prime + 1$ \\
		intersezione   & $n^\prime \cdot n^{\prime\prime}$ & $ n^\prime \cdot n^{\prime\prime} $ \\
		complemento    & $n^\prime$ & $2^{n^\prime}$
	\end{tblr}
\end{center}
L'intersezione di automi funziona anche con NFA con la stessa costruzione dei DFA.

Date tre espressioni regolari
$$ (a*b*)* \equiv (a + b)* \equiv b*(ab*)* $$
Sono tre linguaggi equivalenti.

Ci si può chiedere qual è il numero massimo di star innestate necessarie per generare un linguaggio.
Introduciamo l'\textit{altezza di start}, o star height $h$, come il numero massimo di star innestate.
Definita nel seguente modo
$$ h(E) = 
\begin{cases}
	0 & \text{se } E = \varnothing \vee E = \epsilon \vee E \in \Sigma^* \\
	\max(h(E^\prime), h(E^{\prime\prime})) & \text{se } E = E^\prime \cdot E^{\prime\prime} \vee E = E^\prime + E^{\prime\prime} \\
	1 + h(E^\prime) & \text{se } E = E^\prime*
\end{cases}
$$
Definiamo poi la minima altezza per un linguaggio $L \subseteq \Sigma^*$ come
$$ h(L) = \min \{ h(E) \mid \text{ $E$ denota $L$ } \} $$

\begin{teorema}[Dejan, Schutzenberger]
	$$\forall q > 0 \exists W_q \subseteq \{a, b\}^* \mid h(W_q) = q $$
	Equesto è definito come
	$$ W_q = \{ w \in \{a, b\}^* \mid \#_a(w) \equiv \#_b(w) \mod 2^q \} $$
\end{teorema}

La misura dei cicli è collegata al numero di cicli in un automa.
\begin{tcolorbox}
	% fig 07.1
	Per $q = 1, 2, \dots$, definiamo $W_q$ ed abbiamo che
	$$ W_2 = ((ab + ba) + (aa + bb)(ab + ba)^*(bb + aa))^* $$
\end{tcolorbox}

\begin{nota}
	Se $|\Sigma| = 1$, allora $\forall L \subseteq \Sigma^* \mid h(L) \leq 1$.
\end{nota}

\section{Espressioni regolari estese}
Supponiamo di avere il linguaggio
$$ L = \{ w \in \{a, b\}^* \mid \#_a(w) \text{ è pari } \wedge \#_b(w) \text{ è pari } \} $$

Costruiamo prima il linguaggio con solo le $a$ pari
$$ (b + ab*a)* $$
e con solo le $b$ pari
$$ (a + ba*b)* $$
intuitivamente il linguaggio $L$ sarebbe, se avessimo l'intersezione, 
$$ (b + ab*a)* \cap (a + ba*b)* $$
Intoduciamo quindi le \textbf{espressioni regolari estese}.

Definiamo queste come le espressioni con, oltre alle operazioni di concatenazione, unione e star; l'intersezione ($\cap$) e il complemento ($E^-$).
Introducendo il complemento possiamo ottenere l'intersezione attraverso l'unione.

\begin{tcolorbox}
	Dato il linguaggio su $\Sigma = \{a, b\}$ delle stringhe con 3 $a$ consecutive
	% fig 07.2
	vogliamo riconoscere il suo complemento (senza 3 $a$ consecutive).
	Questo diventa semplicemente
	$$ \overline{\overline{\varnothing} a a a \overline{\varnothing}} $$
	Quindi ora senza usare la star possiamo esprimere anche linguaggi infiniti.
\end{tcolorbox}
Si può definire il problema di trovare il l'altezza di star minima ($eh$) nelle espressioni regolari estese.
Questo è un problema aperto.
Si sa che valgono
\begin{align*}
	\exists L \mid eh(L) = 0 \\
	\exists L \mid eh(L) = 1 
\end{align*}

Definiamo l'operazione di inversione o reversal.
Data una stringa 
$$ w = a_1 a_2 \dots a_n $$
definiamo
$$ w^R = a_n \dots a_2 \dots a_1 $$
Per DFA e NFA basta invertire stati iniziali e finali e le transizioni.
Però può accadere che un DFA invertito diventi non deterministico (e.g. stati finali multiplo diventano stati iniziali multipli, o più transizioni con lo stesso simbolo entranti in uno stato diventano più transizioni uscenti con lo stesso simbolo).
\begin{tcolorbox}
	Ad esempio il linguaggio il cui terzo simbolo è una $a$ è un semplice DFA, mentre il linguaggio il cui terzultimo simbolo è una $a$ ha $2^3$ stati.

	In generale se chiediamo l'$n$-esimo simbolo da destra servono $n + 1$ stati, mentre se chiediamo l'$n$-esimo simbolo da sinistra ne servono $2^n$.
\end{tcolorbox}
Mentre per le espressioni regolari, per la costruzione del reversal, basta solo invertire l'espressione.
O più precisamente
\begin{itemize}
	\item nei casi in cui $E = \varnothing \vee E = \epsilon \vee E \in \Sigma^*$, non cambia niente
	\item nei casi in cui $E = E^\prime + E^{\prime\prime}$, abbiamo $E^{\prime R} + E^{\prime\prime R}$
	\item nei casi in cui $E = E^\prime \cdot E^{\prime\prime}$, abbiamo $E^{\prime\prime R} \cdot E^{\prime R}$
	\item nei casi in cui $E = E^{\prime}*$, abbiamo $(E^{\prime R})*$
\end{itemize}

Definiamo l'operazione di \textit{shuffle}
$$ sh(x, y) $$
che produce ogni possibile combinazione dei suoi argomenti (più complicato di così, produce ogni possibile interpolazione, $a$ non è mai dopo $b$ e $c$ non è mai dopo $d$), ad esempio
$$ sh(ab, cd) = \{ abcd, acdb, acbd, cabd, \dots  \} $$
Dati due linguaggi, abbiamo che
$$ sh(L^\prime, L^{\prime\prime}) = \bigcup_{x \in L^\prime, y \in L^{\prime\prime}} sh(x, y) $$

Supponiamo che $L^\prime \subseteq \Sigma^{\prime *}$ e $L^{\prime\prime} \subseteq \Sigma^{\prime\prime *}$ e $\Sigma^\prime \cap \Sigma^{\prime\prime} = \varnothing$.
% Esercizio, costruisco questo automa, e.g. due sottoinsiemi

Definiamo l'operazione di quoziente tra due linguaggi
$$ L_1 \setminus L_2 = \{ x \in \Sigma^* \mid \exists y \in L_2 . xy \in L_1 \} $$
Quindi le stringhe di $L_1$ in cui ho tolto un suffisso di $L_2$.
\begin{tcolorbox}
	Ad esempio
	\begin{align*}
		L_1 &= a+bc+ \\
		L_2 &= bc+ 
		L_3 &= c+
	\end{align*}
	Ho che
	$$ L_1 \setminus L_2 = a+ $$
	e che
	$$ L_1 \setminus L_3 = a+bc* $$
\end{tcolorbox}

\begin{tcolorbox}
	Ad esempio da un linguaggoi su $\Sigma = \{0, 1\}$, voglio togliere tutti gli zeri finali
	$$(L \setminus 0*) \cap ((0 + 1)*1 + \epsilon)$$
\end{tcolorbox}
\begin{fatto}
	Dato $R$ linguaggio regolare ed $L$ un linguaggio qualsiasi, allora
	$$ R \setminus L $$
	è ancora un linguaggio regolare.
\end{fatto}
\begin{proof}
	Dato
	$$ \mathcal{A} = \langle Q, \Sigma, \delta, q_0, F \rangle $$
	l'automa per $R$.

	Definisco 
	$$ \mathcal{A}^\prime = \langle Q, \Sigma, \delta, q_0, F^\prime = \{ q \mid \exists y \in L . \delta(q, y) \in F \} \rangle $$
	come l'automa per $R \setminus L$.

	Non è detto che sappiamo trovare una $y \in L$.
	Quindi questa dimostrazione non è costruttiva per tutti i linguaggi.
\end{proof}

\subsection{Morfismo}
Il linguaggio delle parentesi bilanciate (Dick) non è regolare.
Dato un linguaggio di programmazione, lo possiamo ricondurre al linguaggio di Dick, ad esempio
\begin{align*}
	\{ &\rightarrow ( \\
	\} &\rightarrow ) \\
	a  &\rightarrow \epsilon 
\end{align*}

Stiamo definendo un morfismo tra due alfabeti
$$ h : \Sigma \rightarrow \Delta^* $$
In qenerale un morfismo su una strina è definito come
$$
\begin{cases}
	h(\epsilon) = \epsilon \\
	h(xa) = h(x) h(a) 
\end{cases}
$$
e il morfismo su un linguaggio come
$$ h(L) = \{ h(w) \mid w \in L \} $$

\begin{tcolorbox}
	Dati $\Sigma = \{a, b \}$, e $\Delta = \{0, 1\}$, definiamo il morfismo $h$ come
	\begin{align*}
		h(a) &= 01 \\
		h(b) &= 1 
	\end{align*}
	E vale che se abbiamo
	$$ L = a b*$$
	questo corrisponde a
	$$ 011* $$
	E agli automi corrisponde
	% fig 07.3
\end{tcolorbox}

\subsection{Sostituzione}
In questa si lettere di un linguaggio con un altro linguaggio, quindi
$$ s : \Sigma \rightarrow 2^{\Delta^*} $$
Ad esempio dato $ \Sigma = \{a, b\}$ e $\Delta = \{0, 1\}$, posso dire che
\begin{align*}
	s(a) &= (01 + 0)* \\
	s(b) &= 1
\end{align*}
e dato
$$ L = (ab)*(a + b) $$
ottengo
$$ ((01 + 0)*1)*((01 + 0)* + 1) $$
e nel caso di automi, basta sostituire una transizione con un automa.
% fig 07.4

Se le sostituzioni sono ancora regolari, allora il linguaggio risultante è regolare.

\begin{tcolorbox}
	Dato un linguaggio $L \subseteq \{a, b\}^*$, voglio creare il linguaggio $L^\prime$ in cui ogni $a$ e $b$ è circondata da un numero arbitrario di $c$, questo è
	\begin{align*}
		s(a) &= c* a c* \\
		s(b) &= c* b c*
	\end{align*}
	Manca però la stringa vuota, quindi abbiamo
	$$ 
	L^\prime = 
	\begin{cases}
		s(L) & \text{se } \epsilon \not \in L \\
		s(L) \cup c^* &\text{altrimenti}
	\end{cases}
	$$
	Questo povtevamo farlo anche con lo shuffle, quindi
	$$ L^\prime = sh(L, c^*) $$
\end{tcolorbox}

\subsection{Cycle}
Definiamo l'operazione di \textit{cycle}, come
$$ cycle(L) = \{ yx \mid xy \in L \} $$
ad esempio, se
$$ L = a*b* $$
$$ cycle(L) = a*b*a* + b*a*b* $$

Dato un automa che riconosca $L$, possiamo costruire un automa che riconosca $cycle(L)$?.
% fig 07.5


\end{document}

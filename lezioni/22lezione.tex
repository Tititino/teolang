\documentclass[12pt]{article}
\usepackage{amsmath}
\usepackage{amsfonts}
\usepackage{amssymb}
\usepackage{amsthm}
\usepackage[italian]{babel}
\usepackage{bytefield}
\usepackage{cancel}
\usepackage{caption}
\usepackage{embedall}\embedfile{\jobname.tex}
\usepackage{float}
\usepackage[bookmarks]{hyperref}
\usepackage{listings}
\usepackage[scr=rsfs]{mathalpha}
\usepackage{siunitx}
\usepackage{subcaption}
\usepackage{tabularray}
\usepackage[most]{tcolorbox}
\usepackage{tikz}\usetikzlibrary{automata, chains, scopes, decorations.text, patterns, decorations.pathmorphing, positioning, decorations.pathreplacing, calligraphy}
\usepackage{todonotes}
\usepackage{xcolor}

\newtheorem{teorema}{Teorema}
\newtheorem{corollario}{Corollario}
\newtheorem{proposizione}{Proposizione}
\newtheorem{proprietà}{Proprietà}
\newtheorem{lemma}{Lemma}
\newtheorem{fatto}{Fatto}
\newtheorem{definizione}{Definizione}
\newtheorem{nota}{Nota}

\renewcommand\qedsymbol{$\blacksquare$}

\usepackage{teolang}

\definecolor{codegray}{gray}{0.95}

\lstdefinestyle{mystyle}{
  numberstyle=\tiny,
  basicstyle=\footnotesize,
  breaklines=true,
  numbers=left,
  numbersep=5pt,
}
\lstset{style=mystyle}

\overfullrule=0.2cm
\begin{document}
\tableofcontents
\newpage
Si possono considerare modelli diversi rispetto ai LBA, ad esempio un automa con un nastro di input read-only e con $n$-nastri ausiliari.

\begin{definizione}[Memoria di Turing ad un nastro]
	Un macchina di Turing è un automa con una testina e un nastro illimitato che, sulle celle che non contengono l'input, contiene un carattere speciale detto \textit{blank}.
	$$ M = \langle Q, \Sigma, \Gamma, \delta, q_0, F \rangle $$
	\begin{itemize}
		\item $Q$ un insieme di stati
		\item $\Sigma$ un alfabeto, e $\Sigma \subseteq \Gamma$
		\item $\Gamma$ un alfabeto di lavoro, e il blank $\# \in \Gamma \setminus \Sigma$
		\item $q_0 \in Q$
		\item $F \subseteq Q$
		\item $\delta : Q \times \Gamma \rightarrow Q \times \Gamma \times \{-1, 0, +1\} $ nel caso determinsitico,
			dove
			\begin{itemize}
				\item $-1$ è un movimento a destra
				\item $0$ nessun movimento
				\item $+1$ un movimento a destra
			\end{itemize}
	\end{itemize}
\end{definizione}
Quindi una macchina di Turing può utilizzare una memoria illimitata.

Ci sono diverse varianti dell'automa, ad esempio con un nastro infinito solo da un lato, o più nastri.

Assumeremo che gli stati finali siano stati \textit{halting}, quindi che non ci siano transizioni da essi, e inoltre assumeremo che il blank non possa essere scritto.

\`E importante notare che la macchina potrebbe entrare in un loop.

Nel caso delle macchine di Turing, il modello nondeterministico e il modello deterministico sono gli stessi, se non ci sono vincoli di risorse.
Infatti una macchina nondetermininistica in ogni momento può fare delle scelte, nella versione deterministica si utilizza una BFS.

Si può mostrare che la classe dei linguaggi riconosciuti da questa macchina sono i linguaggi di tipo 0.
Infatti si può pensare di, tenendo l'input sul nastro, e applicando tutte le produzioni al contrario finché non si trova l'assioma, opzionalmente andando avanti in eterno.
Possiamo avere quindi tre possibili risposte
% fig. 22.1
per questo i linguaggi sono detti ricorsivi numberabili.

Possiamo vedere le macchine di Turing anche come macchine che calcolano funzioni, e le funzioni calcolate da una macchina di Turing sono le stesse calcolate da un linguaggio di programmazione generico.

Vediamo ora alcuni problemi indecibili per le macchine di Turing.
\begin{itemize}
	\item problema dell'arresto (HALT), tale che abbiamo in input una macchina di Turing $M$ e una stringa $x \in \Sigma^*$, e ci chiediamo se la macchina termini su input $x$
	\item vuotezza del linguaggio riconosciuto, tale che abbiamo in input una macchina di Turing $M$ e ci chiediamo se $L(M) = \varnothing$
\end{itemize}

\begin{definizione}[Configurazione]
Definiamo ora la configurazione di una macchina di Turing.
Questa non contiene l'intero nastro infinito, ma solo la parte non-blank.
Suponiamo che la parte a sinistra della testina sia chiamata $x$ e che la parte della stringa da dove la testina si trova all'estremo destro si chiami $y$, una configurazione è
$$ xqy $$
con $q \in Q$ e $x, y \in \Gamma^*$.
\end{definizione}

\begin{definizione}[Configurazione iniziale]
	La configurazione iniziale su input $w$ è
	$$ q_0w $$
\end{definizione}
\begin{definizione}[Configurazione finale]
	Una configurazione è finale o acecttante se lo stato $q$ è finale.
\end{definizione}

\begin{definizione}[Mossa]
	Una mossa, scritta $C \vdash C'$, o nel caso di più mosse $C \overset{*}{\vdash*} C'$ è una applicazione di $\delta$.
\end{definizione}

\begin{definizione}[Linguaggio riconosciuto da una macchina di Turing]
	Data la definizione di mossa definiamo il linguaggio riconosciuto da una macchina di Turing $M$ come
	$$ L(M) = \{ w \in \Sigma^* \mid q_0w \overset{*}{\vdash} xqy, x, y \in \Gamma^*, q \in F \} $$
\end{definizione}
Supponiamo di trovarci nella configurazione
% fig 22.2.
possiamo avere tre casi di mossa $\delta(q, a)$
\begin{itemize}
	\item $(p, b, 0)$ fa passare alla configurazione $xqay' \vdash xpby'$
	\item $(p, b, +1)$ fa passare alla configurazione $xqay' \vdash xbpy'$
	\item $(p, b, -1)$ fa passare alla configurazione $xqay' \vdash x'pcby'$
\end{itemize}

Chiamiamo (questo cancelletto non è il blank, è un altro)
$$ \Delta = Q \cup \Gamma \cup \{\# \} $$
e definiamo un linguaggio su questo alfabeto
$$ L'_{succ} = \{ \alpha \# \beta^{R} \mid \text{$\alpha$ e $\beta$ sono configurazioni di $M$ e $\alpha \vdash \beta$} \} $$
Per riconoscere questo basta un automa a pila, infatti ad esempio sia $\alpha = xqay'$ e $\beta = y'^Rbpx^R$, carico $x$ e con la funzione di transizione calcolo con cosa sostituire $qa$ e carico $y'$ e poi scarico dopo il cancelletto.
Definiamo anche
$$ L''_{succ} = \{ \alpha^R \# \beta \mid \dots \} $$
come sopra.
Questi automi a pila sono deterministici, visto che il successore è deterministico.

Definiamo ora il linguaggio delle computazioni valide per un automa $M$, o le computazioni accettanti
$$ \text{valid}(M) = \{ \alpha_1 \# \alpha_2^R \# \alpha_3 \# \dots \# \alpha_k^{(R)} \mid \alpha_i \in (Q \cup \Gamma)^* \} $$
$(R)$ indica che se $k$ è dispari allora non è rovesciato, altrimenti sì.
E devono valere le seguenti condizioni:
\begin{itemize}
	\item per $i \in 1, \dots, k$ deve valere che $\alpha_i \in \Gamma^* Q \Gamma^*$, cioè $\alpha_i$ rappresenta una configurazione di $M$
	\item $\alpha_1$ rappresenta una configurazione iniziale
	\item $\alpha_k$ rappresenta una configurazione accettante
	\item per $i \in 2, \dots, k$, deve valere che $\alpha_{i - 1} \vdash \alpha_i$, cioè $\alpha_i$ è raggiunto in una mossa da $\alpha_{i - 1}$
\end{itemize}
La prima, la seconda e la terza condizioni possono essere controllate da automi a stati finiti.
Per la quarta condizione, abbiamo visto che un passaggio è il linguaggio $L'_{succ}$ (o $L''_{succ}$), ma confrontare dopo aver confrontato $\alpha_1$ e $\alpha_2^R$, abbiamo distrutto la pila per $\alpha_3$.
Quello che possiamo fare con un automa a pila è confrontare $\alpha_1$ e $\alpha_2^R$, $\alpha_3$ e $\alpha_4^R$ e così via.
Poi si può fare un secondo automa che confronta $\alpha_2^R$ e $\alpha_3$, $\alpha_4^R$ e $\alpha_5$, e così via.
Quindi questo linguaggio è l'intersezione di due linguaggi CF.

\begin{teorema}
	Esistono due linguaggi CF $L_1$ e $L_2$ tali che $\text{valid}(M) = L_1 \cap L_2$, ed esiste un algoritmo che data $M$ costruisce PDA (deterministico) o CFG per $L_1$ e $L_2$.
\end{teorema}
Se la macchina $M$ riconosce l'insieme vuoto, quindi
$$ L(M) = \varnothing $$
sse non ci sono computazioni valide, quindi
$$ \text{valid}(M) = \varnothing $$
sse $L_1 \cap L_2 = \varnothing$.
Quindi non è possibile decidiere se l'intersezione di due linguaggi CF è vuota, perché se si potesse allora potremmo anche decidere se un certo linguaggi riconosciuto da una macchina di Turing è vuota.
\begin{corollario}
	Non è possibile decidere se l'intersezione di due CFL (e anche DCFL) è vuota.
\end{corollario}

Vediamo ora invece
$$ ( \text{valid}(M) )^C = \Delta^* \setminus \text{valid}(M) $$
Costruiamo un dispositivo nondeterministico che scommette quele delle quattro condizioni non sia rispettata.
\begin{itemize}
	\item se la prima è infranta basta controllare che tra due cancelletti non ci sia uno stato, o ci siano più stati.
	\item ...
	\item ...
	\item esiste almeno un $i$ tale che $\alpha_i$ non è raggiunta da una mossa da $\alpha_{i - 1}$.
		Questo è verificabile con una pila.
\end{itemize}
Le prime proprietà sono controllabili ancora con automi a stati finiti.
Mentre la quarta è riconoscibile da una pila.
\begin{teorema}
	$(\text{valid}(M))^C \in \text{CFL}$ ed esiste un algoritmo che, data $M$ costruisce una CFG o un PDA per $(\text{valid}(M))^C$.
\end{teorema}
Vale che $L = \varnothing$ sse $\text{valid}(M) = \varnothing$ sse $(\text{valid}(M))^C = \Delta^*$.
Se esistesse un algoritmo per determinare se un $L \in \text{CFL} = \Delta^*$, allora potremmo determinare che $L(M) = \varnothing$.
\begin{corollario}[Problema dell'universalità]
	Dato $L \in \text{CFL}$ non è possibile determinare (è indecidibile) se $L = \Sigma^*$.
\end{corollario}
Invece visto che i DCFL sono chiusi rispetto al complemento, basta costruire l'automa a pila per un linguaggio e vedere se il suo complemento riconosce il linguaggio vuoto.
\begin{corollario}[Problema dell'equivalenza]
	Dati $L_1, L_2 \in \text{CFL}$ non è possibile decidere se $L_1 = L_2$.
\end{corollario}
Questo perché il problema dell'universalità è un caso particolare dell'equivalenza ($L_1 = \Delta^*$).

\begin{corollario}[Problema dell'equivalenza con regolari]
	Dati $L \in \text{CFL}$ e $L \in \text{Reg}$, non è possibile decidere se $L = R$.
\end{corollario}
La motivazione è la stessa di sopra.

Invece dati due DPDA è possibile determinare se sono equivalenti.

%%%
Un automa a pila con due pile ha la stessa potenza computazionale di una macchina di Turing.
\end{document}




\documentclass[12pt]{article}
\usepackage{amsmath}
\usepackage{amsfonts}
\usepackage{amssymb}
\usepackage{amsthm}
\usepackage[italian]{babel}
\usepackage{bytefield}
\usepackage{cancel}
\usepackage{embedall}\embedfile{\jobname.tex}
\usepackage[bookmarks]{hyperref}
\usepackage{listings}
\usepackage[scr=rsfs]{mathalpha}
\usepackage{siunitx}
\usepackage{tabularray}
\usepackage{tcolorbox}
\usepackage{tikz}\usetikzlibrary{automata, decorations.pathmorphing, positioning, decorations.pathreplacing, calligraphy}
\usepackage{todonotes}
\usepackage{xcolor}

\newtheorem{teorema}{Teorema}
\newtheorem{corollario}{Corollario}
\newtheorem{proposizione}{Proposizione}
\newtheorem{proprietà}{Proprietà}
\newtheorem{fatto}{Fatto}

\renewcommand\qedsymbol{$\blacksquare$}

\usepackage{teolang}

\definecolor{codegray}{gray}{0.95}

\lstdefinestyle{mystyle}{
  numberstyle=\tiny,
  basicstyle=\footnotesize,
  breaklines=true,
  numbers=left,
  numbersep=5pt,
}
\lstset{style=mystyle}

\overfullrule=0.2cm
\begin{document}
\tableofcontents
\newpage

\section{Corrispondenza tra grammatiche di tipo 3 e automi a stati finiti}
\begin{teorema} 
	Dato un automa a stati finiti posso costruire una grammatica equivalente e viceversa.
	Per ora esclundendo le $\epsilon$-regole e mosse.
\end{teorema}
\begin{proof}
	Iniziamo dal primo lato, da un automa 
	$$ \mathcal{A} = \langle Q, \Sigma, \delta, q_0, F \rangle $$
	costruiamo la grammatica che genera lo stesso linguaggio (salvo la parola vuota se questa è in $L(\mathcal{A})$).
	La grammatica è della forma
	$$ G = \langle Q, \Sigma, P, q_0 \rangle $$
	Le produzioni sono costruite a partire dalle transizioni dell'automa: per ogni transizione $\delta(q, a) = p$, aggiungo la produzione $q \rightarrow a p$ e la produzione $q \rightarrow a$ se $p \in F$.

	Ora mostriamo il caso opposto.
	Sia
	$$ G = \langle V, \Sigma, P, S \rangle $$
	e l'automa corrispondente
	$$ \mathcal{A} = \langle Q, \Sigma, \delta, q_0, F \rangle $$
	con 
	\begin{align*}
		Q &= V \cup \{q_F\} \\
		F &= \{ q_F \} \\
		q_0 &= S 
	\end{align*}
	con $q_F$ nuovo.
	Per ogni produzione in $G$ definisco $\delta$
	$$
	\delta(A, a) = 
	\begin{cases}
		B \in \delta(A, a) & \text{per } A \rightarrow a B \\
		q_F \in \delta(A, a) & \text{per } A \rightarrow a
	\end{cases}
	$$

	Ovviamente l'automa generato è non deterministico.
\end{proof}

\begin{tcolorbox}
	Ad esempio costruiamo la grammatica corrispondente a 
	\begin{center}
		\begin{tikzpicture}
			\node[state, initial, accepting] 	(q0) {$q_0$};
			\node[state] 				(q1) [right=of q0] {$q_1$};

			\path[->] (q0)	edge[loop below] node[below]	{\tiny b}	()
					edge[bend left]	 node[above]	{\tiny a}	(q1)
				  (q1)	edge[loop below] node[below]	{\tiny a}	()
					edge[bend left]	 node[below]	{\tiny b}	(q0);
		\end{tikzpicture}
	\end{center}
	Questo corrisponde alla grammatica
	$$ G = \langle \{ q_0, q_1 \}, \{ a, b \}, P, q_0 \rangle$$
	con $P$ definita come
	\begin{align*}
		q_0 &\rightarrow a q_1 \\
		q_0 &\rightarrow b q_0 \\
		q_1 &\rightarrow a q_1 \\
		q_1 &\rightarrow b q_0 \\
		q_1 &\rightarrow b \\
		q_0 &\rightarrow b
	\end{align*}
	E si può dimostrare che
	$$ q_0 \overset{*}{\Rightarrow} x p, x \in \Sigma^* \Leftrightarrow \delta(q_0, x) = p $$
\end{tcolorbox}
\begin{tcolorbox}
	Data la grammatica
	$$ G = \langle \{S, A\}, \{a, b\}, P, S \rangle$$
	con $P$ definita come
	\begin{align*}
		S &\rightarrow a A \mid b S \\
		A &\rightarrow a A \mid a 
	\end{align*}
	Otteniamo
	\begin{center}
		\begin{tikzpicture}
			\node[state, initial] 	(q0) {$q_0$};
			\node[state] 		(q1) [right=of q0] {$q_1$};
			\node[state, accepting] (q2) [below=of q1] {$q_2$};

			\path[->] (q0)	edge[loop below] node[below]	{\tiny b}	()
					edge		 node[above]	{\tiny a}	(q1)
				  (q1)	edge[loop above] node[above]	{\tiny a}	()
					edge	 	 node[left]	{\tiny a}	(q2);
		\end{tikzpicture}
	\end{center}
\end{tcolorbox}

\section{Limiti inferiori di automi non deterministici}
\begin{tcolorbox}
	Dimostriamo la non-regolarità di un linguaggio dal suo insieme di parole distinguibili
	$$ L = \{ w \in \Sigma^* \mid \#_a(w) = \#_b(w) \} $$
	intuitivamente visto che gli automi a stati finiti hanno una memoria finita, non possono contare un numero arbitrario di valori.

	Si può dimostrare che per l'insieme
	$$ X = \{ a^n \mid n \geq 0 \} $$
	tutti i suoi elementi sono distinguibili, quindi $L$ non può essere regolare.
\end{tcolorbox}

Questo limite sul minimo numero di stati non vale per automi non deterministici.
\begin{tcolorbox}
	Dato il linguaggio
	$$ L_n = \{ x \in \{a, b\}^* \mid P(x) \} $$
	Dove $P$ è la proprietà per cui $x$ contiene due simboli uguali tra loro a distanza $n$.

	Ad esempio $ababba \in L_3$, infatti la seconda e l'ultima $a$ e la prima e l'ultima $b$ corrispondono.
	Costruiamo l'automa per $L_3$
	\begin{center}
		\begin{tikzpicture}
			\node[state, initial] 	(q0) {$q_0$};
			\node[state] 		(q1) [above right=of q0] {$q_1$};
			\node[state] 		(q2) [below right=of q0] {$q_2$};
			\node[state] 		(q3) [right=of q1] {$q_3$};
			\node[state] 		(q4) [right=of q2] {$q_4$};
			\node[state] 		(q5) [right=of q3] {$q_5$};
			\node[state] 		(q6) [right=of q4] {$q_6$};
			\node[state, accepting]	(q7) [below right=of q5] {$q_7$};

			\path[->] (q0)	edge[loop above] node[above]	{\tiny a,b}	()
					edge		 node[above]	{\tiny a}	(q1)
					edge		 node[above]	{\tiny b}	(q2)
				  (q1)  edge		 node[above]	{\tiny a, b}	(q3)
				  (q2)  edge		 node[above]	{\tiny a, b}	(q4)
				  (q3)  edge		 node[above]	{\tiny a, b}	(q5)
				  (q4)  edge		 node[above]	{\tiny a, b}	(q6)
				  (q5)  edge		 node[above]	{\tiny a}	(q7)
				  (q6)  edge		 node[above]	{\tiny b}	(q7)
				  (q7)	edge[loop below] node[below]	{\tiny a,b}	();
		\end{tikzpicture}
	\end{center}
	Questo ha $2n + 2$ stati, l'automa deterministico corrispondente deve ricordare la finestra degli utlimi $n$ simboli, quindi dovrebbe avere almeno $2^n$ stati.
	Si può anche ragionare sull'insieme di stringhe distinguibili, infatti scegliamo
	$$ X = \{a, b\}^n $$
	queste sono tutte distinguibili tra loro?
	Prendiamo due stringhe $x, y \in X$, con $ x \neq y $, tali che
	\begin{align*}
		x &= x_1 x_2 \dots x_i \dots x_n \\
		y &= y_1 y_2 \dots y_i \dots y_n 
	\end{align*}
	Sia $i$ la prima la posizione in cui le due stringhe sono diverse.
	Sia $z = \overline{x_1} \; \overline{x_2} \dots \overline{x_{i - 1}} a$.
	Quindi abbiamo trovato un insieme di $2^n$ stringhe distinguibili, quindi ogni DFA ha almeno $2^n$ stati.
\end{tcolorbox}
\begin{tcolorbox}
	Sia
	$$ L^\prime_n = \{ x \in \{a, b\}^* \mid P(x) \} $$
	Dove $P$ è la proprietà per cui per ogni coppia di simboli di $x$ a distanza $n$ i due coincidono.

	Quindi per esempio $abbabba \in L^\prime_3$.
	Alternativamente possiamo dire che $x \in L^\prime_n$ sse $\exists w \in \{a, b\}^n$ e $\exists y$ prefisso di $w$ per cui $\exists k \geq 0 \mid x = w^k y$.
	% fig 1
	\begin{center}
		\begin{tikzpicture}[
				scale=0.3,
				every node/.append style={transform shape},
				level 1/.style={sibling distance=20cm},
				level 2/.style={sibling distance=10cm},
				level 3/.style={sibling distance=5cm}
			]

			\node[state, accepting] (root) {0}
				child { node[state, accepting] {1} 
					child { node[state, accepting] {2} edge from parent[->]
						child { node[state, accepting] {3} edge from parent[->] node[above left]  {a} }
						child { node[state, accepting] {4} edge from parent[->] node[above right] {b} }
						edge from parent[->] node[above left]  {a}
					} 
					child { node[state, accepting] {7} 
						child { node[state, accepting] {8} edge from parent[->] node[above left]  {a} }
						child { node[state, accepting] {11} edge from parent[->] node[above right] {b} }
						edge from parent[->] node[above right] {b}
					} edge from parent[->] node[above left] {a}
				}
				child { node[state, accepting] {14} 
					child { node[state, accepting] {15} 
						child { node[state, accepting] {16} edge from parent[->] node[above left] {a} }
						child { node[state, accepting] {19} edge from parent[->] node[above right] {b} }
						edge from parent[->] node[above left] {a}
					}
					child { node[state, accepting] {22} 
						child { node[state, accepting] {23} edge from parent[->] node[above left] {a} }
						child { node[state, accepting] {26} edge from parent[->] node[above right] {b} }
						edge from parent[->] node[above right] {b}
					} edge from parent[->] node[above right] {b}
				};
			\draw[->, loop below] (root-1-1-1) to node[below] {a} ();

			\node[state, accepting] (a21) [below left=of root-1-1-2] {5};
			\node[state, accepting] (a22) [below right=of root-1-1-2] {6};
			\draw[->] (root-1-1-2) to node[above left] {a} (a21);
			\draw[->] (a21) to node[above] {a} (a22);
			\draw[->] (a22) to node[above right] {b} (root-1-1-2);

			\node[state, accepting] (a31) [below left=of root-1-2-1] {9};
			\node[state, accepting] (a32) [below right=of root-1-2-1] {10};
			\draw[->] (root-1-2-1) to node[above left] {a} (a31);
			\draw[->] (a31) to node[above] {b} (a32);
			\draw[->] (a32) to node[above right] {a} (root-1-2-1);

			\node[state, accepting] (a41) [below left=of root-1-2-2] {12};
			\node[state, accepting] (a42) [below right=of root-1-2-2] {13};
			\draw[->] (root-1-2-2) to node[above left] {a} (a41);
			\draw[->] (a41) to node[above] {b} (a42);
			\draw[->] (a42) to node[above right] {b} (root-1-2-2);

			\node[state, accepting] (a51) [below left=of root-2-1-1] {17};
			\node[state, accepting] (a52) [below right=of root-2-1-1] {18};
			\draw[->] (root-2-1-1) to node[above left] {b} (a51);
			\draw[->] (a51) to node[above] {a} (a52);
			\draw[->] (a52) to node[above right] {a} (root-2-1-1);

			\node[state, accepting] (a61) [below left=of root-2-1-2] {20};
			\node[state, accepting] (a62) [below right=of root-2-1-2] {21};
			\draw[->] (root-2-1-2) to node[above left] {b} (a61);
			\draw[->] (a61) to node[above] {a} (a62);
			\draw[->] (a62) to node[above right] {b} (root-2-1-2);

			\node[state, accepting] (a71) [below left=of root-2-2-1] {24};
			\node[state, accepting] (a72) [below right=of root-2-2-1] {25};
			\draw[->] (root-2-2-1) to node[above left] {b} (a71);
			\draw[->] (a71) to node[above] {b} (a72);
			\draw[->] (a72) to node[above right] {a} (root-2-2-1);

			\draw[->, loop below] (root-2-2-2) to node[below] {b} ();

			\node (empty) [above=of root] {};
			\draw[->] (empty.south) to (root.north);
		\end{tikzpicture}
	\end{center}
	Questo DFA è maggiore di $2^n$ stati.

	Ma anche l'NFA corrispondente non ci permette di risparmiare un gran numero di spazi.

	Come mai l'NFA corrispondente a questo e al precedente hanno una così grande differenza di stati?
	Il primo utilizza un esiste e questo un per ogni, e il non determinismo è utile per l'esiste.
\end{tcolorbox}
Dato un linguaggio $L \subseteq \Sigma^*$ e $P \subseteq \Sigma^* \times \Sigma^* = \{ (x_i, y_i) \mid i \in 1, \dots n\}$, diciamo che $P$ è un \textbf{fooling set} per $L$ sse valgono le seguenti condizioni
\begin{itemize}
	\item $\forall i \in 1, \dots n \mid x_i y_i \in L$
	\item $\forall i, j \in 1, \dots n \mid i \neq j \Rightarrow x_i y_j \not \in L$
\end{itemize}
Noi in particolare ci interessiamo all'\textit{extended} fooling set, con la seconda condizione modificata:
\begin{itemize}
	\item $\forall i, j \in 1, \dots n \mid i \neq j \Rightarrow x_i y_j \not \in L \vee x_j y_i \not \in L $
\end{itemize}

\begin{teorema}
	Sia $L \subseteq \Sigma^*$ e $P$ un extended fooling set per $L$, allora ogni automa non deterministico per $L$ ha almeno $|P|$ stati.
\end{teorema}
\begin{proof}
	Sia $\mathcal{A}$ un NFA per $L$.
	Per ogni membro $(x_i, y_i) \in P$ abbiamo che
	$$ \delta(q_0, x_i) = P_i  \text{ e } \delta(P_i, y_i) = F_i $$
	con $F_i$ finale.
	\begin{center}
		\begin{tikzpicture}[
				SQUIGGLY/.style={->
				, decorate
				, decoration={snake,amplitude=.4mm,segment length=2mm,post length=1mm}},
			]
			\node[state, initial] 	(q0) {};
			\node[state]		(q1) [above right=of q0] {$P_1$};
			\node 			(dots1) [below=of q1] {$\vdots$};
			\node 			(dots2) [below=of q3] {$\vdots$};
			\node[state]		(q2) [below right=of q0] {$P_n$};
			\node[state, accepting] (q3) [right=of q1] {};
			\node[state, accepting] (q4) [right=of q2] {};

			\draw[SQUIGGLY, bend left] 	(q0)	to node[above left] {$x_1$} (q1);
			\draw[SQUIGGLY] 		(q0)	to (dots1);
			\draw[SQUIGGLY, bend right] 	(q0)	to node[below left] {$x_n$} (q2);
			\draw[SQUIGGLY]			(q1)	to node[above]      {$y_1$} (q3);
			\draw[SQUIGGLY]			(q2)	to node[below]	    {$y_n$} (q4);
			\draw[SQUIGGLY] 		(dots1)	to (dots2);
		\end{tikzpicture}
	\end{center}

	Ora supponiamo per assurdo che per due elementi del fooling set $P$ $x_i$ e $x_j$ portino entrambi nello stato $P_i$.
	\begin{center}
		\begin{tikzpicture}[
				SQUIGGLY/.style={->
				, decorate
				, decoration={snake,amplitude=.4mm,segment length=2mm,post length=1mm}},
			]
			\node[state, initial] 	(q0) {};
			\node[state]		(q1) [right=of q0] {$P_i$};
			\node[state, accepting] (q2) [above right=of q1] {};
			\node[state, accepting] (q3) [below right=of q1] {};

			\draw[SQUIGGLY, bend left]  (q0) to node[above] {$x_i$} (q1);
			\draw[SQUIGGLY, bend right] (q0) to node[below] {$x_j$} (q1);
			\draw[SQUIGGLY, bend left ] (q1) to node[above left] {$y_i$} (q2);
			\draw[SQUIGGLY, bend right] (q1) to node[below left] {$y_j$} (q3);
		\end{tikzpicture}
	\end{center}
	Allora è facile vedere che anche le parole $x_i y_j$ e $x_j y_i$ sarebbero accettare, infrangendo la seconda proprietà del fooling set.
	Quindi ad ogni $x_i$ deve essere associato uno almeno uno stato che nessun'altro elemento condivide, quindi ci sono almeno $|P|$ stati.
\end{proof}
 % !!! RECOVER FROM PDF

\begin{tcolorbox}
Ritornando all'esempio di prima, sia $P = \{ (x, x) \mid x \in \{a, b\}^n \}$, vale che ogni elemento $xx \in L^\prime_n$, mentre per elementi diversi, $xy \not \in L^\prime_n$.
Quindi questo è un fooling set valido per $L^\prime_n$.
E dato che l'NFA ha almeno $|P|$ stati, ne servono almeno $2^n$.
\end{tcolorbox}

\begin{tcolorbox}
	Prendiamo il linguaggio
	$$ L_k = \{ 0^n 1^n 2^n \mid n \leq k \} $$
	questo è rappresentabile dal DFA
	\begin{center}
		\begin{tikzpicture}[
				SQUIGGLY/.style={->
				, decorate
				, decoration={snake,amplitude=.4mm,segment length=2mm,post length=1mm}},
			]
			\node[state, initial] 	(q0) {};
			\node[state]		(q1) [right=of q0] {};
			\node[state]		(q2) [right=of q1] {};
			\node			(dots) [right=of q2] {$\dots$};
			\node[state]		(qn) [right=of dots] {};

			\node[state]		(q11) [below=of q1] {};
			\node[state, accepting] (q12) [below=of q11] {};

			\node[state]		(q21) [below=of q2] {};
			\node[state]		(q22) [below=of q21] {};
			\node[state]		(q23) [below=of q22] {};
			\node[state, accepting]	(q24) [below=of q23] {};

			\node[state]		(qn1) [below=of qn] {};
			\node			(dotsn1) [below=of qn1] {$\vdots$};
			\node[state]		(qn2) [below=of dotsn1] {};
			\node			(dotsn2) [below=of qn2] {$\vdots$};
			\node[state, accepting]	(qnn) [below=of dotsn2] {};

			\draw[thick, decorate, decoration = {brace}] ([yshift=0.25cm] q0.north) -- ([yshift=0.25cm] qn.north) node[pos=0.5, above] {\tiny $k$ volte};
			\draw[thick, decorate, decoration = {brace}] ([xshift=0.25cm] qn.east) -- ([xshift=0.25cm, yshift=0.1cm] qn2.east) node[pos=0.5, right] {\tiny $k$ volte};
			\draw[thick, decorate, decoration = {brace}] ([xshift=0.25cm, yshift=-0.1cm] qn2.east) -- ([xshift=0.25cm] qnn.east) node[pos=0.5, right] {\tiny $k$ volte};

			\path[->] (q0) 	 edge	node[above]	{\tiny 0}	(q1)
			(q1)	 edge	node[above]	{\tiny 0}	(q2)
			edge	node[right]	{\tiny 1}	(q11)
			(q2)	 edge	node[above]	{\tiny 0}	(dots)
			edge	node[right]	{\tiny 1}	(q21)
			(dots) edge	node[above]	{\tiny 0}	(qn)
			(qn)	 edge	node[right]	{\tiny 1}	(qn1)
			(q11)	 edge	node[right]	{\tiny 2}	(q12)
			(q21)	 edge	node[right]	{\tiny 1}	(q22)
			(q22)	 edge	node[right]	{\tiny 2}	(q23)
			(q23)	 edge	node[right]	{\tiny 2}	(q24)
			(qn1)	 edge	node[right]	{\tiny 1}	(dotsn1)
			(dotsn1) edge node[right]	{\tiny 1}	(qn2)
			(qn2)	 edge	node[right]	{\tiny 2}	(dotsn2)
			(dotsn2) edge	node[right]	{\tiny 2}	(qnn);
		\end{tikzpicture}
	\end{center}
	Ma è facile mostrare che anche l'automa deterministico corrispondente non ci permette di risparmiare molto.
	% calcolo del fooling set
\end{tcolorbox}
\end{document}

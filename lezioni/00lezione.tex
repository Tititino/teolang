\documentclass[12pt]{article}
\usepackage{amsmath}
\usepackage{amsfonts}
\usepackage[italian]{babel}
\usepackage{bytefield}
\usepackage{embedall}\embedfile{\jobname.tex}
\usepackage[bookmarks]{hyperref}
\usepackage{listings}
\usepackage{siunitx}
\usepackage{tabularray}
\usepackage{tcolorbox}
\usepackage{tikz}
\usepackage{todonotes}
\usepackage{xcolor}

\definecolor{codegray}{gray}{0.95}

\lstdefinestyle{mystyle}{
  numberstyle=\tiny,
  basicstyle=\footnotesize,
  breaklines=true,
  numbers=left,
  numbersep=5pt,
}
\lstset{style=mystyle}

\overfullrule=0.2cm
\begin{document}
\tableofcontents
\newpage

\section{Convenevoli}
Studieremo dei sistemi particolari (macchine/grammatiche) e il loro comportamento (il linguaggio descritto) e le risorse utilizzate (risorse utilizzate, risorse per trasformazioni da una macchina e l'altra, \dots). 

\section{Definizione di un linguaggio}
Quando si parla di un linguagio ci sono due aspetti fondamentali:
\begin{itemize}
	\item sintassi, formata da
	\begin{itemize}
		\item i segni o simboli utilizzati
		\item le regole che ci permettono di combinarli
	\end{itemize}
	\item semantica, il processo che associa alle frasi del linguaggio un significato
\end{itemize}

Noi ci occuperemo principalmente di sintassi.

In ambito linguistico nel 1956 Chomsky introduce le grammatiche formali.
Questi sono degli oggetti matematici, il cui scopo originale era descrivere in modo rigoroso la grammatica dell'inglese.
Nello stesso periodo si stava sviluppando l'idea di compilatore, specificamente il compilatore Fortran.
Si vide che il modello proposto da Chomksy andava benissimo per formalizzare i linguaggi di programmazione.

Definiamo
\begin{itemize}
	\item l'alfabeto ($\Sigma$) è un insieme finito non vuoto di simboli, e.g. 
		$$ \Sigma = { a , b , c } $$
	\item una stringa o parola è una sequenza di simboli sull'alfabeto.
		La sua lunghezza è definita come il numero di simboli che la compongono
		$$ | abc | = 3 $$
		Tra tutte le parole abbiamo una parola speciale, che è la parola vuota costituita da zero simboli.
		Questa è indicata con $\epsilon$ (alcuni la indicano con $\lambda$ o $\Lambda$).
		La lunghezza di questa è 0.

		Sia $a \in \Sigma$ e $x \in \Sigma^*$, definiamo 
		$$ \#_a(x) : \Sigma \times \Sigma^* \mapsto \mathbb{N} $$
		come il numero di $a$ in $x$.
	\item indichiamo l'insieme di tutte le possibili stringhe sull'alfabeto $\Sigma$ con $\Sigma^*$.
		Queste sono enumerabili.

		Chiamiamo $\Sigma^n$ l'insieme di tutte le stringhe di $n$ lettere sull'alfabeto $\Sigma$.
	\item definiamo un'operazione 
		$$ \cdot : \Sigma^* \times \Sigma^* \mapsto \Sigma^*$$
		come la concatenazione o prodotto di due stringhe.
		Ad esmepio 
		\begin{align*}
			x &= abbac \\
			y &= ca \\
			xy &= abbacca \\
			yx &= caabbac
		\end{align*}
		La parola vuota $\epsilon$ è l'identità sia destra che sinistra di questa operazione.
		L'operazione è associativa ma non commutativa.
		La struttura $\langle \Sigma^*, \cdot, \epsilon \rangle$ è detto monoide libero generato dall'alfabeto $\Sigma$.
	\item data una stringa $x \in \Sigma^*$ diciamo che $y$ è prefisso di $x$ se
		$$ \exists z \in \Sigma^* | x = yz $$
		Una stringa ha un numero di prefissi pari alla sua lunghezza + 1.
		$y$ è prefisso proprio di $x$ se è prefisso e non è $\epsilon$.
		Similmente definiamo il suffisso di una parola $x$, quindi $y$ è suffisso di $x$ se 
		$$ \exists x \in \Sigma^* | x = zy $$
		Infine diciamo che $y$ è fattore di $x$ se
		$$ \exists z, w \in \Sigma^* | x = zyw $$
		I prefissi e i suffissi sono particolari tipi di fattori.
		Ogni parola ha all'incirca $\frac{n^2}{2}$ fattori, infatti è facile notare che $x = aaaa$ ha diversi fattori uguali.

		Una sottosequenza di solito è una serie di elementi scelti da una parola, quindi un fattore può essere definito come una sottosequenza contigua. 
		Ad esmepio una sottosequenza di $x = abbac$ può essere $abc$.
		Sottostringa e sottoparola di solito sono sinonimi di fattore, ma a volte possono essere utilizzate come sinonimo di sottosequenza.
	\item un linguaggio $L$ è definito come un qualunque sottoinsieme di $\Sigma^*$.
		Abbiamo alcuni linguaggi particolari:
		\begin{itemize}
			\item il linguaggio vuoto $L = \empty$
			\item il linguaggio della parola vuota $L = { \epsilon }$
			\item il linguaggio pieno $L = \Sigma^*$
		\end{itemize}
		Ad esempio definiamo il linguaggio delle parole che finiscono con due $a$ come
		$$ L = { w \in \Sigma^* | \exists y \in \Sigma^* w = yaa } $$
		quindi con un insieme finito di simboli siamo risuciti a rappresentare un insieme infinito.
		Questo tipo di descrizione è detta dichiarativa.

		Definiamo l'alfabeto $\Sigma = { 0, 1, \dots, 9 }$, e definiamo $L$ come tutte le sequenze formate da 3 cifre decimali che appaiono consecutivamente nella rappresentazione del $\pi$ greco.

		In altri casi utilizzeremo delle descrizioni:
		\begin{itemize}
			\item generative: fornisco un insieme di regole che permette di costruire tutte le possibili stringhe del linguaggio, ad esempio le grammatiche
			\item riconoscitive: fornisco un riconoscitore che riceve una stringa $x$ e risponde alla domanda $x \overset{?}{\in} L$, in alcuni casi queste macchine si limitano a dire sì se la stringa appartiene o non terminare in altro caso
		\end{itemize}

		Non tutti i linguaggi si possono descrivere in maniera finita.
		Prendiamo come alfabeto $\Sigma = { (, ) }$, e voglio descrivere $L$ l'insieme delle sequenze di parentesi bilanciate correttamente.
		Definiamo la rappresentazione generativa che genera $L_G$:
		\begin{enumerate}
			\item $\epsilon \in L_G$
			\item se $x \in L_G$ allora $(x) \in L_G$
			\item se $x, y \in L_G$ allora $xy \in L_G$
		\end{enumerate}

		E definiamo un riconoscitore, che riconosca $L_R$:
		\begin{itemize}
			\item il numero di aperte deve essere uguale al numero di chiuse, quindi $\#_((x) = \#_)(x)$
			\item per ogni prefisso il numero di parentesi aperte è maggiore o uguale del numero di chiuse, quindi per $y$ prefisso $\#_((y) \geq \#_)(y)$.
		\end{itemize}
		Questo può essere generato come un automa con contatore.

		Si può dimostrare che questi due definiscono lo stesso linguaggio. \todo{Da fare}

		Ora prendiamo l'alfabeo $\Sigma = {(, ), [, ]}$, e definiamo $L$ come il linguaggio delle sequenze di parentesi bilanciate correttamente.
		Definiamo la rappresentazione generativa che genera $L_G$:
		\begin{enumerate}
			\item $\epsilon \in L_G$
			\item se $x \in L_G$ allora $(x) \in L_G$ e $[x] \in L_G$
			\item se $x, y \in L_G$ allora $xy \in L_G$
		\end{enumerate}

		E definiamo un riconoscitore, che riconosca $L_R$:
		\begin{itemize}
			\item il numero di aperte deve essere uguale al numero di chiuse, quindi $\#_((x) = \#_)(x)$ e $\#_[(x) = \#_](x)$
			\item si utilizza una pila in cui si mette ogni parentesi aperta e si toglie quando si trova una parentesi chiusa nel caso corrisponda
		\end{itemize}
		Quindi un automa a pila.

		In alcuni casi è più facile definire come si genera, e in altri come riconoscere.
\end{itemize}

\section{Materiale}
\begin{itemize}
	\item Hopcroft Ullman, Introduction to Automata Theory and Languages of Computation (?), edizione 1979
		C'è una prima versione con un altro titolo del 1969 che si può reperire lagalmente dall'università; mancano alcune parti perché non erano ancora state scoperte.
	\item Shallit, A second course in ??? and Automata Theory, edizione 2008
\end{itemize}
L'esame è orale e può essere fatto in due modi:
\begin{itemize}
	\item approfondire un arogmento prefissato
	\item approfondire un articolo di ricerca
\end{itemize}
In ogni caso nella parte di orale viene fatto fare un esercizio simile a quelli assegnati durante il corso, non sullo stesso argomento presentato.

Inizio 8:45.

\end{document}

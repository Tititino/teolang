\documentclass[12pt]{article}
\usepackage{amsmath}
\usepackage{amsfonts}
\usepackage{amssymb}
\usepackage{amsthm}
\usepackage[italian]{babel}
\usepackage{bytefield}
\usepackage{cancel}
\usepackage{embedall}\embedfile{\jobname.tex}
\usepackage[bookmarks]{hyperref}
\usepackage{listings}
\usepackage[scr=rsfs]{mathalpha}
\usepackage{siunitx}
\usepackage{tabularray}
\usepackage{tcolorbox}
\usepackage{tikz}\usetikzlibrary{automata}
\usepackage{todonotes}
\usepackage{xcolor}

\newtheorem{teorema}{Teorema}
\newtheorem{corollario}{Corollario}
\newtheorem{proposizione}{Proposizione}
\newtheorem{proprietà}{Proprietà}
\newtheorem{fatto}{Fatto}
\newtheorem{definizione}{Definizione}

\renewcommand\qedsymbol{$\blacksquare$}

\usepackage{teolang}

\definecolor{codegray}{gray}{0.95}

\lstdefinestyle{mystyle}{
  numberstyle=\tiny,
  basicstyle=\footnotesize,
  breaklines=true,
  numbers=left,
  numbersep=5pt,
}
\lstset{style=mystyle}

\overfullrule=0.2cm
\begin{document}
\tableofcontents
\newpage
\section{Operazioni sui linguaggi}
Visto che un linguaggio $L \subseteq \Sigma^*$ è un sottoinsieme di un insieme, possiamo fare tutte le operazioni insiemistiche su un linguaggio:
\begin{itemize}
	\item intersezione $L \cap L^\prime$
	\item unione $L \cup L^\prime$
	\item complemento $L^C$.
		Si può pensare di calcolare il complemento di un linguaggio $L \subseteq \Sigma^*$ rispetto ad un alfabeto più grande $\Sigma \subseteq \Gamma$.
		Il complemento è esattamente $\Gamma^* \setminus L$.
\end{itemize}

Definiamo il prodotto di due linguaggi $L^\prime, L^{\prime\prime} \subseteq \Sigma^*$
$$ L^\prime \cdot L^{\prime\prime} = \{ z \mid \exists x \in L^\prime \exists y \in L^{\prime\prime} . z = xy \} $$
E se $L^\prime \subseteq \Sigma^{*\prime}, L^{\prime\prime} \subseteq \Sigma^{*\prime\prime}$, allora $L^\prime \cdot L^{\prime\prime} \subseteq \Sigma^\prime \cup \Sigma^{\prime\prime}$.
Salvo casi particolari (e.g. alfabeto da una lettera sola) questa operazione non è commutativa.
E vale $\{ \epsilon \}$ è l'indentità destra e sinistra, e $\varnothing$ è l'elemento nullo destro e sinistro.

Definiamo la potenza di un linguaggio
$$ L^k = \underbrace{L \cdot L \dots}_{k \text{ volte}} $$
e vale che 
$$ L^0 = \{ \epsilon \} $$
ovvero il linguaggio ottenuto concatenando zero parole di $L$.
Il prodotto e la potenza di un linguaggio finito sono ancora finiti.

Definiamo la chiusura di Kleene di un linguaggio
$$ L^* = \bigcup_{k \geq 0} L^k $$
La chiusura di Kleene anche di un linguaggio finito è finito.
\begin{tcolorbox}
	\begin{align*}
		L &= \{ a, bb\}^* \\
		  & = \{ x \in \{a, b\}^* \mid \text{ Ogni fattore massimale di $b$ è di lunghezza pari }\}
	\end{align*}
	Dove massimale indica che non può essere allungato.
\end{tcolorbox}
Vale inoltre che
$$ \{\epsilon \}^* = \{ \epsilon \}$$
e che
$$ \varnothing^* = \{ \epsilon \} $$

Definiamo la chiusura di Kleene positiva di un linguaggio
$$ L^+ = \bigcup_{k \geq 1} L^k $$
Inoltre vale che $L$ non contiene la parola vuota, allora $L^+$ a sua volta non la contiene, e questo è uguale a $L^+ = L^* \setminus \{\epsilon\}$.
Nota bene, non vale che generalmente $L^+ = L^* \setminus \{\epsilon\}$.
\begin{fatto}
	$$ L\cdot L^* = L \bigcup_{k \geq 0} L^k = \bigcup_{k \geq 0} L L^k = \bigcup_{k \geq 1} L^k = L^+ = L^* \cdot L $$
\end{fatto}

\section{Espressioni regolari}
Le espressioni regolari sono in un certo senso una descrizione dichiarativa di un linguaggio.
Dato un alfabeto $\Sigma$, definiamo ricorsivamente le espressioni regolari come
\begin{itemize}
	\item $\varnothing$, rappresenta il linguaggio vuoto
	\item $\epsilon$, rappresenta il linguaggio della parola vuota
	\item $a \in \Sigma$, rappresenta il linguaggio di solo $a$
\end{itemize}
ora definiamo
\begin{itemize}
	\item $E_1 + E_2$, rappresenta il linguaggio $L(E_1) \cup L(E_2)$
	\item $E_1 \cdot E_2$, rappresenta il linguaggio $L(E_1) \cdot L(E_2)$
	\item $E^*$, rappresenta il linguaggio $L(E)^*$
\end{itemize}
\begin{tcolorbox}
	L'espressione regolare
	$$ (a + bb)^*$$
	rappresenta il linguaggio $\{a, bb\}^*$.
\end{tcolorbox}
\begin{tcolorbox}
	Il linguaggio dove il terzultimo simbolo è una $a$ è rappresentato dall'espressione regolare
	$$ (a + b)^* a (a + b) (a + b) $$
	E volendo generalizzare al linguaggio il cui $n$-esimo simbolo da destra è una $a$
	$$ (a + b)^* a (a + b)^n $$
	Dove le potenze sulle espressioni regolari rappresentano una serie di prodotti.
\end{tcolorbox}
\begin{tcolorbox}
	Il linguaggio dove due simboli a distanza $n$ sono uguali
	$$ ((a + b)^* a (a + b)^{n - 1} a (a + b)^*) + ((a + b)^* b (a + b)^{n - 1} b (a + b)^*)$$
	Questo può essere semplificato in
	$$ (a + b)^* ((a (a + b)^{n - 1} a) + (b (a + b)^{n - 1} b)) (a + b)^* $$
\end{tcolorbox}

\begin{definizione}[State complexity]
	Definiamo la complessità di stati o state complexity di un linguaggio $L \subseteq \Sigma^*$, indicata $\operatorname{sc}(L)$, come il minimo numero di stati del DFA che accetta $L$.
	E definiamo la non-deterministic state complexity, indicata con $\operatorname{nsc}(L)$, come il minimo numero di stati del NFA che accetta $L$.
\end{definizione}
\begin{fatto}
	$$\operatorname{sc}(L) \leq 2^{\operatorname{nsc}(L)}$$
\end{fatto}
\begin{tcolorbox}
	Dato $L_n = (a + b)^* a (a + b)^{n - 1}$, vale che $\operatorname{nsc}(L_n) \leq n + 1$, mentre $\operatorname{sc}(L_n) \geq 2^n$, e visto che per questo linguaggio sappiamo costruire un automa da $2^n$ stati, allora
	$\operatorname{sc}(L_n) = 2^n$.

	La stringa più corta di questa linguaggio è la stringa che inizia con $a$ ed ha $n$ simboli.
	Supponendo di avere meno di $n$ stati, allora ce ne deve essere almeno uno che si ripete; quindi c'è un loop.
	Se io elimino il loop potrei riconoscere una stringa più breve di quella più breve.

	Quindi abbiamo che per forza $\operatorname{nsc}(L_n) = n + 1$.

	Alternativamente si può costruire il fooling set composto da tutte le possibili scomposizioni in due stringhe di $ab^{n - 1}$.
\end{tcolorbox}
\begin{proposizione}
	Presa la stringa più corta del linguaggio, ogni NFA deve per forza avere $n + 1$ stati, dove $n$ sono i simboli della stringa.
\end{proposizione}

\begin{teorema}[Teorema di Kleene o Teorema fondamentale degli automi a stati finiti]
	La classe dei linguaggi accettati da automi a stati finiti è la più piccola sottoinsieme di $\Sigma^*$ che contiene i linguaggi finiti ed è chiusa rispetto alle operazioni di unione, prodotto e chiusura di Kleene.
\end{teorema}
Quindi la classe degli automi a stati finiti coincide con la classe esprimibile dalle espressioni regolari.
\begin{proof}
	Dimostriamo il primo lato, passandro da un automa ad una regex.
	Dato un automa
	$$ \mathcal{A} = \langle Q, \Sigma, \delta, q_1, F \rangle$$
	e supponiamo che gli stati siano numerati da uno ad $n$
	$$ Q = \{ q_1, q_2, \dots, q_n \} $$
	possiamo vedere come il linguaggio come tutti i cammini dallo stato iniziale agli stati finali, e che concantenando le etichette del cammino otteniamo un parola accettata dal percorso.le etichette del cammino otteniamo un parola accettata dal percorso.


	Costruiamo un algoritmo simile a quello di Floyd-Wharshall che trova tutti i camminini minimi.
	Chiamiamo $R_{ij}^{(k)}$ l'insieme di tutti i percorsi da $q_i$ a $q_j$ che passano solo per stati con indice minore di $k$.
	Quando metteremo $k = n$ avrò trovato tutte le stringhe da $i$ a $j$.
	Andando per induzione definiamo l'insieme come
	\begin{itemize}
		\item $R_{ij}^{(0)}$ è il caso in cui c'è una transizione tra lo stato $i$ e lo stato $j$.
			$$ R_{ij}^{(0)} = 
			\begin{cases}
				\{ \alpha \in \Sigma \mid \delta(q_i, \alpha) = q_j \} & \text{se } i \neq j \\
				\{\epsilon\} \cup \{ \alpha \in \Sigma \mid \delta(q_i, \alpha) = q_i \} & \text{se } i = j 
			\end{cases}
			$$
		\item supponiamo di avere $R_{ij}^{(k - 1)}$, voglio trovare $R_{ij}^{(k)}$.
			Cerco tutti i punti per cui il nuovo cammino passa per $q_k$, nelle sezioni tra i $q_k$ gli stati intermedi sono tutti minori di $k$.
			% Fig 1
			Quindi 
			$$ R_{ij}^{(k)} = R_{ij}^{(k - 1)} \cup R_{ik}^{(k - 1)} \left (R_{kk}^{(k - 1)}\right )^* R_{kj}^{(k - 1)} $$
	\end{itemize}
	Con $k = n$ abbiamo tutti i percorsi e vale che
	$$ L = \bigcup_{q_i \in F} R_{1i}^{(n)} $$

	Dimostriamo il secondo lato, passandro da una regex ad un automa.
	%% DA FINIRE
\end{proof}

\begin{tcolorbox}
	Mostriamo una tecnica diversa per generare la regex da un automa.
	\begin{center}
		\begin{tikzpicture}
			\node[state, initial] 	(q0) {$q_0$};
			\node[state, accepting] (q1) [right=of q0] {$q_1$};

			\path[->] (q0)	edge[loop above] node[above]	{\tiny a}	()
					edge		 node[above]	{\tiny b}	(q1);
		\end{tikzpicture}
	\end{center}
	Costruiamo un sistema di equazioni per ogni stato
	$$
	\begin{cases}
		X &= a X_1 + b Y \\
		Y &= \epsilon
	\end{cases}
	$$
	e sostituendo 
	$$
	\begin{cases}
		X &= a X + b  \\
		Y &= \epsilon
	\end{cases}
	$$
	se ci troviamo in uno stato $X = A X + B$, questo corrisponde alla regex $A^* B$.
	Questo è banalmente $a^*b$.
\end{tcolorbox}

\begin{tcolorbox}
	Un altro esempio più complesso
	\begin{center}
		\begin{tikzpicture}
			\node[state, initial] 	(q0) {$q_0$};
			\node[state] 		(q1) [right=of q0] {$q_1$};
			\node[state, accepting]	(q2) [right=of q1] {$q_2$};

			\path[->] (q0)	edge[loop below] node[below]	{\tiny b}	()
					edge[bend left]	 node[above]	{\tiny a}	(q1)
				  (q1)	edge[bend left]  node[below] 	{\tiny b}	(q0)
				  	edge[bend left]  node[above]	{\tiny a}	(q2)
				  (q2)	edge[loop below] node[below]	{\tiny a}	()
					edge[bend left]	 node[below]	{\tiny b}	(q1);
		\end{tikzpicture}
	\end{center}
	Costruiamo un sistema di equazioni per ogni stato
	$$
	\begin{cases}
		X_0 &= a X_1 + b X_0 \\
		X_1 &= a X_2 + b X_0 \\
		X_2 &= a X_2 + b X_1 + \epsilon
	\end{cases}
	$$
	$\epsilon$ perché $X_2$ è finale.
	Sostitendo $X_1$ abbiamo
	\begin{align*}
		X_0 &= aa X_2 + ab X_0 + bX_0 \\
		    &= aaX_2 + (ab + b)X_0 
	\end{align*}
	e 
	\begin{align*}
		X_2 &= aX_2 + baX_2 + bbX_0 + \epsilon \\
		    &= (a + ba) X_2 + bbX_0 + \epsilon \\
		    &= (a + ba)^* + bb X_0 + \epsilon
	\end{align*}
	che corrisponde alla regex $(a + ba)^* (bb X_0 + \epsilon)$, prendendo come $A = (a + ba)$ e $B = (bb X_0 + \epsilon)$.
	E sostituendo ancora
	\begin{align*}
		X_0 &= aa X_2 + (ab + b) X_= \\
		    &= aa (a + ba)^*(bb X_0 + \epsilon) + (ab + b) X_0 \\ 
		    &= (aa (a + ba)^* bb + ab + b) X_0 + aa (a + ba)^*
	\end{align*}
	che quindi è $(aa (a + ba)^* bb + ab + b)^* + aa (a + ba)^*$.
\end{tcolorbox}

% ci torneremo
Il complemento di linguaggio riconosciuto da un automa deterministico è banalmente il complemento dell'insieme dei finali.

\end{document}

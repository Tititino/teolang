\documentclass[12pt]{article}
\usepackage{amsmath}
\usepackage{amsfonts}
\usepackage{amssymb}
\usepackage{amsthm}
\usepackage[italian]{babel}
\usepackage{bytefield}
\usepackage{cancel}
\usepackage{embedall}\embedfile{\jobname.tex}
\usepackage[bookmarks]{hyperref}
\usepackage{listings}
\usepackage[scr=rsfs]{mathalpha}
\usepackage{siunitx}
\usepackage{tabularray}
\usepackage{tcolorbox}
\usepackage{tikz}\usetikzlibrary{automata}
\usepackage{todonotes}
\usepackage{xcolor}

\newtheorem{teorema}{Teorema}
\newtheorem{corollario}{Corollario}
\newtheorem{proposizione}{Proposizione}
\newtheorem{proprietà}{Proprietà}
\newtheorem{lemma}{Lemma}
\newtheorem{fatto}{Fatto}
\newtheorem{definizione}{Definizione}
\newtheorem{nota}{Nota}

\renewcommand\qedsymbol{$\blacksquare$}

\usepackage{teolang}

\definecolor{codegray}{gray}{0.95}

\lstdefinestyle{mystyle}{
  numberstyle=\tiny,
  basicstyle=\footnotesize,
  breaklines=true,
  numbers=left,
  numbersep=5pt,
}
\lstset{style=mystyle}

\overfullrule=0.2cm
\begin{document}
\tableofcontents
\newpage
Nell'automa minimo tutti gli stati sono -- ovviamente -- distinguibili.

\section{Algoritmo di Brzozowski}
Studiamo la proprietà su cui è basata la correttezza dell'algoritmo di Brzozowski.

Sia
$$ M = \langle Q, \Sigma, \delta, q_0, F \rangle$$
un DFA, e ipotizziamo di aver eliminato ogni stato non raggiungibile.

Dato questo DFA costruiamo l'NFA 
$$ M_R = \langle Q, \Sigma, \delta^R, F, \{ q_0 \} \rangle $$ % finali e iniziali scambiati
per il reversal.
La funzione 
$$ \forall p \in Q, a \in \Sigma^* \mid \delta^R(p, a) = \{ q \mid \delta(q, a) = p \} $$
Applichiamo la costruzione dei sottoinsiemi $sub$ (avendo eliminato gli stati irraggiungibili).
$$ N = sub(M^R) = \langle Q'', \Sigma, \delta'', q_0'', F'' \rangle $$
Con
\begin{itemize}
	\item $Q'' \subseteq 2^Q $
	\item $\forall \alpha \in Q'', a \in \Sigma \delta''(\alpha, a) = \bigcup_{p \in \alpha} \delta^R (p, a) $
	\item $ q_0'' = F $
	\item $ F'' = \{ \alpha \in Q'' \mid q_0 \in \alpha \} $
\end{itemize}

\begin{lemma}
	$N$ è un DFA minimo per $L(M)^R$.
\end{lemma}
\begin{proof}
	Quello che vogliamo mostrare è che se $A, B \in Q''$ e non sono distinguibili, allora $A = B$.
	Visto che $A$ e $B$ sono due sottoinsiemi, mostriamo che $A \subseteq B$ e $B \subseteq A$.

	Sia $p \in A$, questo è anche uno stato di $M$, quindi per ipotesi è raggiungibile, quindi
	$$ \exists w \in \Sigma^* \mid \delta(q_0, w) = p $$
	e di conseguenza, seguendo il cammino al contrario
	$$ q_0 \in \delta^R (p, w^R) $$
	e quindi
	$$ q_0 \in \delta'' (A, w^R) $$
	e quindi $w^R$ è accettata a partire da $A$.

	Siccome $A$ e $B$ non sono distinguibili, allora $w^R$ è accettata anche a partire da $B$.
	Quindi
	$$ \exists p' \in B \mid q_0 \in \delta''(p', w^R) $$
	Quindi, visto che $\delta''$ è un cammino rovesciato, $\delta(q_0, w) = p'$.
	E visto che $M$ è deterministico $p = p'$ e $p \in B$.

	Quindi abbiamo che $A \subseteq B$, e si può procedere ugualmente per $B \subseteq A$, quindi $A = B$.
\end{proof}

Rivediamo i passi dell'algoritmo di Brzozowski:
\begin{center}
	\begin{tblr}{ colspec = { c c c c } }
		& automa & tipo & linguaggio \\
		0 & $ K $            & DFA & $L$ \\
		1 & $ K^R $          & NFA & $L^R$ \\
		2 & $ M = sub(K^R) $ & DFA & $L^R$ \\
		3 & $ M^R $          & NFA & $L$ \\
		4 & $ N = sub(M^R)$  & DFA & $L$ 
	\end{tblr}
\end{center}
E $N = sub(M^R)$ è minimo per il lemma.

\section{?}
Sia $L$ un linguaggio regolare, definitamo 
$$\frac{1}{2}L = \{ x \in \Sigma^* \mid \exists y |x| = |y| \wedge xy \in L \} $$
come il linguaggio delle prime metà del linguaggio $L$.
Questo è ancora regolare.

Definiamo prima la versione semplicifcata, 
$$ g(L) = \{ x \in \Sigma^* \mid xx \in L \} $$
In questo caso si può utilizzare il non determinismo per scommettere sullo stato $p$ a cui ci porterà leggere $x$.
Poi facciamo andare in parallelo $L$ da $q_0$ e $p$, e alla fine di $x$ si deve controllare che da $q_0$ siamo effettivamente arrivati in $p$ e che da $p$ siamo effettivamente arrivati in uno stato finale.
In questo modo ogni stato deve avere tre informazioni: lo stato $p$ scelto e la coppia degli stati della prima e della seconda parte.

Tornando al linguaggio della prima metà, questo è simile al precedente, ma senza la condizione che le due metà siano uguali, ma che solo le lunghezze siano uguali.

In questo caso si scommette sia sullo stato $p$ finale, che sul simbolo che troveremo nella seconda metà.

Quindi dato $\mathcal{A} = \langle Q, \Sigma, \delta, q_0, F \rangle$ definiamo 
$$ \mathcal{A}' = \langle Q', \Sigma, \delta', I', F' \rangle$$
tale che
\begin{align*}
	Q' &\subseteq Q \times Q \times Q \\
	I' &= \{ (q_0, p, p) \mid p \in Q \} \\
	F' &= \{ (p, p, q_F) \mid q_F \in F \} \\
	\delta'((q', p, q''), a) &= \{ (\delta(q', a), p \delta(q'', b)) \mid b \in \Sigma \} 
\end{align*}
Dove -- in $Q'$ -- la prima $Q$ rappresenta la simulazione della prima metà, la seconda lo stato intermedio previsto, e l'utlima la simulazione della seconda metà

\section{Rappresentazione matriciale}
Dato un automa
$$ \mathcal{A} = \langle Q, \Sigma, \delta, q_0, F \rangle $$
con $Q = \{q_0, q_1, \dots, q_{n - 1} \}$, definiamo per un elemento $x \in \Sigma$
$$ M_x[i, j] = 1 \Leftrightarrow q_j \in \delta(q_i, x) $$
Nel caso $\mathcal{A}$ sia un DFA vale che ogni riga contiene un singolo uno.

Per un DFA, possiamo definire $ M_\epsilon$ come la matrice identità, e, attraverso questa, definire induttivamente la matrice per una parola $w \in \Sigma^*$ come
$$ \forall x \in \Sigma, \forall w \in \Sigma^* \mid M_{xw} = M_x \cdot M_w $$

\begin{tcolorbox}
	Dato l'automa seguente
	\begin{center}
		\begin{tikzpicture}
			\node[state, initial, accepting] 	(q0) {$q_0$};
			\node[state] 				(q1) [right=of q0] {$q_1$};
			\node[state, accepting] 		(q2) [below=of q1] {$q_2$};

			\path[->] (q0)	edge[loop above] node[above]	{\tiny b}	()
					edge		 node[above]	{\tiny a}	(q1)
				  (q1)	edge[bend left]	 node[right]	{\tiny a,b}	(q2)
				  (q2)	edge[bend left]	 node[left]	{\tiny b}	(q1)
					edge		 node[below left]	{\tiny a}	(q0);
		\end{tikzpicture}
	\end{center}
	Questo è rappresentato da
	\begin{align*}
		M_a &= \begin{bmatrix} 0 & 1 & 0 \\ 0 & 0 & 1 \\ 1 & 0 & 0 \end{bmatrix} \\
		M_b &= \begin{bmatrix} 1 & 0 & 0 \\ 0 & 0 & 1 \\ 0 & 1 & 0 \end{bmatrix}
	\end{align*}
	E, ad esempio,
	$$ M_{aa} = M_a \cdot M_a = \begin{bmatrix} 0 & 1 & 0 \\ 0 & 0 & 1 \\ 1 & 0 & 0 \end{bmatrix} \cdot \begin{bmatrix} 0 & 1 & 0 \\ 0 & 0 & 1 \\ 1 & 0 & 0 \end{bmatrix} = \begin{bmatrix} 0 & 0 & 1 \\ 1 & 0 & 0 \\ 0 & 1 & 0 \end{bmatrix} $$
\end{tcolorbox}

Definiamo il vettore iniziale come
$$ \pi = \begin{pmatrix} 1 & 0 & 0 \end{pmatrix} $$
e il vettore di accettazione come
$$ \eta = \begin{pmatrix} 1 & 0 & 1 \end{pmatrix} $$
In questo modo una parola $w$ è accettata se
$$ \pi M_w \eta^T = 1 $$

Con questa rappresentazione si possono fare i tipici calcoli che si fanno sui grafi, ad esempio sia 
$$ M = \sum_{x \in \Sigma} M_x $$
allora $M[i, j]$ è il minimo numeri di simboli necessario per passare dallo stato $i$ allo stato $j$.	% o viceversa?

Mentre $M^k[i, j]$ rappresenta ???.

Ritornando all'esempio di sopra dell'automa per la metà, dato l'automa $\mathcal{A} = \langle Q, \Sigma, \delta, q_0, F\rangle$, con $|Q| = n$, definiamo
$$ \mathcal{A}' = \langle Q', \Sigma, \delta', q_0', F' \rangle $$
Con
\begin{align*}
	Q' &= \{ [q, C] \mid q \in Q, C \in \text{Mat}_{n \times n}(\mathbb{B}) \}  \\
	\delta'([q, C], a) &= [\delta(q, a), C \cdot M ] \\
	q_0' &= [q_0, I] 
\end{align*}
Dove $\text{Mat}_{n \times n}(\mathbb{B})$ è l'insieme delle matrici su valori booleani e $I$ è la matrice identità.
In questo caso 
$$ M = \bigvee_{x \in \Sigma} M_x $$
è la matrice in cui $M[i, j] = 1$ sse $\exists a \mid \delta(q_i, a) = q_j$.
Data questa matrice vale che data una parola $x$
$$ q_0' \overset{x}{\rightsquigarrow} \left [\delta(q_0, x), M^{|x|} \right ] $$
Dove $M^{|x|}$ rappresenta i cammini di lunghezza $|x|$.

\begin{tcolorbox}
	Utilizzando l'esempio di prima abbiamo che
	$$ q_0' \overset{aab}{\rightsquigarrow} [\delta(q_0, aab), M^3 ] $$
	% finisco
\end{tcolorbox}

Ora se mi trovo in $[p, M]$, per accettare controllo che in $M$ ci sia un uno corrispondente ad uno stato finale alla riga $p$, quindi
$$ M[q, f] = 1 \Leftrightarrow \exists y \mid |x| = |y| \Rightarrow \delta(q, y) = f $$

Nel caso di NFA una riga può avere più di un uno.
Con queste matrici se si utilizza l'and si ottiene la raggiungibilità dell'automa, mentre utilizzando il prodotto si ottiene il numero di cammini.

Esistono anche versioni di NFA con pesi sugli archi, un caso particolare sono gli NFA con probabilità associate agli archi.
Ad esempio
\begin{center}
	\begin{tikzpicture}
		\node[state, initial] 	(q0) {$q_0$};
		\node[state, accepting]	(q2) [right=of q0] {$q_1$};

		\path[->] (q0) edge[loop above] 	node[above]	{\tiny b, $\frac{1}{2}$} ()
			       edge[bend left]		node[above]	{\tiny b, $\frac{1}{2}$} (q1)
			       edge[loop below]		node[below]	{\tiny a, 1}		 ()
			  (q1) edge[loop above]		node[above]	{\tiny b, 1}		 ()
			       edge[loop below]		node[below]	{\tiny a, $\frac{1}{2}$} ()
			       edge[bend left]		node[below]	{\tiny a, $\frac{1}{2}$} (q0);
	\end{tikzpicture}
\end{center}
Anche allo stato iniziale può essere associata una probabilità.

In questo automa le varie matrici sono definite
\begin{align*}
	\pi  &= \begin{pmatrix} 1 & 0 \end{pmatrix} \\
	\eta &= \begin{pmatrix} 0 & 1 \end{pmatrix} \\
	M_a  &= \begin{bmatrix} 1 & 0 \\ \frac{1}{2} & \frac{1}{2} \end{bmatrix} \\
	M_a  &= \begin{bmatrix} \frac{1}{2} & \frac{1}{2} \\ 0 & 1 \end{bmatrix} \\
\end{align*}
Ovviamente vale che nelle matrici $M$ la somma delle righe deve essere 1.

Questo tipo di automa associa ad una stringha una probabilità che questa venga accettata.
\begin{tcolorbox}
	Dato l'automa di sopra, la stringa $w = bb$ ha i seguenti percorsi accettanti
	\begin{center}
		\begin{tblr}{ colspec = {cccccc}, cells = { mode = math } }
			q_0 &\overset{b, \frac{1}{2}}{\rightarrow} & q_0 &\overset{b, \frac{1}{2}}{\rightarrow} &  q_1 &= \frac{1}{4} \\
			q_0 &\overset{b, \frac{1}{2}}{\rightarrow} & q_1 &\overset{b, 1}{\rightarrow} & q_1 &= \frac{1}{2} \\
		\end{tblr}
	\end{center}
	Quindi una probabilità di $\frac{1}{4} \cdot \frac{1}{2} = \frac{3}{4}$ di essere accettata.
\end{tcolorbox}
Di solito si definisce una soglia $\lambda$ e
$$ L_\lambda = \{ w \in \Sigma^* \mid \mathbb{P}(\delta(q_0, w) \in F) > \lambda \} $$
Diciamo che una soglia è non isolata se riusciamo ad avvicinarci a piacere a $\lambda$
\begin{teorema}[Teorema di Rabin]
	Se la soglia è isolata, il linguaggio è regolare.
\end{teorema}
\end{document}

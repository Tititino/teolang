\documentclass[12pt]{article}
\usepackage{amsmath}
\usepackage{amsfonts}
\usepackage{amssymb}
\usepackage[italian]{babel}
\usepackage{bytefield}
\usepackage{cancel}
\usepackage{embedall}\embedfile{\jobname.tex}
\usepackage[bookmarks]{hyperref}
\usepackage{listings}
\usepackage[scr=rsfs]{mathalpha}
\usepackage{siunitx}
\usepackage{tabularray}
\usepackage{tcolorbox}
\usepackage{tikz}\usetikzlibrary{automata}
\usepackage{todonotes}
\usepackage{xcolor}

\usepackage{teolang}

\definecolor{codegray}{gray}{0.95}

\lstdefinestyle{mystyle}{
  numberstyle=\tiny,
  basicstyle=\footnotesize,
  breaklines=true,
  numbers=left,
  numbersep=5pt,
}
\lstset{style=mystyle}

\overfullrule=0.2cm
\begin{document}
\tableofcontents
\newpage
La gerarchia di Chomksy vale per alfabeti di almeno due lettere.

\section{Parola vuota ed $\epsilon$-produzioni}
Supponendo di avere una grammatica $G$ di tipo 1, supponiamo di voler aggiungere la parola vuota al linguaggio generato.
Se si aggiunge banalmente la regola per la parola vuota
$$ S \prd \epsilon $$
non ci sono garanzie che noi non abbiamo aggiunto anche altre stringhe al linguaggio, ad esempio nel caso
$$ \alpha \der \dots S \dots $$
inoltre perdiamo la proprietà della monotonicità delle forme sentenziali.

Quindi bisogna essere leggermente più delicati, dato $L = L(G)$, voglio costruire $G^\prime$ tale che $L^\prime = L(G^\prime) = L(G) \cup \{ \epsilon \}$.
Per gare ciò definisco $G^\prime = \grammar{V \cup \{S^\prime\}}{\Sigma}{P \cup \{ S^\prime \prd \epsilon \mid S\}}{S^\prime}$.

Dal simbolo iniziale posso utilizzare $S \prd \epsilon$ purché $S$ non appaia su lato destro di nessuna produzione.
Questo ovviamente vale anche per grammatiche di tipo 2 e 3.

Data una grammatica di tipo 2 che utilizza $\epsilon$-produzioni 
$$ A \prd \epsilon $$
è possibile trovare una grammatica di tipo 2 equivalente che genera lo stesso linguaggio, meno la parola vuota, che non utilizza $\epsilon$-produzioni.

Anche nelle gramamtiche di tipo 3 possiamo ammettere $\epsilon$-produzioni
$$ A \prd \epsilon $$

\section{Automa a stati finiti}
Un automa è un dispositivo che ha
\begin{itemize}
	\item un nastro diviso in celle che contiene l'input, ogni cella contiene un simbolo dell'alfabeto
		Questo non può essere modificato.
	\item una testina che passa il nastro da sinistra a destra, per questo vengono chiamati anche automi $one-way$.
		Quando questa finisce il nastro deve poter rispondere se la parola appartiene o no al linguaggio.
		La testina contiene una macchina che ha un insieme finito di stati.
\end{itemize}
Un automa è quindi una quintupla $\mathscr{A} = \langle Q, \Sigma, \delta, q_0, F \rangle$, dove
\begin{itemize}
	\item $Q$ è l'insieme degli stati
	\item $\Sigma$ è l'alfabeto finito di input
	\item $\delta$ è il programma dell'automa, o funzione di transizione
	\item $q_0 \in Q$ è lo stato iniziale
	\item $F \subseteq Q$ è l'insieme degli stati finali
\end{itemize}
Ad ogni passo l'automa legge un simbolo, ed in base allo stato attuale e il simbolo in input sceglie il prossimo stato con $\delta$
$$ \delta : Q \times \Sigma \rightarrow Q $$
Definiamo 
$$ \delta^* : Q \times \Sigma^* \rightarrow Q $$
per induzione sulla lunghezza delle stringhe in input:
\begin{itemize}
	\item $\forall q \in Q \mid \delta^*(q, \epsilon) = q $
	\item $\forall q \in Q, x \in \Sigma^*, a \in \Sigma \mid \delta^*(q, x a) = \delta(\delta^*(q, x), a)$
\end{itemize}
Visto che $\delta^*$ è effettivamente una estensione di $\delta$, chiameremo $\delta^*$ $\delta$.

Ora possiamo definire il linguaggio riconosciuto dall'automa $\mathscr{A}$ come
$$ L(\mathscr{A}) = \{ w \in \Sigma^* \mid \delta(q_0, w) \in F \} $$

\begin{tcolorbox}
	Ad esempio $\mathscr{A} = \langle Q = \{ q_0, q_1 \}, \Sigma = \{ a, b \}, \delta, q_0, F = \{ q_1 \} \rangle$, con $\delta$ definita come
	\begin{center}
	%	\begin{tblr}{c| c c }
	%		\delta & a & b \\
	%		\hline
	%		q_0    & q_0 & q_1 \\
	%		q_1    & q_1 & q_0 \\
	%	\end{tblr}
	\end{center}
	Per $aabb$ abbiamo
	$$ q_0 \overset{a}{\rightarrow} q_0 \overset{a}{\rightarrow} q_0 \overset{b}{\rightarrow} q_1 \overset{b}{\rightarrow} q_0 $$
	quindi la stringa non è accettata perché $q_0 \not \in F$.
	Questo è linguaggoi di tutte le stringhe con un numero dispari di $b$.
\end{tcolorbox}
Invece che rappresentare $\delta$ come una tabella, di solito è comodo rappresentarla come un diagramma di transizione:
\begin{itemize}
	\item Lo stato iniziale è rappresentato da una freccia entrante
	\item gli stati finali da un doppio cerchio.
\end{itemize}
Quindi una derivazione non è altro che un cammino in questo grafo, e la parola è accettata se si finisce in uno stato finale.

\begin{tcolorbox}
	Dato $\Sigma = \{ a, b \}$, e il linguaggio 
	$$ \mathscr{L} = \{ x \in \Sigma^* \mid P(x) \}$$
	con $P$ definita come la proprietà per cui tra ogni coppia di $b$ successive in $x$ c'è un numero di $a$ pari.
	\begin{center}
		\begin{tikzpicture}
			\node[state, initial, accepting] 	(q_0) {$q_0$};
			\node[state, accepting] 		(q_1) [right=of q_0] {$q_1$};
			\node[state, accepting]			(q_2) [right=of q_1] {$q_2$};
			\node[state]				(q_3) [right=of q_2] {$q_3$};

			\path[->] (q_0)	edge[loop above]	node[above]	{a}	()
					edge			node[above]	{b}	(q_1)
				  (q_1)	edge[loop above]	node[above]	{b}	()
					edge			node[above]	{a}	(q_2)
				  (q_2)	edge			node[above]	{}	(q_1)
				  	edge			node[above]	{b}	(q_3)
				  (q_3)	edge[loop above]	node[above]	{a, b}	();
		\end{tikzpicture}
	\end{center}
	% 
	%  init(final(q_0)) --b--> final(q_1) <--a--> final(q_2) --b--> trap(q_3)
	%         \_a_/             \_b_/                                \_a,b_/
	Visto che 
	$$ \delta : Q \times \Sigma \rightarrow Q $$
	nello stato $q_2$ bisogna aggiungere una freccia ad uno stato da cui non si può uscire, infatti se siamo in quello stato vuol dire che abbiamo trovato un numero dispari di $a$ e una $b$, quindi la parola non è valida.
	Questo tipo di stati è detto stato trappola, e di solito vengono lasciati impliciti.
\end{tcolorbox}

\begin{tcolorbox}
	Dato $\Sigma = \{ a, b \}$, e il linguaggio 
	$$ \mathscr{L} = \{ x \in \Sigma^* \mid \text{Il terzo simbolo di $x$ è una $a$}\}$$
	% init(q_0) --a,b--> q_1 --a,b--> q_2 --a--> final(q_3)
	%                                             \_a,b_/
	\begin{center}
		\begin{tikzpicture}
			\node[state, initial] 	(q_0) {$q_0$};
			\node[state]		(q_1) [right=of q_0] {$q_1$};
			\node[state]		(q_2) [right=of q_1] {$q_2$};
			\node[state, accepting]	(q_3) [right=of q_2] {$q_3$};

			\path[->] (q_0)	edge			node[above]	{a, b}	(q_1)
				  (q_1)	edge			node[above]	{a,b}	(q_2)
				  (q_2)	edge			node[above]	{a}	(q_3)
				  (q_3)	edge[loop above]	node[above]	{a, b}	();
		\end{tikzpicture}
	\end{center}
\end{tcolorbox}
\begin{tcolorbox}
	Dato $\Sigma = \{ a, b \}$, e il linguaggio 
	$$ \mathscr{L} = \{ x \in \Sigma^* \mid \text{ Il terzultimo simbolo di $x$ è una $a$} \} $$
	Teniamo uno stato per ogni tripla di $a, b$, quindi $2^3 = 8$ stati
	% graph automa {
	% aaa -b-> aab
	% aaa -a-> aaa
	% aab -a-> aba
	% aab -b-> abb
	% aba -a-> baa
	% aba -b-> bab
	% bab -a-> aba
	% bab -b-> abb
	% abb -a-> bba
	% abb -b-> bbb
	% baa -a-> aaa
	% baa -b-> aab
	% bbb -b-> bbb
	% bbb -a-> bba
	% bba -a-> baa
	% bba -b-> bab
	% fin(aaa)
	% fin(aab)
	% fin(aba)
	% fin(abb)
	% init(bbb)
	% }
\end{tcolorbox}

Una operazione che può interessare per le stringhe è costruire il linguaggio che riconosce le stringhe roversciate di uno.
Dato un linguaggio $\mathscr{L}$, il linguaggio che riconosce le sue stringhe inverse è detto il reversal di $\mathscr{L}$.
Nel caso dei due linguaggi precedenti se prendiamo l'automa del primo e lo invertiamo ottieniamo un automa valido (non deterministico) per il secondo.

\subsection{Automi non deterministici}
Definiamo un automa non deterministico come la quintupla $\mathscr{A} = \langle Q, \Sigma, \delta, q_0, F \rangle$, ma ora $\delta$ è definita come
$$ \delta : Q \times \Sigma \rightarrow 2^Q $$
quindi $\delta$ non ritorna più un singolo insieme ma un insieme di possibili stati.

Una stringa viene accettata se esiste almeno un cammino che mi porta in uno stato finale.
Estensiamo ancora $\delta$ alle stringhe 
$$ \delta^* : Q \times \Sigma \rightarrow 2^Q $$
è definita come
\begin{itemize}
	\item $\forall q \in Q \mid \delta^*(q, \epsilon) = \{ q \} $
	\item $\forall q \in Q, x \in \Sigma^*, a \in \Sigma \mid \delta^*(q, xa) = \bigcup_{p \in \delta^*(q, x)} \delta(p, a) $
\end{itemize}

Come prima chiameremo $\delta^*$ semplicemente $\delta$.

Ora il linguaggio accettato dall'automa $\mathscr{A}$ non deterministico è $L(\mathscr{A}) = \{ w \in \Sigma^* \mid \delta(q_0, w) \cap F \neq \varnothing \}$.

\begin{tcolorbox}
	Dato l'automa non deterministico
	\begin{center}
		\begin{tikzpicture}
			\node[state, initial] 	(r_0) {$r_0$};
			\node[state] 	      	(r_1) [right=of r_0] {$r_1$};
			\node[state] 	      	(r_2) [right=of r_1] {$r_2$};
			\node[state, accepting]	(r_3) [right=of r_2] {$r_3$};

			\path[->] (r_0)	edge	[loop above]	node[above]	{a, b}	()
					edge			node[above]	{a}	(r_1)
				  (r_1)	edge			node[above]	{a, b}	(r_2)
				  (r_2)	edge			node[above]	{a, b}	(r_3);
		\end{tikzpicture}
	\end{center}
	con la stringha $ababb$ mostriamo tutte le possibili strade che si possono prendere
	\begin{center}
		\begin{tikzpicture}
			\node {$r_0$}
				child { node {$r_0$}
					child {	node {$r_0$}
						child {	
							node {$r_0$} 
							child {
								node {$r_0$}
								child {
									node {$\cancel{r_0}$}
									edge from parent node[left] {b}
								}
								edge from parent node[left] {b}
							}
							edge from parent node[left] {a}
						}
						child { 
							node {$r_1$} 
							child {
								node {$r_2$}
								child {
									node {$r_3$}
									edge from parent node[right] {b}
								}
								edge from parent node[right] {b}
							}
							edge from parent node[right] {a}
						}
						edge from parent node[left] {b}
					}
					edge from parent node[left] {a}
				}
				child {	
					node {$r_1$}
					child {
						node {$r_2$}
						child {
							node {$\cancel{r_3}$}
							edge from parent node[right] {a}
						}
						edge from parent node[right] {b}
					}
					edge from parent node[right] {a}
				};
		\end{tikzpicture}
	\end{center}
	è necessario che esista almeno una strada dell'albero che risponda sì per accettare.
	Questa struttura è chiamato albero di computazione, e un cammino descrive una computazione.

	Bisogna supporre che l'automa riesca a sceglie la strada esatta.
	Alternativamente si può pensare che l'automa possa visitare tutte le strade in parallelo, quindi possiamo trasformare l'albero in
	\begin{center}
		\begin{tikzpicture}
			\node[state, initial] 	(r0) {$\{ r_0 \}$};
			\node[state] 	      	(r0r1)	[right=of r0] {$\{ r_0, r_1 \}$};
			\node[state] 	      	(r0r2)	[right=of r0r1] {$\{ r_0, r_2 \}$};
			\node[state] 	      	(r0r1r3)[below right=of r0r2] {$\{ r_0, r_1, r_3 \}$};
			\node[state, accepting]	(r0r3) 	[right=of r0r2] {$\{ r_0, r_3 \}$};

			\path[->] (r0)		edge	node[above]	{a}	(r0r1)
				  (r0r1) 	edge	node[above]	{b}	(r0r2)
				  (r0r2)	edge	node[above]	{a}	(r0r1r3)
				  		edge 	node[above]	{b}	(r0r3)
				  (r0r1r3)	edge	node[above]	{a}	(r0r2);
		\end{tikzpicture}
	\end{center}
	Questo 
\end{tcolorbox}
Ogni autma non deterministico (NFA) con $n$ stati può sempre essere trasformato in un automa deterministico con al più $2^n$ stati.
Nel caso degli automi a pila invece il modello non deterministico è più potente.
Qui invece tra DFA e NFA non cambia la potenza computazionale, ma la semplicità descrittiva.
\end{document}

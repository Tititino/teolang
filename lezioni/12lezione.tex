\documentclass[12pt]{article}
\usepackage{amsmath}
\usepackage{amsfonts}
\usepackage{amssymb}
\usepackage{amsthm}
\usepackage[italian]{babel}
\usepackage{bytefield}
\usepackage{cancel}
\usepackage{caption}
\usepackage{embedall}\embedfile{\jobname.tex}
\usepackage{float}
\usepackage[bookmarks]{hyperref}
\usepackage{listings}
\usepackage[scr=rsfs]{mathalpha}
\usepackage{siunitx}
\usepackage{subcaption}
\usepackage{tabularray}
\usepackage[most]{tcolorbox}
\usepackage{tikz}\usetikzlibrary{automata, chains, scopes, decorations.text, patterns, decorations.pathmorphing, positioning, decorations.pathreplacing, calligraphy}
\usepackage{todonotes}
\usepackage{xcolor}

\newtheorem{teorema}{Teorema}
\newtheorem{corollario}{Corollario}
\newtheorem{proposizione}{Proposizione}
\newtheorem{proprietà}{Proprietà}
\newtheorem{lemma}{Lemma}
\newtheorem{fatto}{Fatto}
\newtheorem{definizione}{Definizione}
\newtheorem{nota}{Nota}

\renewcommand\qedsymbol{$\blacksquare$}

\usepackage{teolang}

\definecolor{codegray}{gray}{0.95}

\lstdefinestyle{mystyle}{
  numberstyle=\tiny,
  basicstyle=\footnotesize,
  breaklines=true,
  numbers=left,
  numbersep=5pt,
}
\lstset{style=mystyle}

\overfullrule=0.2cm
\begin{document}
\tableofcontents
\newpage
\section{Automi a pila}
Gli automi a pila sono automi che oltre ad avere un controllo a stati finiti hanno una memoria arbitrariamente grande, organizzata come una pila, quindi non si può accedere liberamente alle informazioni in questa.
Questo modello è one-way sul nastro di input (la versione two-way è più potente).

Inizieremo subito con vedere il modello non deterministico.

Un automa a pila è una tupla
$$ M = ( Q, \Sigma, \Gamma, \delta, q_0, Z_0, F ) $$
\begin{itemize}
	\item $\Gamma$ l'alfabeto della pila o alfabeto di lavoro
	\item $\delta$ funzione di transizione
	\item $q_0 \in Q$, lo stato iniziale dell'automa
	\item $Z_0 \in \Gamma$, lo stato iniziale della pila
	\item $F$ un insieme di stati finali
\end{itemize}
La funzione di transizione dipenderà quindi da tre cose: dallo stato corrente, dal simbolo dell'input corrente e dallo stato corrente della pila
$$ \delta : Q \times (\Sigma \cup \{\epsilon\}) \times \Gamma \rightarrow \text{PF}(Q \times \Gamma^*) $$
Quindi la funzione di transizione cambia lo stato dell'automa e rimpiazza il simbolo in cima alla pila con una stringa di stati della pila.
$\text{PF}(-)$ sta per le parti finite, infatti se utilizzassimo $2^{Q \times \Gamma^*}$ potremmo avere programmi infiniti, visto che $\Gamma^*$ è un insieme infinito.
$\Sigma \cup \epsilon$ perché sono contemplate mosse in base allo stato dell'automa che modificano la pila senza leggere un simbolo in input.
% fig 12.1
\begin{center}
	\begin{tikzpicture}[ SQUIGGLY/.style={->
			  		     , decorate
			                     , decoration={snake,amplitude=.4mm,segment length=2mm,post length=1mm}},
			   ]
 		\begin{scope}[local bounding box=wordScope, start chain=word, node distance=0pt]
    			\node [draw, minimum width=40pt, minimum height=20pt, on chain=word] {$\dots$};
    			\node [draw, minimum height=20pt, on chain=word] {a};
    			\node [draw, minimum width=40pt, minimum height=20pt, on chain=word] {$\dots$};
 		\end{scope}

		\node[draw, minimum width=20pt, minimum height=20pt] (state) [below=of word-2]	{$q$};

		\begin{scope}[local bounding box=stackScope, start chain=stack going below, node distance=0pt]
			\node [draw, minimum width=20pt, on chain=stack] [right=of state, xshift=1cm] {A};
    			\node [draw, minimum width=20pt, minimum height=40pt, on chain=stack] {$\vdots$};
 		\end{scope}

		\draw[SQUIGGLY] (state.north) to (word-2.south);
		\draw[SQUIGGLY] (state.east) to (stack-1.west);
	\end{tikzpicture}
\end{center}
\begin{nota}
	Per convenzione la stringa di stati viene messa sulla pila di destra a sinistra, quindi il simbolo più a sinistra sarà in cima alla pila.
\end{nota}
Visto che ammettiamo $\epsilon$-mosse il modello di sopra potrebbe non esaurire tutte le possibilità.
Supponiamo che la funzione $\delta$ sia definita
$$\delta(q, a, A) = \{(q_1, \epsilon), (q_2, BCC)\} $$
\begin{figure}[H]
	\centering
	\begin{subfigure}{0.4\textwidth}
		\centering
		\begin{tikzpicture}[ SQUIGGLY/.style={->
			, decorate
			, decoration={snake,amplitude=.4mm,segment length=2mm,post length=1mm}},
			]
			\begin{scope}[local bounding box=wordScope, start chain=word, node distance=0pt]
				\node [draw, minimum width=40pt, minimum height=20pt, on chain=word] {$\dots$};
				\node [draw, minimum height=20pt, on chain=word] {a};
				\node [draw, minimum height=20pt, on chain=word] {?};
				\node [draw, minimum width=40pt, minimum height=20pt, on chain=word] {$\dots$};
			\end{scope}

			\node[draw, minimum width=20pt, minimum height=20pt] (state) [below=of word-3]	{$q_1$};

			\begin{scope}[local bounding box=stackScope, start chain=stack going below, node distance=0pt]
				\node [draw, minimum width=20pt, minimum height=40pt, on chain=stack] [right=of state, xshift=1cm] {$\vdots$};
			\end{scope}

			\draw[SQUIGGLY] (state.north) to (word-3.south);
			\draw[SQUIGGLY] (state.east) to (stack-1.west);
		\end{tikzpicture}
		\caption{Lo stato per $(q_1, \epsilon)$}
	\end{subfigure}
	\begin{subfigure}{0.4\textwidth}
		\centering
		\begin{tikzpicture}[ SQUIGGLY/.style={->
			, decorate
			, decoration={snake,amplitude=.4mm,segment length=2mm,post length=1mm}},
			]
			\begin{scope}[local bounding box=wordScope, start chain=word, node distance=0pt]
				\node [draw, minimum width=40pt, minimum height=20pt, on chain=word] {$\dots$};
				\node [draw, minimum height=20pt, on chain=word] {a};
				\node [draw, minimum height=20pt, on chain=word] {?};
				\node [draw, minimum width=40pt, minimum height=20pt, on chain=word] {$\dots$};
			\end{scope}

			\node[draw, minimum width=20pt, minimum height=20pt] (state) [below=of word-3]	{$q_2$};

			\begin{scope}[local bounding box=stackScope, start chain=stack going below, node distance=0pt]
				\node [draw, minimum width=20pt, on chain=stack] [right=of state, xshift=1cm] {B};
				\node [draw, minimum width=20pt, on chain=stack] {C};
				\node [draw, minimum width=20pt, on chain=stack] {C};
				\node [draw, minimum width=20pt, minimum height=40pt, on chain=stack] {$\vdots$};
			\end{scope}

			\draw[SQUIGGLY] (state.north) to (word-3.south);
			\draw[SQUIGGLY] (state.east) to (stack-1.west);
		\end{tikzpicture}
		\caption{Lo stato per $(q_2, BCC)$}
	\end{subfigure}
\end{figure}
Inoltre potremmo anche avere $\epsilon$-mosse, ad esempio
$$ \delta(q, \epsilon, A) = \{(r, B)\} $$
\begin{center}
	\begin{tikzpicture}[ SQUIGGLY/.style={->
			  		     , decorate
			                     , decoration={snake,amplitude=.4mm,segment length=2mm,post length=1mm}},
			   ]
 		\begin{scope}[local bounding box=wordScope, start chain=word, node distance=0pt]
    			\node [draw, minimum width=40pt, minimum height=20pt, on chain=word] {$\dots$};
    			\node [draw, minimum height=20pt, on chain=word] {a};
    			\node [draw, minimum width=40pt, minimum height=20pt, on chain=word] {$\dots$};
 		\end{scope}

		\node[draw, minimum width=20pt, minimum height=20pt] (state) [below=of word-2]	{$r$};

		\begin{scope}[local bounding box=stackScope, start chain=stack going below, node distance=0pt]
			\node [draw, minimum width=20pt, on chain=stack] [right=of state, xshift=1cm] {B};
    			\node [draw, minimum width=20pt, minimum height=40pt, on chain=stack] {$\vdots$};
 		\end{scope}

		\draw[SQUIGGLY] (state.north) to (word-2.south);
		\draw[SQUIGGLY] (state.east) to (stack-1.west);
	\end{tikzpicture}
\end{center}


\subsection{Definizioni}
Ci riferiremo agli automi a pila come PDA (Push Down Automaton) e assumeremo che siano sempre nondeterministici, a meno che diversamente specificato.

Chiameremo lo stato complessivo dell'automa a pila la sua \textbf{configurazione}, questa verrà rappresentata compattamente come una tupla dello stato, la porzione di input ancora da leggere da quello corrente, e il contenuto della pila.
Quindi 
% fig 12.3
è rappresentato dalla configurazione
$$ (q, ay, A\alpha) $$
con $q \in Q, a \in \Sigma \cup \{\epsilon\}, y \in \Sigma^*, A \in \Gamma, \alpha \in \Gamma^*$.

Una mossa, indica che da una configurazione $q$ posso passare ad un'altra $p$, scritto
$$ q \vdash p$$
Ad esempio sopra abbiamo che
\begin{align*}
	(q, ay, A\alpha) &\vdash (q_1, y, \alpha) \\
	(q, ay, A\alpha) &\vdash (q_2, y, BCC\alpha) \\
	(q, ay, A\alpha) &\vdash (r, ay, B\alpha)
\end{align*}
Più rigorosamente, sia $(q, ay, Z\alpha)$ la configurazione corrente, con $Z \in \Gamma$ e $M$ l'automa a pila, diciamo che
$$ (q, ay, Z\alpha) \underset{M}{\vdash} (p, y, \beta\alpha) $$
sse $(p, \beta) \in \delta(q, a, Z)$ dove $q, p \in Q$, $y \in \Sigma^*$, $a \in \Sigma \cup \{\epsilon\}$, $Z \in \Gamma$ e $\alpha, \beta \in \Gamma^*$.
Se l'automa è ovvio dal contensto possiamo ometterlo da $\underset{M}{\vdash}$ e scrivere solo $\vdash$.

Da una configurazione $C'$ arrivo ad una configurazione $C''$ in un certo numero di mosse -- scritto 
$$ C' \underset{M}{\overset{*}{\vdash}} C'' $$
sse esistono $C_0, \dots, C_k$ con $C_0 = C'$ e $C_k = C''$ e $\forall i \in 1, \dots, k \; C_{i - 1} \underset{M}{\vdash} C_i$.

La configurazione iniziale di un automa su input $w \in \Sigma^*$ è 
$$ (q_0, w, Z_0) $$

Per accettare possiamo dare alcune diverse definizioni di configurazione accettante:
\begin{itemize}
	\item una volta finito l'input mi trovo in uno stato $q \in F$ e la pila può essere una stringa qualunque, questa è detta \textit{accettazione per stati finali}, ed indichiamo il linguaggio accettato per stati finali dall'automa a pila $M$ come
		$$ L(M) = \{ w \in \Sigma^* \mid (q_0, w, Z_0) \overset{*}{\vdash} (q, \epsilon, \gamma), q \in F, \gamma \in \Gamma^* \} $$
		\begin{nota}
			Siccome sono accettate le $\epsilon$ mosse può esserci il caso in arriviamo alla fine dell'input con uno stato non finale, e si può fare una $\epsilon$-mossa ed arrivare ad uno stato finale.
		\end{nota}
	\item è ragionevole pensare che tutto quello che viene messo sulla pila debba anche essere tolto, questa è detta \textit{accettazione per pila vuota} per cui si deve arrivare alla fine dell'input ed aver svuotato l'intera pila, ingorando lo stato.
		Il linguaggio accettato per pila vuota dall'automa $M$ lo indichiamo come
		$$ N(M) = \{ w \in \Sigma^* \mid (q_0, w, Z_0) \overset{*}{\vdash} (q, \epsilon, \epsilon), q \in Q \} $$
		In questo caso ovviamente si può omettere $F$ dalla definizione dell'automa.
	\item si può pensare di richiedere entrambe le precedenti, come vedremo più avanti queste tre nozioni sono equivalenti (nel caso nondeterministico)
\end{itemize}
\begin{nota}
	Questa cosa la vedremo meglio, ma visto che la pila è la struttura fondamentale per la ricorsione, i linguaggi CF sono i linguaggi regolari a cui è stata aggiunta la ricorsione.
\end{nota}

\begin{tcolorbox}
	Definiamo il linguaggio
	$$ \mathcal{L} = \{ a^n b^n \mid n \geq 1 \} $$
	possiamo usare la pila per contare il numero di $a$.
	\begin{align*}
		\delta(q_0, a, Z_0) &= \{(q_0, A)\} \\
		\delta(q_0, a, A)   &= \{(q_0, AA) \} \\
		\delta(q_0, b, A)   &= \{ (q_1, \epsilon) \} \\
		\delta(q_1, b, A)   &= \{ (q_1, \epsilon) \} 
	\end{align*}
	E vale che data questa $\delta$
	$$ \mathcal{L} = N(M) $$
	Questo è un caso particolare di automa a pila in cui utilizziamo in simbolo solo (cioè $A$, oltre a $Z_0$) è detto \textit{automa a contatore}.

	Definiamo alternativamente
	\begin{align*}
		\delta(q_0, a, Z_0) &= \{(q_0, AZ_0)\} \\
		\delta(q_0, a, A)   &= \{(q_0, AA)\} \\
		\delta(q_0, b, A)   &= \{(q_1, \epsilon)\} \\
		\delta(q_1, b, A)   &= \{(q_1, \epsilon)\} \\
		\delta(q_1, \epsilon, Z_0) &= \{(q_F, \epsilon)\}
	\end{align*}
	con $F = \{q_F\}$, e vale che con questa $\delta$
	$$ \mathcal{L} = L(M) $$
\end{tcolorbox}

\begin{tcolorbox}[breakable]
	Vediamo ora il caso di sopra, ma in cui
	$$ \mathcal{L} = \{ a^n b^n \mid n \geq 0 \} $$
	possiamo usare la pila per contare il numero di $a$.
	\begin{align*}
		\delta(q_0, \epsilon, Z_0) &= \{(q_0, \epsilon)\} \\
		\delta(q_0, a, Z_0) &= \{(q_0, A)\} \\
		\delta(q_0, a, A)   &= \{(q_0, AA) \} \\
		\delta(q_0, b, A)   &= \{ (q_1, \epsilon) \} \\
		\delta(q_1, b, A)   &= \{ (q_1, \epsilon) \} 
	\end{align*}
	E vale che data questa $\delta$ in cui si può direttamente accettare dallo stato $q_0$
	$$ \mathcal{L} = N(M) $$
	L'introduzione della prima regola è problematica, perché con input non vuoto permette di svuotare la pila da $Z_0$, bloccando la continuazione dell'automa, quindi abbiamo introdotto il nondeterminismo tra le due regole $\delta(q_0, \epsilon, Z_0)$ e $\delta(q_0, a, Z_0)$.

	Definiamo similmente a sopra
	\begin{align*}
		\delta(q_0, \epsilon, Z_0) &= \{(q_F, \epsilon)\} \\
		\delta(q_0, a, Z_0) &= \{(q_0, AZ_0)\} \\
		\delta(q_0, a, A)   &= \{(q_0, AA)\} \\
		\delta(q_0, b, A)   &= \{(q_1, \epsilon)\} \\
		\delta(q_1, b, A)   &= \{(q_1, \epsilon)\} \\
		\delta(q_1, \epsilon, Z_0) &= \{(q_F, \epsilon)\}
	\end{align*}
	con $F = \{q_F\}$, e vale che con questa $\delta$
	$$ \mathcal{L} = L(M) $$
	sempre introducendo non determinismo.
	
	Questa versione, a differenza di quello di sopra per pila vuota, può anche essere fatta senza non determinismo infatti definiamo $q_I$ come nuovo stato iniziale che è anche finale, se la stringa è vuota possono direttamente accettare, mentre 
	$$ \delta(q_I, a, Z_0) = \{(q_0, AZ_0)\} $$
	ci riconduce all'automa di sopra.
\end{tcolorbox}
\begin{tcolorbox}
	Supponiamo invece di voler riconoscere
	$$ L = \{ w \in \{a, b\}^* \mid \#_a(w) = \#_b(w) \} $$
	bisogna contare sia le $a$ che le $b$: finché si incontrano $a$ si aggiungono alla pila, e appena si incontrano $b$ le si tolgono. 
	Se durante questa operazione si svuota la pila iniziamo a contare le $b$ e facciamo la operazione inversa.
	Queste due operazioni continuano ad alternarsi finché la stringa non è svuotato.
\end{tcolorbox}

Definiamo ora l'automa a pila deterministico.
Questo in ogni configurazione permette una singola scelta:
\begin{itemize}
	\item sono vietate configurazioni che ammettono una mossa e una $\epsilon$-mossa, quindi $\forall q \in Q, z \in \Gamma$ se $\delta(q, \epsilon, z) \neq \varnothing$ allora $\forall a \in \Sigma \delta(q, a, Z) = \varnothing$
	\item $\forall q \in Q, z \in \Gamma, a \in \Sigma \cup \{\epsilon\} \mid |\delta(q, a, Z)|\leq 1$, quindi per ogni tripletta $q, a, Z$ è ammessa al massimo una mossa.
\end{itemize}
A questo punto abbiamo definito quattro modelli: deterministico e nondeterministico che possono accettare per pila vuota o per stato finale.
Vedremo che il caso nondeterminismo in questo caso è più potente del caso deterministico, e che nel modello nondeterministico automi che possono accettare per pila vuota o per stato finale sono equivalenti.

\subsection{Equivalenza tra le due nozioni di accettazione nel modello nondeterministico}
\subsubsection{Da stati finali a pila vuota}
Dato un automa $M = (Q, \Sigma, \Gamma, \delta, q_0, Z_0, F)$ e supponiamo che $L = L(M)$ sia il linguaggio accettato per stati finali.
Definiamo l'automa 
$$M' = (Q \cup \{q_0', q_e\}, \Sigma, \Gamma \cup \{X\}, \delta', q_0', X, \varnothing)$$
con $q_0', q_e \not \in Q$ e $X \not \in \Gamma$, vogliamo che $L= N(M')$.

Ad alto livello quando $M$ arriva in uno stato finale, $M'$ si sposta nello stato $q_e$ in cui inizia a svuotare la pila.
Infatti la $e$ di $q_e$ sta per ``empty''.

Definiamo ora $\delta'$:
\begin{enumerate}
	\item prima di tutto 
		$$ \delta'(q_0', \epsilon, X) = \{(q_0, Z_0X)\} $$
		questo serve solo ad infilare $X$ in fondo alla pila.
		La $X$ è necessaria per evitare che se l'automa iniziale $M$ svuota la pila si accetti la stringa.
	\item per ogni altra cosa $M'$ si può comportare come $M$:
		$$ \forall q \in Q, a \in \Sigma \cup \{\epsilon\}, z \in \Gamma \mid \delta(q, a, Z) \subseteq \delta'(q, a, Z) $$
	\item ogni qualvolta $M$ entra in uno stato finale $M'$ può -- enfasi su può -- iniziare a svuotare l'intera pila:
		$$ \forall q \in F, z \in \Gamma \cup \{X\} \mid (q_e, \epsilon) \in \delta'(q, \epsilon, Z) $$
	\item una volta entrato nello stato di svuotamento, continua a svuotare:
		$$ \forall z \in \Gamma \cup \{X\} \mid \delta'(q_e, \epsilon, Z) = \{(q_e, \epsilon)\}$$ 
\end{enumerate}

Questo necessariamente introduce nondeterminismo, infatti l'automa $M$ potrebbe entrare in uno stato finale prima di essere arrivato alla fine della stringa.
Ed anche se l'automa di partenza è deterministico il punto $3$ potrebbe in ogni caso introdurre nondeterminismo.

Supponiamo di avere un automa deterministico che accetta la stringa $w$ a pila vuota, allora ogni stringa che ha $w$ come prefisso non può essere accettata, perché il prefisso $w$ svuoterebbe la pila e un automa con pila vuota non può andare a avanti.
Quindi il nondeterminismo è in un certo senso necessario per automi a pila che accettano con pila vuota.

\subsubsection{Da pila vuota a stati finali}
Dato un automa $M = (Q, \Sigma, \Gamma,\delta, q_0, Z_0, \varnothing)$ che accetta per pila vuota il linguaggio $L = N(M)$, vogliamo creare un automa che accetti per stati finali.
Sia questo 
$$M' = (Q \cup \{q_0', q_F\}, \Sigma, \Gamma \cup \{X\}, \delta', q_0', X, F = \{q_F\})$$
con $q_0', q_F \not \in Q, X \not \in \Gamma$.

Definiamo ora $\delta'$:
\begin{itemize}
	\item come prima inizialmente infiliamo $X$ in fondo alla pila:
		$$ \delta'(q_0', \epsilon, X) = \{(q_0, Z_0X)\} $$
		$X$ serve a riconoscere quando la pila è vuota.
	\item a questo punto copiamo tutte le mosse di $M$, per cui
		$$ \forall q \in Q, a \in \Sigma \cup \{\epsilon\}, Z \in \Gamma \mid \delta'(q, a, Z) = \delta(q, a, Z) $$
	\item nel momento in cui $M$ svuota la prima, $M'$ si trova $X$ sulla pila, a questo punto può entrare in uno stato finale
		$$ \forall q \in Q \mid \delta'(q, \epsilon, X) = \{(q_F, \epsilon)\}$$
\end{itemize}
Supponendo che $M$ sia deterministico, $M'$ rimane deterministico -- la trasformazione preserva il determinismo.
\end{document}

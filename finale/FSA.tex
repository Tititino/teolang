\documentclass[12pt]{report}
\usepackage{amsmath}
\usepackage{amsfonts}
\usepackage{amssymb}
\usepackage{amsthm}
\usepackage[italian]{babel}
\usepackage{bytefield}
\usepackage{cancel}
\usepackage{embedall}\embedfile{\jobname.tex}
\usepackage[bookmarks]{hyperref}
\usepackage{listings}
\usepackage[scr=rsfs]{mathalpha}
\usepackage{siunitx}
\usepackage{tabularray}
\usepackage{tcolorbox}
\usepackage{tikz}\usetikzlibrary{automata, decorations.text, patterns, decorations.pathmorphing, positioning, decorations.pathreplacing, calligraphy}
\usepackage{todonotes}
\usepackage{xcolor}

\newtheorem{teorema}{Teorema}
\newtheorem{corollario}{Corollario}
\newtheorem{proposizione}{Proposizione}
\newtheorem{proprietà}{Proprietà}
\newtheorem{lemma}{Lemma}
\newtheorem{fatto}{Fatto}
\newtheorem{definizione}{Definizione}
\newtheorem{nota}{Nota}

\renewcommand\qedsymbol{$\blacksquare$}

\usepackage{teolang}

\definecolor{codegray}{gray}{0.95}

\lstdefinestyle{mystyle}{
  numberstyle=\tiny,
  basicstyle=\footnotesize,
  breaklines=true,
  numbers=left,
  numbersep=5pt,
}
\lstset{style=mystyle}

\overfullrule=0.2cm
\begin{document}
\tableofcontents
\newpage

\chapter{Intro}
Studieremo dei sistemi particolari (macchine/grammatiche) e il loro comportamento (il linguaggio descritto) e le risorse utilizzate (risorse utilizzate, risorse per trasformazioni da una macchina e l'altra, \dots). 

\section{Terminologia}
Quando si parla di un linguagio ci sono due aspetti fondamentali:
\begin{itemize}
	\item sintassi, formata da
	\begin{itemize}
		\item i segni o simboli utilizzati
		\item le regole che ci permettono di combinarli
	\end{itemize}
	\item semantica, il processo che associa alle frasi del linguaggio un significato
\end{itemize}

Noi ci occuperemo principalmente di sintassi.

In ambito linguistico nel 1956 Chomsky introduce le grammatiche formali.
Questi sono degli oggetti matematici, il cui scopo originale era descrivere in modo rigoroso la grammatica dell'inglese.
Nello stesso periodo si stava sviluppando l'idea di compilatore, specificamente il compilatore Fortran.
Si vide che il modello proposto da Chomksy era adatto per formalizzare i linguaggi di programmazione.

Definiamo
\begin{itemize}
	\item l'alfabeto ($\Sigma$) è un insieme finito non vuoto di simboli, e.g. 
		$$ \Sigma = \{ a , b , c \} $$
	\item una stringa o parola è una sequenza di simboli sull'alfabeto.
		La sua lunghezza è definita come il numero di simboli che la compongono
		$$ | abc | = 3 $$
		Tra tutte le parole abbiamo una parola speciale, che è la parola vuota costituita da zero simboli.
		Questa è indicata con $\epsilon$ (alcuni la indicano con $\lambda$ o $\Lambda$).
		La lunghezza di questa è 0.

		Sia $a \in \Sigma$ e $x \in \Sigma^*$, definiamo 
		$$ \#_a(x) : \Sigma \times \Sigma^* \mapsto \mathbb{N} $$
		come il numero di $a$ in $x$.
	\item indichiamo l'insieme di tutte le possibili stringhe sull'alfabeto $\Sigma$ con $\Sigma^*$.
		Queste sono enumerabili.

		Chiamiamo $\Sigma^n$ l'insieme di tutte le stringhe di $n$ lettere sull'alfabeto $\Sigma$.
		% non esattamente
	\item definiamo un'operazione 
		$$ - \cdot - : \Sigma^* \times \Sigma^* \mapsto \Sigma^*$$
		come la concatenazione o prodotto di due stringhe.
		Ad esempio
		\begin{align*}
			x &= abbac \\
			y &= ca \\
			xy &= abbacca \\
			yx &= caabbac
		\end{align*}
		La parola vuota $\epsilon$ è l'identità sia destra che sinistra di questa operazione.
		L'operazione è associativa ma non commutativa.
		La struttura $\langle \Sigma^*, \cdot, \epsilon \rangle$ è detto monoide libero generato dall'alfabeto $\Sigma$.
	\item data una stringa $x \in \Sigma^*$ diciamo che $y$ è prefisso di $x$ se
		$$ \exists z \in \Sigma^* \mid x = yz $$
		Una stringa ha un numero di prefissi pari alla sua lunghezza + 1.
		$y$ è prefisso proprio di $x$ se è prefisso e non è $\epsilon$.
		Similmente definiamo il suffisso di una parola $x$, quindi $y$ è suffisso di $x$ se 
		$$ \exists x \in \Sigma^* \mid x = zy $$
		Infine diciamo che $y$ è fattore di $x$ se
		$$ \exists z, w \in \Sigma^* \mid x = zyw $$
		I prefissi e i suffissi sono particolari tipi di fattori.
		Ogni parola ha all'incirca $\frac{n^2}{2}$ fattori, infatti è facile notare che $x = aaaa$ ha diversi fattori uguali.

		Una sottosequenza di solito è una serie di elementi scelti da una parola, quindi un fattore può essere definito come una sottosequenza contigua. 
		Ad esmepio una sottosequenza di $x = abbac$ può essere $abc$.
		Sottostringa e sottoparola di solito sono sinonimi di fattore, ma a volte possono essere utilizzate come sinonimo di sottosequenza.
	\item un linguaggio $L$ è definito come un qualunque sottoinsieme di $\Sigma^*$.
		Abbiamo alcuni linguaggi particolari:
		\begin{itemize}
			\item il linguaggio vuoto $L = \varnothing$
			\item il linguaggio della parola vuota $L = \{ \epsilon \}$
			\item il linguaggio pieno $L = \Sigma^*$
		\end{itemize}
\end{itemize}

Ad esempio definiamo il linguaggio delle parole che finiscono con due $a$ come
$$ L = { w \in \Sigma^* \mid \exists y \in \Sigma^* w = yaa } $$
quindi con un insieme finito di simboli siamo risuciti a rappresentare un insieme infinito.
Questo tipo di descrizione è detta \textit{dichiarativa}.

% Definiamo l'alfabeto $\Sigma = { 0, 1, \dots, 9 }$, e definiamo $L$ come tutte le sequenze formate da 3 cifre decimali che appaiono consecutivamente nella rappresentazione del $\pi$ greco.
% embé?
In altri casi utilizzeremo delle descrizioni:
\begin{itemize}
	\item \textit{generative}: si fornisce un insieme di regole che permette di costruire tutte le possibili stringhe del linguaggio, ad esempio le grammatiche
	\item \textit{riconoscitive}: si fornisce un riconoscitore che riceve una stringa $x$ e risponde alla domanda $x \overset{?}{\in} L$, in alcuni casi queste macchine si limitano a dire sì se la stringa appartiene o non terminare in altro caso
\end{itemize}

\begin{fatto}
	Non tutti i linguaggi si possono descrivere in maniera finita, infatti le descrizioni finite sono numerabili, mentre i linguaggi no.
\end{fatto}

\begin{tcolorbox}
 	Prendiamo come alfabeto $\Sigma = \{ (, ) \}$, vogliamo descrivere $L$ l'insieme delle sequenze di parentesi bilanciate correttamente.
 	Definiamo la rappresentazione generativa che genera $L_G$:
 	\begin{enumerate}
 		\item $\epsilon \in L_G$
 		\item se $x \in L_G$ allora $(x) \in L_G$
 		\item se $x, y \in L_G$ allora $xy \in L_G$
 	\end{enumerate}
 
 	E definiamo un riconoscitore, che riconosca $L_R$:
 	\begin{itemize}
 		\item il numero di aperte deve essere uguale al numero di chiuse, quindi $\#_((x) = \#_)(x)$
 		\item per ogni prefisso il numero di parentesi aperte è maggiore o uguale del numero di chiuse, quindi per $y$ prefisso $\#_((y) \geq \#_)(y)$.
	\end{itemize}
 	Questo può essere generato come un automa con contatore.
 
 	Si può dimostrare che questi due definiscono lo stesso linguaggio. % da fare
\end{tcolorbox}

\begin{tcolorbox}
	Ora prendiamo l'alfabeo $\Sigma = \{(, ), [, ]\}$, e definiamo $L$ come il linguaggio delle sequenze di parentesi bilanciate correttamente.
	Definiamo la rappresentazione generativa che genera $L_G$:
	\begin{enumerate}
		\item $\epsilon \in L_G$
		\item se $x \in L_G$ allora $(x) \in L_G$ e $[x] \in L_G$
		\item se $x, y \in L_G$ allora $xy \in L_G$
	\end{enumerate}

	E definiamo un riconoscitore, che riconosca $L_R$:
	\begin{itemize}
		\item il numero di aperte deve essere uguale al numero di chiuse, quindi $\#_((x) = \#_)(x)$ e $\#_[(x) = \#_](x)$
		\item si utilizza una pila in cui si mette ogni parentesi aperta e si toglie quando si trova una parentesi chiusa nel caso corrisponda
	\end{itemize}
	Quindi un automa a pila.

	In alcuni casi è più facile definire come si genera, e in altri come riconoscere.
\end{tcolorbox}

\chapter{Grammatiche}
Una grammatica è una quadrupla $\langle V, \Sigma, P, S \rangle$, dove
\begin{itemize}
	\item $V$ è l'insieme finito e non vuoto delle variabili, questi vengono anche chiamate non terminali.
		Questo lo possiamo chiamare anche l'alfabeto delle variabili.
	\item $\Sigma$ è l'insieme finito e non vuoto dei termianli, o anche alfabeto dei terminali.
	\item $P$ è l'insieme finito delle regole di produzione.
		Queste sono della forma
		$$ \alpha \rightarrow \beta $$
		con $\alpha \in (V \cup \Sigma)^+$ e $\beta \in (V \cup \Sigma)^*$.
	\item $S$ è un elemento di $V$ che chiamiamo simbolo iniziale o assioma.
\end{itemize}
Per $x, y \in (V \cup \Sigma)^*$, diciamo che $x$ deriva $y$, indicato con $x \Rightarrow y$ (opzionalmente indicando la grammatica $x \underset{G}{\Rightarrow} y$), se e solo se 
$$ \exists (\alpha \rightarrow \beta) \in P, \exists \eta, \delta \in (V \cup \Sigma)^* \mid x = \eta \alpha \delta \wedge y = \eta \beta \delta $$
Inoltre diciamo che $x$ deriva $y$ in $k$ passi, con $k \in \mathbb{N}$, indicato $x \overset{k}{\Rightarrow} y$, se e solo se
$$ \exists x_0, x_1, \dots, x_k \mid x_0 = x \wedge x_k = y \wedge \forall i \in 1, \dots, k x_{i-1} \Rightarrow x_i $$
Infine diciamo che $x$ deriva $y$ in un certo numero di passi, indicato con $x \overset{*}{\Rightarrow} y$, se e solo se
$$ \exists k \geq 0 \mid x \overset{k}{\Rightarrow} y $$
e che $x$ deriva $y$ in almeno un passo, indicato con $x \overset{+}{\Rightarrow} y$, se e solo se
$$ \exists k > 0 \mid x \overset{k}{\Rightarrow} y $$

Definiamo il linguaggio generato dalla grammatica $G$ 
$$L(G) = \{ w \in \Sigma^* \mid S \overset{*}{\Rightarrow} w \} $$

Una stringa di terminali e non terminali è detta \textit{forma sentenziale}.

Due grammatiche $G_1$ e $G_2$ sono equivalenti se e solo se $L(G_1) = L(G_2)$.

\begin{tcolorbox}% [breakable]
	Sia $\Sigma = \{ (, ) \}$.
	Costruiamo $V$, inizialmente contentente solo $S$.
	Ora la sequenza vuota è bilanciata, quindi $S \rightarrow \epsilon$.
	Una sequenza di parentesi bilanciata rimane bilanciata se inserita tra due parentesi, quindi $S \rightarrow (S)$.
	Infine la concatenazione di due sequenze bilanciate è ancora bilanciata, quindi $S \rightarrow S S $.

	Ad esempio 
	\begin{align*}
		S &\Rightarrow ( S ) \\
		  &\Rightarrow ( ( S ) ) \\
		  &\Rightarrow ( ( S S ) ) \\
		  &\Rightarrow ( ( ( S ) S ) ) \\
		  &\Rightarrow ( ( ( S ) ( S ) ) ) \\
		  &\Rightarrow ( ( ( ) ( S ) ) ) \\
		  & \Rightarrow ( ( ( ) ( ) ) ) 
	\end{align*}
\end{tcolorbox}

\begin{tcolorbox}
	Data la grammatica $\langle \{S, B\}, \{a, b, c\}, P, S \rangle$, con $P$
	\begin{align*}
		S &\rightarrow a B S c \\
		S &\rightarrow a b c \\
		B a &\rightarrow a B \\
		B b &\rightarrow b b 
	\end{align*}
	Eseguiamo una derivazione
	\begin{align*}
		S &\Rightarrow a B S c \\
		  &\Rightarrow a B a b c c \\
		  &\Rightarrow a a B b c c \\
		  &\Rightarrow a a b b c c \\
		S &\Rightarrow a B S c \\
		  &\Rightarrow a B a B S c c \\
		  &\Rightarrow a a B B S c c \\
		  &\Rightarrow a a B B a b c c c \\
		  &\Rightarrow a a B a B b c c c \\
		  &\Rightarrow a a a B B b c c c \\
		  &\Rightarrow a a a B b b c c c \\
		  &\Rightarrow a a a b b b c c c
	\end{align*}
	Si può dimostrare che il linguaggio generato da questa grammatica è $L = {a^nb^nc^n \mid n > 0 }$.
\end{tcolorbox}

Le grammatiche sono classificate in base alla forma delle produzioni, possiamo definire quattro tipi di grammatiche:
\begin{center}
\begin{tblr}{|c|c|c|}
	\hline
	tipo 0 & \parbox{0.6\textwidth}{nessuna restrizione} &  \\
	\hline
	tipo 1 & \parbox{0.6\textwidth}{Se $\alpha \rightarrow \beta$ è una produzione, allora $|\beta| \geq |\alpha|$. Equivalentemente una grammatica in cui tutte le produzioni sono della forma $\alpha_1 A \alpha_2 \rightarrow \alpha_1 \beta \alpha_2$ dove $\alpha_1, \alpha_2 \in (V \cup \Sigma)^*, A \in V, \beta \in (V \cup \Sigma)^+$} & \parbox{0.2\textwidth}{anche dette context sensitive o dipendenti da contensto} \\
	\hline
	tipo 2 & \parbox{0.6\textwidth}{Se $\alpha \rightarrow \beta \in P$, allora $\alpha \in V$ e $\beta \in (V \cup \Sigma)^+$} & \parbox{0.2\textwidth}{contenxt free o libere da contesto} \\
	\hline
	tipo 3 & \parbox{0.6\textwidth}{Possiamo avere produzioni solo della forma $A \rightarrow \alpha B$ o $A \rightarrow \alpha$, con $A, B \in V, \alpha \in \Sigma$} & \parbox{0.2\textwidth}{grammatiche regolari}  \\
	\hline
\end{tblr}
\end{center}
Ogni grammatica restringe il livello precedente

\todo{Sistemo}
\begin{center}
	\begin{tikzpicture}
		\node[rectangle, fill=white!80] at (4.5cm, 3cm) {\tiny \ Tipo 0\ };
		\draw[pattern={north east lines}, pattern color=gray] (-5cm, -3.5cm) rectangle (5cm, 3.5cm);
 		\draw[ postaction={ decoration={ text effects along path
 		                               , text={\tiny \ Tipo 1\ }
 				               , text align=right
 				               , reverse path
 				               , text effects/.cd
 				               , text along path, every character/.style={fill=white, yshift=-0.5ex} 
 					       }
 				  , decorate 
				  }, fill=white] (0, 0) ellipse (4cm and 3cm);
 		\draw[ postaction={ decoration={ text effects along path
 		                               , text={\tiny \ Tipo 2\ }
 				               , text align=right
 				               , reverse path
 				               , text effects/.cd
 				               , text along path, every character/.style={fill=white, yshift=-0.5ex} 
 					       }
 				  , decorate 
				  }, fill=white] (0, 0) ellipse (3cm and 2cm);
 		\draw[ postaction={ decoration={ text effects along path
 		                               , text={\tiny \ Tipo 3\ }
 				               , text align=right
 				               , reverse path
 				               , text effects/.cd
 				               , text along path, every character/.style={fill=white, yshift=-0.5ex} 
 					       }
 				  , decorate 
 				  }, fill=white] (0, 0) ellipse (2cm and 1cm);
	\end{tikzpicture}
\end{center}

\section{Classificazione dei linguaggi}
Sia $\mathscr{L} \subseteq \Sigma^*$, diciamo che $\mathscr{L}$ è di tipo $0 \leq i \leq 3$, se e solo se $\exists G$ di tipo $i$ tale che $\mathscr{L} = L(G)$.
Questa classificazione genera una gerachia dei linguaggi, detta gerarchia di Chomksy, e chiamiamo
\begin{itemize}
	\item i linguaggi di tipo 3 come linguaggi regolari
	\item i linguaggi di tipo 2 come linguaggi conxtext free o CF
	\item i linguaggi di tipo 1 come linguaggi context sensitive o CS
	\item i linguaggi di tipo 0 come linguaggi ricorsivamente enumerabili (RE) o semidecibili
\end{itemize}

E per ognuna di queste classi esiste un modello di macchina riconoscitrice:
\begin{itemize}
	\item per i linguaggi di tipo 3 sono gli automi a stati finiti
	\item per i linguaggi di tipo 2 sono gli automi a pila
	\item per i linguaggi di tipo 1 sono gli automi limitati linearmente, questi hanno un nastro finito che possono modificare
	\item per i linguaggi di tipo 0 sono le macchine di Turing, nella variante più semplice queste sono macchine con un nastro infinito su cui inizialmente c'è scritto l'input.
		Per questi linguaggi se la stringa appartiene al linguaggio il riconoscitore termina sempre con sì, mentre se la stringa non appartiene al linguaggio il riconoscitore può anche non terminale ($\bot$)
\end{itemize}

\begin{teorema}
	I linguaggi di tipo 1, 2 e 3 sono anche detti decidibili o ricorsivi.
	(Un insieme ricorsivo è un insieme per cui posso avere una macchina che può rispondere sì o no alla domanda se un insieme appartenga)
\end{teorema}
\begin{proof}
Sia $G$ una grammatica di tipo 1 e $w$ una stringa di terminali di lunghezza $n = |w|$, voglio sapere se $w$ appartiene a $G$.
Siccome la grammatica di tipo 1 è monotonica (le regole forme sentenziali non possono decrescere), queste non possono mai superare $n$ durante la derivazione di $w$.
Chiamiamo 
$$T_i = { \gamma \in (V \cup \Sigma)^{\leq n} \mid S \overset{\leq i}{\Rightarrow} \gamma }$$
Costruiamo gli insiemi $T_i$ per $i = 0, 1, \dots$
\begin{align*}
	T_0 &= { S } \\
	    &\vdots \\
	T_i &= T_{i - 1} \cup { \gamma \in (V \cup \Sigma)^{\leq n} \mid \exists \beta \in T_{i - 1} . \beta \Rightarrow \gamma } 
\end{align*}
Allora 
$$T_0 \subseteq T_1 \subseteq \dots \subseteq T_{i - 1} \subseteq T_{i} \subseteq \dots \subseteq (V \cup \Sigma)^{\leq n}$$
Dato che $(V \cup \Sigma)^{\leq n}$ è un insieme finito, quindi esiste un certo $i$ tale che $T_i = T_{i - 1}$.
A questo punto ho trovato tutte le forme sentenziali di lunghezza $n$, e per vedere se $w$ appartiene al linguaggio basta controllare che $w \in T_i$.


Quello che posso fare è considerare l'insieme 
$$U_i = { \gamma \in (V \cup \Sigma)^* \mid S \overset{\leq i}{\Rightarrow} \gamma }$$
Quindi ancora
\begin{align*}
	U_0 &= { S } \\
	    &\vdots \\
	U_i &= U_{i - 1} \cup { \gamma \in (V \cup \Sigma)^* \mid \exists \beta \in U_{i - 1} . \beta \Rightarrow \gamma } 
\end{align*}
$$U_0 \subseteq U_1 \subseteq \dots \subseteq U_{i - 1} \subseteq U_{i} \subseteq \dots \subseteq (V \cup \Sigma)^*$$
Quindi che nei linguaggi di tipo 0 perdo la monotonicità delle derivazioni, non posso più trovare una $i$ tale che $U_i = U_{i - 1}$.
Però per ogni $U_i$ posso controllare se $w \in U_i$, infatti se la grammatica genera $w$, prima o poi questa la troverò, se non la genera non la troverò mai.
\end{proof}

Il motivo per cui si dicono ricorsivamente enumerabili è perché si può costruire un programma che elenchi tutti gli elementi del linguaggio.

%% fine lezione 2, inizio lezione 3
La gerarchia di Chomksy vale per alfabeti di almeno due lettere.

\section{Parola vuota ed $\epsilon$-produzioni}
Supponendo di avere una grammatica $G$ di tipo 1, supponiamo di voler aggiungere la parola vuota al linguaggio generato.
Se si aggiunge banalmente la regola per la parola vuota
$$ S \prd \epsilon $$
non ci sono garanzie che noi non abbiamo aggiunto anche altre stringhe al linguaggio, ad esempio nel caso
$$ \alpha \der \dots S \dots $$
inoltre perdiamo la proprietà della monotonicità delle forme sentenziali.

Quindi bisogna essere leggermente più delicati, dato $L = L(G)$, voglio costruire $G^\prime$ tale che $L^\prime = L(G^\prime) = L(G) \cup \{ \epsilon \}$.
Per gare ciò definisco $G^\prime = \grammar{V \cup \{S^\prime\}}{\Sigma}{P \cup \{ S^\prime \prd \epsilon \mid S\}}{S^\prime}$.

Dal simbolo iniziale posso utilizzare $S \prd \epsilon$ purché $S$ non appaia su lato destro di nessuna produzione.
Questo ovviamente vale anche per grammatiche di tipo 2 e 3.

Data una grammatica di tipo 2 che utilizza $\epsilon$-produzioni 
$$ A \prd \epsilon $$
è possibile trovare una grammatica di tipo 2 equivalente che genera lo stesso linguaggio, meno la parola vuota, che non utilizza $\epsilon$-produzioni.

Anche nelle gramamtiche di tipo 3 possiamo ammettere $\epsilon$-produzioni
$$ A \prd \epsilon $$

\section{Automa a stati finiti}
Un automa è un dispositivo che ha
\begin{itemize}
	\item un nastro diviso in celle che contiene l'input, ogni cella contiene un simbolo dell'alfabeto
		Questo non può essere modificato.
	\item una testina che passa il nastro da sinistra a destra, per questo vengono chiamati anche automi $one-way$.
		Quando questa finisce il nastro deve poter rispondere se la parola appartiene o no al linguaggio.
		La testina contiene una macchina che ha un insieme finito di stati.
\end{itemize}
Un automa è quindi una quintupla $\mathscr{A} = \langle Q, \Sigma, \delta, q_0, F \rangle$, dove
\begin{itemize}
	\item $Q$ è l'insieme degli stati
	\item $\Sigma$ è l'alfabeto finito di input
	\item $\delta$ è il programma dell'automa, o funzione di transizione
	\item $q_0 \in Q$ è lo stato iniziale
	\item $F \subseteq Q$ è l'insieme degli stati finali
\end{itemize}
Ad ogni passo l'automa legge un simbolo, ed in base allo stato attuale e il simbolo in input sceglie il prossimo stato con $\delta$
$$ \delta : Q \times \Sigma \rightarrow Q $$
Definiamo 
$$ \delta^* : Q \times \Sigma^* \rightarrow Q $$
per induzione sulla lunghezza delle stringhe in input:
\begin{itemize}
	\item $\forall q \in Q \mid \delta^*(q, \epsilon) = q $
	\item $\forall q \in Q, x \in \Sigma^*, a \in \Sigma \mid \delta^*(q, x a) = \delta(\delta^*(q, x), a)$
\end{itemize}
Visto che $\delta^*$ è effettivamente una estensione di $\delta$, chiameremo $\delta^*$ $\delta$.

Ora possiamo definire il linguaggio riconosciuto dall'automa $\mathscr{A}$ come
$$ L(\mathscr{A}) = \{ w \in \Sigma^* \mid \delta(q_0, w) \in F \} $$

\begin{tcolorbox}
	Ad esempio $\mathscr{A} = \langle Q = \{ q_0, q_1 \}, \Sigma = \{ a, b \}, \delta, q_0, F = \{ q_1 \} \rangle$, con $\delta$ definita come
	\begin{center}
	%	\begin{tblr}{c| c c }
	%		\delta & a & b \\
	%		\hline
	%		q_0    & q_0 & q_1 \\
	%		q_1    & q_1 & q_0 \\
	%	\end{tblr}
	\end{center}
	Per $aabb$ abbiamo
	$$ q_0 \overset{a}{\rightarrow} q_0 \overset{a}{\rightarrow} q_0 \overset{b}{\rightarrow} q_1 \overset{b}{\rightarrow} q_0 $$
	quindi la stringa non è accettata perché $q_0 \not \in F$.
	Questo è linguaggoi di tutte le stringhe con un numero dispari di $b$.
\end{tcolorbox}
Invece che rappresentare $\delta$ come una tabella, di solito è comodo rappresentarla come un diagramma di transizione:
\begin{itemize}
	\item Lo stato iniziale è rappresentato da una freccia entrante
	\item gli stati finali da un doppio cerchio.
\end{itemize}
Quindi una derivazione non è altro che un cammino in questo grafo, e la parola è accettata se si finisce in uno stato finale.

\begin{tcolorbox}
	Dato $\Sigma = \{ a, b \}$, e il linguaggio 
	$$ \mathscr{L} = \{ x \in \Sigma^* \mid P(x) \}$$
	con $P$ definita come la proprietà per cui tra ogni coppia di $b$ successive in $x$ c'è un numero di $a$ pari.
	\begin{center}
		\begin{tikzpicture}
			\node[state, initial, accepting] 	(q_0) {$q_0$};
			\node[state, accepting] 		(q_1) [right=of q_0] {$q_1$};
			\node[state, accepting]			(q_2) [right=of q_1] {$q_2$};
			\node[state]				(q_3) [right=of q_2] {$q_3$};

			\path[->] (q_0)	edge[loop above]	node[above]	{a}	()
					edge			node[above]	{b}	(q_1)
				  (q_1)	edge[loop above]	node[above]	{b}	()
					edge			node[above]	{a}	(q_2)
				  (q_2)	edge			node[above]	{}	(q_1)
				  	edge			node[above]	{b}	(q_3)
				  (q_3)	edge[loop above]	node[above]	{a, b}	();
		\end{tikzpicture}
	\end{center}
	% 
	%  init(final(q_0)) --b--> final(q_1) <--a--> final(q_2) --b--> trap(q_3)
	%         \_a_/             \_b_/                                \_a,b_/
	Visto che 
	$$ \delta : Q \times \Sigma \rightarrow Q $$
	nello stato $q_2$ bisogna aggiungere una freccia ad uno stato da cui non si può uscire, infatti se siamo in quello stato vuol dire che abbiamo trovato un numero dispari di $a$ e una $b$, quindi la parola non è valida.
	Questo tipo di stati è detto stato trappola, e di solito vengono lasciati impliciti.
\end{tcolorbox}

\begin{tcolorbox}
	Dato $\Sigma = \{ a, b \}$, e il linguaggio 
	$$ \mathscr{L} = \{ x \in \Sigma^* \mid \text{Il terzo simbolo di $x$ è una $a$}\}$$
	% init(q_0) --a,b--> q_1 --a,b--> q_2 --a--> final(q_3)
	%                                             \_a,b_/
	\begin{center}
		\begin{tikzpicture}
			\node[state, initial] 	(q_0) {$q_0$};
			\node[state]		(q_1) [right=of q_0] {$q_1$};
			\node[state]		(q_2) [right=of q_1] {$q_2$};
			\node[state, accepting]	(q_3) [right=of q_2] {$q_3$};

			\path[->] (q_0)	edge			node[above]	{a, b}	(q_1)
				  (q_1)	edge			node[above]	{a,b}	(q_2)
				  (q_2)	edge			node[above]	{a}	(q_3)
				  (q_3)	edge[loop above]	node[above]	{a, b}	();
		\end{tikzpicture}
	\end{center}
\end{tcolorbox}
\begin{tcolorbox}
	Dato $\Sigma = \{ a, b \}$, e il linguaggio 
	$$ \mathscr{L} = \{ x \in \Sigma^* \mid \text{ Il terzultimo simbolo di $x$ è una $a$} \} $$
	Teniamo uno stato per ogni tripla di $a, b$, quindi $2^3 = 8$ stati
	% graph automa {
	% aaa -b-> aab
	% aaa -a-> aaa
	% aab -a-> aba
	% aab -b-> abb
	% aba -a-> baa
	% aba -b-> bab
	% bab -a-> aba
	% bab -b-> abb
	% abb -a-> bba
	% abb -b-> bbb
	% baa -a-> aaa
	% baa -b-> aab
	% bbb -b-> bbb
	% bbb -a-> bba
	% bba -a-> baa
	% bba -b-> bab
	% fin(aaa)
	% fin(aab)
	% fin(aba)
	% fin(abb)
	% init(bbb)
	% }
\end{tcolorbox}

Una operazione che può interessare per le stringhe è costruire il linguaggio che riconosce le stringhe roversciate di uno.
Dato un linguaggio $\mathscr{L}$, il linguaggio che riconosce le sue stringhe inverse è detto il reversal di $\mathscr{L}$.
Nel caso dei due linguaggi precedenti se prendiamo l'automa del primo e lo invertiamo ottieniamo un automa valido (non deterministico) per il secondo.

\subsection{Automi non deterministici}
Definiamo un automa non deterministico come la quintupla $\mathscr{A} = \langle Q, \Sigma, \delta, q_0, F \rangle$, ma ora $\delta$ è definita come
$$ \delta : Q \times \Sigma \rightarrow 2^Q $$
quindi $\delta$ non ritorna più un singolo insieme ma un insieme di possibili stati.

Una stringa viene accettata se esiste almeno un cammino che mi porta in uno stato finale.
Estensiamo ancora $\delta$ alle stringhe 
$$ \delta^* : Q \times \Sigma \rightarrow 2^Q $$
è definita come
\begin{itemize}
	\item $\forall q \in Q \mid \delta^*(q, \epsilon) = \{ q \} $
	\item $\forall q \in Q, x \in \Sigma^*, a \in \Sigma \mid \delta^*(q, xa) = \bigcup_{p \in \delta^*(q, x)} \delta(p, a) $
\end{itemize}

Come prima chiameremo $\delta^*$ semplicemente $\delta$.

Ora il linguaggio accettato dall'automa $\mathscr{A}$ non deterministico è $L(\mathscr{A}) = \{ w \in \Sigma^* \mid \delta(q_0, w) \cap F \neq \varnothing \}$.

\begin{tcolorbox}
	Dato l'automa non deterministico
	\begin{center}
		\begin{tikzpicture}
			\node[state, initial] 	(r_0) {$r_0$};
			\node[state] 	      	(r_1) [right=of r_0] {$r_1$};
			\node[state] 	      	(r_2) [right=of r_1] {$r_2$};
			\node[state, accepting]	(r_3) [right=of r_2] {$r_3$};

			\path[->] (r_0)	edge	[loop above]	node[above]	{a, b}	()
					edge			node[above]	{a}	(r_1)
				  (r_1)	edge			node[above]	{a, b}	(r_2)
				  (r_2)	edge			node[above]	{a, b}	(r_3);
		\end{tikzpicture}
	\end{center}
	con la stringha $ababb$ mostriamo tutte le possibili strade che si possono prendere
	\begin{center}
		\begin{tikzpicture}
			\node {$r_0$}
				child { node {$r_0$}
					child {	node {$r_0$}
						child {	
							node {$r_0$} 
							child {
								node {$r_0$}
								child {
									node {$\cancel{r_0}$}
									edge from parent node[left] {b}
								}
								edge from parent node[left] {b}
							}
							edge from parent node[left] {a}
						}
						child { 
							node {$r_1$} 
							child {
								node {$r_2$}
								child {
									node {$r_3$}
									edge from parent node[right] {b}
								}
								edge from parent node[right] {b}
							}
							edge from parent node[right] {a}
						}
						edge from parent node[left] {b}
					}
					edge from parent node[left] {a}
				}
				child {	
					node {$r_1$}
					child {
						node {$r_2$}
						child {
							node {$\cancel{r_3}$}
							edge from parent node[right] {a}
						}
						edge from parent node[right] {b}
					}
					edge from parent node[right] {a}
				};
		\end{tikzpicture}
	\end{center}
	è necessario che esista almeno una strada dell'albero che risponda sì per accettare.
	Questa struttura è chiamato albero di computazione, e un cammino descrive una computazione.

	Bisogna supporre che l'automa riesca a sceglie la strada esatta.
	Alternativamente si può pensare che l'automa possa visitare tutte le strade in parallelo, quindi possiamo trasformare l'albero in
	\begin{center}
		\begin{tikzpicture}
			\node[state, initial] 	(r0) {$\{ r_0 \}$};
			\node[state] 	      	(r0r1)	[right=of r0] {$\{ r_0, r_1 \}$};
			\node[state] 	      	(r0r2)	[right=of r0r1] {$\{ r_0, r_2 \}$};
			\node[state] 	      	(r0r1r3)[below right=of r0r2] {$\{ r_0, r_1, r_3 \}$};
			\node[state, accepting]	(r0r3) 	[right=of r0r2] {$\{ r_0, r_3 \}$};

			\path[->] (r0)		edge	node[above]	{a}	(r0r1)
				  (r0r1) 	edge	node[above]	{b}	(r0r2)
				  (r0r2)	edge	node[above]	{a}	(r0r1r3)
				  		edge 	node[above]	{b}	(r0r3)
				  (r0r1r3)	edge	node[above]	{a}	(r0r2);
		\end{tikzpicture}
	\end{center}
	Questo 
\end{tcolorbox}
Ogni autma non deterministico (NFA) con $n$ stati può sempre essere trasformato in un automa deterministico con al più $2^n$ stati.
Nel caso degli automi a pila invece il modello non deterministico è più potente.
Qui invece tra DFA e NFA non cambia la potenza computazionale, ma la semplicità descrittiva.

%% FINE LEZIONE 02 -- INIZIO LEZIONE 03
\section{Numero di stati}
Dato un linguaggio riconoscibile da un automa a stati finiti, quanti stati sono necessari per riconoscerlo?

Nell'ambito degli automi deterministici definiamo il concetto di \textbf{distribuibilità} tra stringhe rispetto al linguaggio $L \subseteq \Sigma^*$.
Due stringhe qualunque $x, y \in \Sigma^*$ sono distinguibili per $L$ se 
$$ \exists z \in \Sigma^* (xz \in L \wedge yz \not \in L) \vee (xz \not \in L \wedge yz \in L) $$

Molto semplicemente se $x$ e $y$ sono distinguibili queste ci porteranno in due stati diversi.
Da questi due stati diversi si può costruire una $z$ che in un caso ci porta in uno stato finale e nell'altro no.
Quindi in un automa deterministico se due stringhe sono distinguibili, non ci possono mandare nello stesso stato.

\begin{teorema}
	Sia $L \subseteq \Sigma^*$ e $X \subseteq \Sigma^*$ tale che ogni coppia di stringhe in $X$ è distinguibile in rispetto ad $L$.
 	Allora ogni DFA che accetta $L$ deve avere almeno $\# X$ stati (cardinalità di $X$).
\end{teorema}
\begin{proof}
	Supponiamo che $X = \{x_1, x_2, \dots, x_k \}$.
	Sia $\mathcal{A} = \langle Q, \Sigma, \delta, q_0, F \rangle$ un DFA per $L$.
	Sia $\forall i \in 1, \dots, k \mid p_i = \delta(q_0, x_i)$.
	Se $\# Q < k$ allora esistono due stati uguali
	$$ \exists i, j \in 1, \dots, k \mid i \neq j \wedge p_1 = p_j $$
	Ma $x_i, x_j$ sono distinguibili, quindi
	$$ \exists z \mid (xz \in L \wedge yz \not \in L) \vee (xz \not \in L \wedge yz \in L) $$
	Ma questo $z$ non può esistere, quindi la $\# Q$ deve per forza essere $\leq k$.
\end{proof}
Questo può essere anche utile per determinare se un linguaggio non può essere definito da un automa a stati finiti.
Infatti se si trova un insieme $X$ infinito, allora il linguaggio non può essere regolare.

\begin{tcolorbox}
	Sia $\Sigma = \{ a, b \}$ e 
	$$L = \{ x \in \Sigma^* \mid \#_a(x) \text{ è pari } \wedge \#_b(x) \text{ è pari } \} $$
	\begin{center}
		\begin{tikzpicture}
			\node[state, initial, accepting] 	(00) {$00$};
			\node[state] 				(10) [right=of 00] {$10$};
			\node[state]				(01) [below =of 00] {$01$};
			\node[state]				(11) [right=of 01] {$11$};

			\path[->] (00)	edge[bend left]	node[above]	{\tiny a}	(10)
					edge[bend left]	node[right]	{\tiny b}	(01)
				  (10)	edge[bend left]	node[right]	{\tiny b}	(11)
					edge[bend left]	node[below]	{\tiny a}	(00)
				  (01)	edge[bend left]	node[left]	{\tiny b}	(00)
					edge[bend left]	node[above]	{\tiny a}	(11)
				  (11)	edge[bend left]	node[left]	{\tiny b}	(10)
					edge[bend left]	node[below]	{\tiny a}	(01);
		\end{tikzpicture}
	\end{center}
	Costruisco $X = \{ \epsilon, a, b, ab, \}$, per vedere che sono distinguibili costruiamo
	\begin{center}
		\begin{tblr}{ c | c c c c }
			z          & $\epsilon$ & a 	     & b          & ab \\
			\hline
			$\epsilon$ &            & $\epsilon$ & $\epsilon$ & $\epsilon$ \\
			a          &            &            & a          & a \\
			b          &            &            &            & b \\
			ab         &            &            &            &
		\end{tblr}
	\end{center}
\end{tcolorbox}
\begin{tcolorbox}
	Dato il linguaggio 
	$$ \mathcal{L}_n = \{ x \in \{ a, b \}^* \mid \text{l'$n$-esimo simbolo da destra di $x$ è una $a$} \} $$
	L'automa si ricorda gli ultimi $n$ simboli, quindi ha $2^n$ stati.
	Mentre l'NFA corrispondente è 
	\begin{center}
		\begin{tikzpicture}
			\node[state, initial] 	(q_0) {$q_0$};
			\node[state] 		(q_1) [right=of q_0] {$q_1$};
			\node[state]		(q_2) [right =of q_1] {$q_2$};
			\node			(dots) [right =of q_2] {$\dots$};
			\node[state, accepting]	(q_k) [right=of dots] {$q_k$};

			\path[->] (q_0)		edge[loop above]	node[above]	{a, b}	()
						edge	node[above]	{a}	(q_1)
				  (q_1)		edge	node[above]	{a}	(q_2)
				  (q_2)		edge	node[above]	{a,b}	(dots)
				  (dots)	edge	node[above]	{a,b}	(q_k);
		\end{tikzpicture}
	\end{center}
	% aggiungo graffa con scritto $n$ volte
	questo ha $n + 1$ stati.

	Possiamo fare meglio per il deterministico?
	Costruiamo $X = \{a, b\}^n = \{ x \in \{a, n\}^* \mid |x| = n \} $.
	Siano $x, y \in X$, con 
	\begin{align*}
		x &= x_1 x_2 \dots x_i \dots x_n \\
		y &= y_1 y_2 \dots y_i \dots y_n 
	\end{align*}
	Visto che $x \neq y$, $\exists i \mid x_i \neq y_i$.
	Supponiamo che $x_i = a$, ed $y_i = b$, quindi
	\begin{align*}
		x &= x_1 x_2 \dots a \dots x_n \\
		y &= y_1 y_2 \dots b \dots y_n 
	\end{align*}
	Ora per distinguere le tue parole basta scegliere $z = \{a, b\}^{i - 1}$, ad esempio $z = a^{i - 1}$, con questa $z$ $xz \in \mathcal{L}$ e $yz \not \in \mathcal{L}$.
	Visto che la cardinalità di $X$ è di $2^n$ allora non possiamo costruire un automa di meno di $2^n$ stati.

	Questo mostra come gli automi non deterministici possano essere molto più compatti.
\end{tcolorbox}
\begin{tcolorbox}
	Se definiamo $\mathcal{L} = \{ w \in \{ a, b\}^* \mid \#_a(w) = \#_b(w) \}$ ha $X$ infinito % ???
\end{tcolorbox}

\subsection{Costruzione coi sottoinsiemi}
\begin{center}
	\begin{tikzpicture}
		\node[state, initial] 	(q_0) {$q_0$};
		\node[state, accepting]	(q_1) [right=of q_0] {$q_1$};

		\path[->] (q_0)		edge[loop above]	node[above]	{a}	()
					edge			node[above]	{a, b}	(q_1)
			(q_1)		edge[loop above]	node[above]	{b}	()
			edge	node[above]	{b}	(q_0);
	\end{tikzpicture}
\end{center}
$\mathcal{A} = \langle Q, \Sigma, \delta, q_0, F \rangle$ NFA definisco $\mathcal{A}^\prime = \langle Q^\prime, \Sigma, \delta^\prime, q_0^\prime, F^\prime \rangle$, con $Q^\prime = 2^Q$ e $q_0^\prime = \{ q_0 \}$.
% \begin{center}
% 	\begin{tikzpicture}
% 		\node[state, initial] 	(q0) {$\{q_0}$};
% 		\node[state] 		(q0q1) {$\{q_0, q_1}$};
% 		\node[state, accepting]	(q1) {$\{q_1}$};
% 		\node[state] 		(empty) {$\varnothing}$};
% 
% 		\path[->] (q0)		edge	node[above]	{a}	(q0q1)
% 					edge	node[above]	{b}	(q1)
% 		(q1)			edge[loop above]	node[above]	{b}	()
% 		edge	node[above]	{b}	(q_0);
% 		% finisco costruzione
% 	\end{tikzpicture}
% \end{center}
Quindi
$$\forall a \in Q^\prime, a \in \Sigma \mid \delta^\prime(\alpha, a) = \bigcup_{q \in \alpha} \delta(q, a) $$
ed
$$ F^\prime = \{ \alpha \in Q^\prime \mid \alpha \cap F \neq \varnothing\} $$

Se l'NFA ha $n$ stati, il DFA ottenuto ha $2^n$ stati.

	Costruiamo ora l'automa $M_n$, o automa di Meyer\&Fisher (1971), non deterministico tale per cui l'automa deterministico corrispondente ha necessariamente $2^n$ stati.
	% \begin{center}
	% 	\begin{tikzpicture}
	% 		\node[state, initial, final] 	(q0) {$q_0$};
	% 		\node[state] 			(q1) [right=of q0]{$q_1$};
	% 		\node[state]			(q2) [right=of q1]{$q_2$};
	% 		\node[state] 			(q3) [right=of q2]{$q_3}$};
	% 		\node[state]			(q_k) [right=of q3] {$q_4$};
	% 
	% 		\path[->] (q0)		edge	node[above]	{a}	(q0q1)
	% 					edge	node[above]	{b}	(q1)
	% 		(q1)			edge[loop above]	node[above]	{b}	()
	% 		edge	node[above]	{b}	(q_0);
	% 		% finisco costruzione
	% 	\end{tikzpicture}
	% 	% per ogni stato la b crea un loop e può riportare l'automa allo stato q_0
	% \end{center}
	Quindi è l'automa $\mathcal{A} = \langle Q = \{0, 1, \dots, n - 1\}, \{a, b \}, \delta, q_0 = 0, F = \{ 0 \} \rangle$ con $\delta$ definita come
	\begin{align*}
		\delta(i, a) &= (i + 1) \mod n \\
		\delta(i, b) &= \begin{cases} 0 	& \text{se } i = 0 \\ \{i, 0\} & \text{altrimenti} \end{cases}
	\end{align*}
	Sia $S \subseteq \{0, \dots, n - 1 \}$, definisco $w_s$ come
	$$ w_s = \begin{cases} b & \text{se } S = \varnothing \\ a^i & \text{se } S = \{ i \} \\ a^{e_k - e_{k - 1}}ba^{e_{k - 1} - e_{k - 2}}b\dots ba^{e_2 - e_1}ba^{e_1} & \text{se } S = \{e_1, e_2, \dots, e_k \} \text{ con } k \geq 2, e_1 < e_2 < \dots < e_k \end{cases} $$
	Ad esempio prendiamo $S = \{1, 3, 4\}$, la $w_S$ corrispondente è
	$$w_S = aba^2ba = abaaba $$
	% disegno l'albero di computazione per questa parola
	Mostriamo che l'insieme di stati raggiunto da questa stringa è esattamente $\{1, 3, 4\} = S$.
	Mostriamo anche per $S = \{0, 2\}$, $w_S = a^2b$.
	O ancora $S = \varnothing$, $w_S = b$.

	% proprietà 1
	\begin{proprietà}
	\`E possibile dimostrare che $\forall S \subseteq \{0, \dots, n - 1\} \mid \delta(0, w_S) = S$. \todo{Proprietà 1}
	\end{proprietà}
	% proprietà 2
	\begin{proprietà}
	Siano $S, T \subseteq \{0, \dots, n - 1\}$, se $S \neq T$ allora $w_S$ e $w_T$ sono distinguibili. \todo{Proprietà 2}
	\end{proprietà}
	\begin{proof}	% proprietà 2
		Se $S \neq T$ esiste un numero che appartiene a $S$ ma non a $T$, chiamiamolo $x \in S \ T$ \todo{Complemento, non ricordo il simbolo}.
		Abbiamo che
		\begin{align*}
			\delta(0, w_S) &= S \\
			\delta(0, w_T) &= T \\
		\end{align*}
		Dato uno stato $y$ ci si mettono $a^{n - y}$ per arrivare allo stato finale.
		Quindi da $w^S$ ci si mettono $a^{n - x}$ per arrivare ad uno stato finale.
		$$ 0 \overset{w_S}{\rightsquigarrow} x \overset{a^{n - x}}{\rightsquigarrow} 0 $$
		Sia $y \in T$, se $y \neq x$, allora
		$$ 0 \overset{w_T}{\rightsquigarrow} y \not \overset{a^{n - x}}{\rightsquigarrow} 0 $$
		Allora $a^{n - x}$ distingue $S$ e $T$.

		Ed abbiamo che $X = \{ w_S | S \subseteq \{0, \dots, n - 1 \} \}$ è foramto da stringhe a coppie di stringhe % ???
		Quindi ogni DFA ha almeno $2^n$ stati.
	\end{proof}
	Questo è il linguaggio (più o meno) delle stringhe che hanno un suffisso della forma $bx$ dove $\#_a(x)$ è multiplo di $n$.

\subsection{$\epsilon$-mosse}
Possiamo, in certi casi, introdurre delle transizioni sulle parole vuote.
\begin{tcolorbox}
	Supponiamo di volere l'automa che riconosce i numeri con segno (opzionale)
	\begin{center}
		\begin{tikzpicture}[remember picture]
			\node[state, initial] 	(q0) {$q_0$};
			\node[state] 		(q1) [right=of q0] {$q_1$};
			\node[state, accepting]	(q2) [right=of q1] {$q_2$};
	
			\path[->] (q0)		edge	node[above]	{\tiny $+,-,\epsilon$}	(q1)
				  (q1)		edge	node[above]	{\tiny $0, \dots, 9$}		(q2)
				  (q2)		edge[loop above] node[above]	{\tiny $0, \dots, 9$}	();
		\end{tikzpicture}
	\end{center}
	Oppure supponiamo di volere l'automa per $a^lb^mc^n$
	\begin{center}
		\begin{tikzpicture}[remember picture]
			\node[state, initial] 	(q0) {$q_0$};
			\node[state] 		(q1) [right=of q0] {$q_1$};
			\node[state, accepting]	(q2) [right=of q1] {$q_2$};
	
			\path[->] (q0)		edge[loop above]	node[above]	{a} ()
						edge			node[above]	{$\epsilon$}	(q1)
				  (q1)		edge[loop above]	node[above]	{b}		()
				  		edge			node[above]	{$\epsilon$}	(q2)
				  (q2)		edge[loop above]	node[above]	{$c$}		();
		\end{tikzpicture}
	\end{center}
\end{tcolorbox}
Generalmente se ho
$$ q_0 \overset{\epsilon}{\rightarrow} q_1 \overset{a}{\rightarrow} q_2 \overset{\epsilon}{\rightarrow} q_3 $$
questo diventa
$$ q \overset{a}{\rightarrow} q_3 $$
Inoltre se uno stato ha un cammino formato da $\epsilon$-mosse fino ad uno stato finale, esso stesso è finale.
Questo necessariamente introduce non determinismo.

\subsection{Stati iniziali multipli}
Questa è un'altra estensione che genera non determinismo.
\begin{center}
	\begin{tikzpicture}[remember picture]
		\node[state, accepting, initial] 	(q0) {$q_0$};
		\node[state, accepting, initial] 	(q1) [below=of q0] {$q_1$};

		\path[->] (q0)		edge[loop above]	node[above]	{b} ()
			  		edge[bend left]		node[right]	{a} (q1)
			  (q1)		edge[bend left]		node[left]	{b} (q0);
	\end{tikzpicture}
\end{center}
% non mi ricordo di che linguaggio fosse
Per renderli deterministici si crea lo stato iniziale che è l'insieme dei diversi stati iniziali.

Se si prende un automa con stati iniziali multipli, ma transizioni deterministiche, si può comunque avere un gap esponenziale tra il numero di stati del'NFA e del DFA.

\chapter{Espressioni regolari}
Date tre espressioni regolari
$$ (a*b*)* \equiv (a + b)* \equiv b*(ab*)* $$
Sono tre linguaggi equivalenti.

Ci si può chiedere qual è il numero massimo di star innestate necessarie per generare un linguaggio.
Introduciamo l'\textit{altezza di star}, o star height $h$, come il numero massimo di star innestate.
Definita nel seguente modo
$$ h(E) = 
\begin{cases}
	0 & \text{se } E = \varnothing \vee E = \epsilon \vee E \in \Sigma^* \\
	\max(h(E^\prime), h(E^{\prime\prime})) & \text{se } E = E^\prime \cdot E^{\prime\prime} \vee E = E^\prime + E^{\prime\prime} \\
	1 + h(E^\prime) & \text{se } E = E^\prime*
\end{cases}
$$
Definiamo poi la minima altezza per un linguaggio $L \subseteq \Sigma^*$ come
$$ h(L) = \min \{ h(E) \mid \text{ $E$ denota $L$ } \} $$

\begin{teorema}[Dejan, Schutzenberger]
	$$\forall q > 0 \exists W_q \subseteq \{a, b\}^* \mid h(W_q) = q $$
	Equesto è definito come
	$$ W_q = \{ w \in \{a, b\}^* \mid \#_a(w) \equiv \#_b(w) \mod 2^q \} $$
\end{teorema}

La misura dei cicli è collegata al numero di cicli in un automa.
\begin{tcolorbox}
	% fig 07.1
	Per $q = 1, 2, \dots$, definiamo $W_q$ ed abbiamo che
	$$ W_2 = ((ab + ba) + (aa + bb)(ab + ba)^*(bb + aa))^* $$
\end{tcolorbox}

\begin{nota}
	Se $|\Sigma| = 1$, allora $\forall L \subseteq \Sigma^* \mid h(L) \leq 1$.
\end{nota}

\section{Espressioni regolari estese}
Supponiamo di avere il linguaggio
$$ L = \{ w \in \{a, b\}^* \mid \#_a(w) \text{ è pari } \wedge \#_b(w) \text{ è pari } \} $$

Costruiamo prima il linguaggio con solo le $a$ pari
$$ (b + ab*a)* $$
e con solo le $b$ pari
$$ (a + ba*b)* $$
intuitivamente il linguaggio $L$ sarebbe, se avessimo l'intersezione, 
$$ (b + ab*a)* \cap (a + ba*b)* $$
Intoduciamo quindi le \textbf{espressioni regolari estese}.

Definiamo queste come le espressioni con, oltre alle operazioni di concatenazione, unione e star; l'intersezione ($\cap$) e il complemento ($E^-$).
Introducendo il complemento possiamo ottenere l'intersezione attraverso l'unione.

\begin{tcolorbox}
	Dato il linguaggio su $\Sigma = \{a, b\}$ delle stringhe con 3 $a$ consecutive
	% fig 07.2
	vogliamo riconoscere il suo complemento (senza 3 $a$ consecutive).
	Questo diventa semplicemente
	$$ \overline{\overline{\varnothing} a a a \overline{\varnothing}} $$
	Quindi ora senza usare la star possiamo esprimere anche linguaggi infiniti.
\end{tcolorbox}
Si può definire il problema di trovare il l'altezza di star minima ($eh$) nelle espressioni regolari estese.
Questo è un problema aperto.
Si sa che valgono
\begin{align*}
	\exists L \mid eh(L) = 0 \\
	\exists L \mid eh(L) = 1 
\end{align*}

\chapter{Operazioni sugli automi a stati finiti}
In questo capitolo vedremo una serie di operazioni sugli automi a stati finiti e soprattutto il loro impatto sulla dimensione dell'automa.
Abbiamo già visto alcune operazioni tra automi nell'ambito delle espressioni regolari, specificamente 
\begin{center}
	\begin{tblr}{c c c}
		op & DFA & NFA \\
		\hline
		unione & $n^\prime \cdot n^{\prime\prime}$ & $1 + n^\prime + n^{\prime\prime} $ \\
		concatenazione & $n^\prime \cdot 2^{n^{\prime\prime}} $ & $n^\prime + n^{\prime\prime}$ \\
		star           & $1 + 2^{n^\prime} $ & $n^\prime + 1$ \\
		intersezione   & $n^\prime \cdot n^{\prime\prime}$ & $ n^\prime \cdot n^{\prime\prime} $ \\
		complemento    & $n^\prime$ & $2^{n^\prime}$
	\end{tblr}
\end{center}
L'intersezione di automi funziona anche con NFA con la stessa costruzione dei DFA.

\section{Reversal}
Definiamo l'operazione di inversione o reversal.
Data una stringa 
$$ w = a_1 a_2 \dots a_n $$
definiamo
$$ w^R = a_n \dots a_2 \dots a_1 $$
Per DFA e NFA basta sostituire stati iniziali con i finali e viceversa, e invertire le transizioni.
Però può accadere che un DFA invertito diventi non deterministico, infatti 
\begin{itemize}
	\item se l'automa ha diversi stati finali, questi diventeranno stati iniziali multipli
	\item se uno stato ha più frecce entranti con lo stesso simbolo, queste diventeranno uscenti con lo stesso simbolo
\end{itemize}
\begin{tcolorbox}
	Ad esempio il linguaggio il cui terzo simbolo è una $a$ è un semplice DFA, mentre il linguaggio il cui terzultimo simbolo è una $a$ abbiamo visto che ha $2^3$ stati. % TODO: link

	In generale se chiediamo l'$n$-esimo simbolo da destra servono $n + 1$ stati, mentre se chiediamo l'$n$-esimo simbolo da sinistra ne servono $2^n$.
\end{tcolorbox}

Mentre per le espressioni regolari, per la costruzione del reversal, basta solo invertire l'espressione.
O più precisamente
\begin{itemize}
	\item nei casi in cui $E = \varnothing$ o $E = \epsilon$ non cambia niente
	\item nel caso in cui $E \in \Sigma^*$ questo diventa $E^R$
	\item nel caso in cui $E = E^\prime + E^{\prime\prime}$, abbiamo $E^{\prime R} + E^{\prime\prime R}$
	\item nel caso in cui $E = E^\prime \cdot E^{\prime\prime}$, abbiamo $E^{\prime\prime R} \cdot E^{\prime R}$
	\item nel caso in cui $E = E^{\prime}*$, abbiamo $(E^{\prime R})*$
\end{itemize}

\section{Shuffle}
Definiamo l'operazione di \textit{shuffle}
$$ sh(x, y) $$
che produce ogni possibile interpolazione dei suoi argomenti 
$$ sh(ab, cd) = \{ abcd, acdb, acbd, cabd, \dots  \} $$
Cioè vale che se
$$ sh(w', w'') = w $$
allora se definiamo che per due simboli $x_i, x_j \in \Sigma$ in posizioni $i$ e $j$ vale $x_i \prec_w x_j$ se $i < j$
$$ \forall x, y \in w \mid x \prec_{w'} y \Rightarrow x \prec_w y \wedge x \prec_{w''} y \Rightarrow x \prec_w y $$
Dati due linguaggi, abbiamo che
$$ sh(L^\prime, L^{\prime\prime}) = \bigcup_{x \in L^\prime, y \in L^{\prime\prime}} sh(x, y) $$
cioè lo shuffle di due linguaggi è definito come l'unione degli shuffle di tutte le loro stringhe.

% Supponiamo che $L^\prime \subseteq \Sigma^{\prime *}$ e $L^{\prime\prime} \subseteq \Sigma^{\prime\prime *}$ e $\Sigma^\prime \cap \Sigma^{\prime\prime} = \varnothing$.
% Esercizio, costruisco questo automa, e.g. due sottoinsiemi

\section{Quoziente}
Definiamo l'operazione di quoziente tra due linguaggi
$$ L_1 \setminus L_2 = \{ x \in \Sigma^* \mid \exists y \in L_2 \; xy \in L_1 \} $$
Quindi le stringhe di $L_1$ in cui ho tolto un suffisso di $L_2$.
\begin{tcolorbox}
	Ad esempio
	\begin{align*}
		L_1 &= a^+bc^+ \\
		L_2 &= bc^+  \\
		L_3 &= c^+
	\end{align*}
	Ho che
	$$ L_1 \setminus L_2 = a+ $$
	e che
	$$ L_1 \setminus L_3 = a+bc* $$
\end{tcolorbox}

\begin{tcolorbox}
	Ad esempio da un linguaggoi su $\Sigma = \{0, 1\}$, voglio togliere tutti gli zeri finali
	$$(L \setminus 0*) \cap ((0 + 1)*1 + \epsilon)$$
\end{tcolorbox}
\begin{fatto}
	Dato $R$ linguaggio regolare ed $L$ un linguaggio qualsiasi, allora
	$$ R \setminus L $$
	è ancora un linguaggio regolare.
\end{fatto}
\begin{proof}
	Dato
	$$ \mathcal{A} = \langle Q, \Sigma, \delta, q_0, F \rangle $$
	l'automa per $R$.

	Definisco 
	$$ \mathcal{A}^\prime = \langle Q, \Sigma, \delta, q_0, F^\prime = \{ q \mid \exists y \in L . \delta(q, y) \in F \} \rangle $$
	come l'automa per $R \setminus L$.

	Non è detto che sappiamo trovare una $y \in L$.
	Quindi questa dimostrazione non è costruttiva per tutti i linguaggi.
\end{proof}

\subsection{Morfismo}
Il linguaggio delle parentesi bilanciate (Dick) non è regolare.
Dato un linguaggio di programmazione, lo possiamo ricondurre al linguaggio di Dick, ad esempio
\begin{align*}
	\{ &\rightarrow ( \\
	\} &\rightarrow ) \\
	a  &\rightarrow \epsilon 
\end{align*}
Con questa operazione stiamo definendo un morfismo tra due alfabeti
$$ h : \Sigma \rightarrow \Delta^* $$
Questa funzione è facilmente estendibile alla stringa
$$
\begin{cases}
	h(\epsilon) = \epsilon \\
	h(xa) = h(x) h(a) 
\end{cases}
$$
e al linguaggio 
$$ h(L) = \{ h(w) \mid w \in L \} $$

\begin{tcolorbox}
	Dati $\Sigma = \{a, b \}$, e $\Delta = \{0, 1\}$, definiamo il morfismo $h$ come
	\begin{align*}
		h(a) &= 01 \\
		h(b) &= 1 
	\end{align*}
	E vale che se abbiamo
	$$ L = a^*b$$
	questo corrisponde a
	$$ (01)^*1 $$
	E agli automi corrisponde
	\begin{center}
		\begin{tikzpicture}[remember picture]
			\node[state, initial] 	(q0) {$q_0$};
			\node[state, accepting]	(q1) [right=of q0] {$q_1$};

			\path[->] (q0)	edge[loop above] node[above]	{a}	()
					edge		 node[above]	{b}	(q1);
		\end{tikzpicture}
	\end{center}
	diventa
	\begin{center}
		\begin{tikzpicture}[remember picture]
			\node[state, initial] 	(q0) {$q_0$};
			\node[state]		(q1) [right=of q0] {$q_1$};
			\node[state, accepting]	(q2) [below=of q0] {$q_2$};

			\path[->] (q0)	edge[bend left]	node[above]	{0}	(q1)
					edge		node[left]	{1}	(q2)
				  (q1)	edge[bend left]	node[below]	{1}	(q0);

		\end{tikzpicture}
	\end{center}
	Quindi basta sostituire ogni stato che riconosce un simbolo $\sigma \in \Sigma$ con gli stati corrispondenti che riconoscerebbero $h(\sigma) \in \Delta$.
\end{tcolorbox}

\subsection{Sostituzione}
In questa versione si sostituiscono le lettere di un linguaggio con un altro linguaggio, quindi
$$ s : \Sigma \rightarrow 2^{\Delta^*} $$
Nel caso di espressioni regolari questo è molto semplice, basta sostituire ogni occorrenza del simbolo $\sigma \in \Sigma$ nell'espressione con $s(\sigma)$.
\begin{tcolorbox}
	Ad esempio dato $ \Sigma = \{a, b\}$ e $\Delta = \{0, 1\}$, posso dire che
	\begin{align*}
		s(a) &= (01 + 0)^* \\
		s(b) &= 1
	\end{align*}
	e
	$$ L = (ab)^*(a + b) $$
	ottengo
	$$ ((01 + 0)^*1)^*((01 + 0)^* + 1) $$
\end{tcolorbox}

Nel caso di automi, basta sostituire una transizione con l'intero automa corrispondente a quel simbolo, quindi:
\begin{tcolorbox}
	Ad esempio dato l'automa
	\begin{center}
		\begin{tikzpicture}[remember picture]
			\node[state] 	(qi) {$q$};
			\node[state]	(qj) [right=of qi] {$p$};

			\path[->] (qi)	edge	node[above]	{a}	(qj);
		\end{tikzpicture}
	\end{center}
	questo verrebbe trasformato in 
	\begin{center}
		\begin{tikzpicture}[ remember picture
				   , SQUIGGLY/.style={->
				  		     , decorate
				                     , decoration={snake,amplitude=.4mm,segment length=2mm,post length=1mm}},
				   ]

			\node[state]	(q0)	at (0, 0)	{};
			\node[state]	(qa)	at (2, 1)	{};
			\node		(qb)	at (2, 0)	{$\vdots$};
			\node[state]	(qc)	at (2, -1)	{};
			\node[state]	(q)	[left=of q0, xshift=-1cm]	{$q$};
			\node[state]	(p)	[right=of qb, xshift=1cm]	{$p$};

			\draw (-1, -2)	rectangle ++(4, 4);

			\node at (1, 2.5)	{$s(\text{a})$};

			\draw[SQUIGGLY, bend left]  (q0) to (qa);
			\draw[SQUIGGLY, bend right] (q0) to (qc);

			\draw[->] 	      (q)	to node[above]	{$\epsilon$}	(q0);
			\draw[->, bend left]  (qa)	to node[above]	{$\epsilon$}	(p);
			\draw[->, bend right] (qc)	to node[below]	{$\epsilon$}	(p);

		\end{tikzpicture}
	\end{center}
\end{tcolorbox}
Se le sostituzioni sono ancora regolari, allora il linguaggio risultante è regolare.

\begin{tcolorbox}
	Dato un linguaggio $L \subseteq \{a, b\}^*$, voglio creare il linguaggio $L^\prime$ in cui ogni $a$ e $b$ è circondata da un numero arbitrario di $c$, questo è
	\begin{align*}
		s(a) &= c^* a c^* \\
		s(b) &= c^* b c^*
	\end{align*}
	Manca però la stringa vuota, quindi abbiamo
	$$ 
	L^\prime = 
	\begin{cases}
		s(L) & \text{se } \epsilon \not \in L \\
		s(L) \cup c^* &\text{altrimenti}
	\end{cases}
	$$
	Questo povtevamo farlo anche con lo shuffle, quindi
	$$ L^\prime = sh(L, c^*) $$
\end{tcolorbox}

\section{Cycle}
Definiamo l'operazione di \textit{cycle}, come
$$ cycle(L) = \{ yx \mid xy \in L \} $$
ad esempio, se
$$ L = a^*b^* $$
$$ cycle(L) = a^*b^*a^* + b^*a^*b^* $$

Dato un automa che riconosca $L$, possiamo costruire un automa che riconosca $cycle(L)$?.
% fig 07.5



\end{document}
